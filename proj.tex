%the sources and quality of evidences consulted

% justificativa: relevância do problema
\subsection{Justificativa}
\label{ssec:just}

% metodologia: como 
% procedimentos experimentais
% data gathered 
% controls
% statistical methods
\subsection{Metodologia}
\label{ssec:metod}


% risco tecnológico
% ADAPTAR PARA O PROJETO EM QUESTÃO
\subsection{Análise preliminar de riscos}
\label{ssec:risco}
A seguir são apontados os riscos tecnológicos antevistos no projeto e as estratégias a serem adotadas para mitigá-los: 
\begin{enumerate}
	% manter, melhorar
	\item Indisponibilidade de conjuntos de dados adequados para treinamento do sistema de reconhecimento de atividades: prioridade no planejamento de procedimento experimental, realização de setup piloto, e submissão de proposta de coleta de dados ao comitê de ética em pesquisa ainda no primeiro semestre de execução do projeto. Outro fator que pode comprometer a obtenção de dados é a atual situação de pandemia de COVID-19. Nesse sentido, o protocolo experimental a ser enviado ao comitê de ética deve incluir medidas de segurança de conformidade com as recomendações pertinentes.
 	% manter, melhorar, adaptar	
	\item Perda de dados experimentais: um dos elementos de maior valor agregado no projeto são os vídeos utilizados para treinamento e validação do sistema de reconhecimento de atividades. Isso se deve ao esforço necessário para fazer com que pessoas se disponham a participar dos experimentos de coleta de dados e também ao custo investido na inspeção de cada vídeo no processo de rotulação de atividades e edição. Uma estratégia automática e estrita de backup, com uso de redundâncias em relação ao número e frequência de cópias, deve ser estabelecida para manter sua  probabilidade de ocorrência sob controle.
	% mater, melhorar, adaptar
	\item Perda de código produzido: o produto mesmo do projeto consiste em softwares (sistema de reconhecimento de atividades, aplicativo mobile e aplicativo web). A perda de código provocada por fatores acidentais (queima de dispositivos de armazenamento por surtos de energia) ou intencionais (ataques cibernéticos) pode acarretar atrasos nas entregas do projeto ou mesmo inviabilizar a concretização de seu objetivo. A mitigação desse risco requer o uso de equipamentos de nobreak (sistemas de proteção a surtos) e o armazenamento de código em repositórios distribuídos (local e em nuvem) com controle de acesso para evitar uso não autorizado.
	% manter, melhorar, adaptar
	\item Indisponibilidade de recursos de software e hardware suficientes para o desenvolvimento da proposta: a alta cambial do Dólar em relação ao Real pode levar à impossibilidade de aquisição de hardware e software especificados inicialmente na etapa de elaboração da proposta. Os meios de mitigação envolvem: considerar a variação do câmbio nos últimos 6 meses por ocasião da elaboração do orçamento do projeto; priorização e presteza nas atividades de compra de equipamentos e programas de computador, tão logo haja liberação de recursos; e contratação de recursos de computação em nuvem no início da etapa de desenvolvimento.
	% manter, melhorar, adaptar
	\item Violação de confidencialidade: as informações privadas de participantes dos experimentos de coleta de dados para treinamento e validação do sistema proposto, bem como as de futuros usuários, podem ser hackeadas. A redução desse risco envolve em: jamais associar informações sensíveis aos vídeos que farão parte do conjunto de dados do projeto; não reter informações pessoais que permitam identificação de indivíduos participantes dos experimentos de coleta de dados; estabelecer procedimentos automáticos e obrigatórios de anonimização nos vídeos (borramento de faces e eliminação, nos arquivos, dos metadados com informação de localização, ip de câmeras, etc.); condicionar a participação em experimentos de coleta de dados de pessoas que tenham assinado termo de consentimento livre e esclarecido (TCLE), em procedimento aprovado por comitê de ética em pesquisa.
\end{enumerate}


% grau de inovação
% ADAPTAR PARA O PROJETO EM QUESTÃO
\subsection{Grau de inovação}
\label{ssec:inova}
O sistema proposto é composto por: software para reconhecimento automático de atividades humanas, voltado para a detecção de situações de risco à integridade física de pessoas com mobilidade reduzida (idosos, pessoas com deficiência, acidentados, etc.), tais como quedas, acidentes domésticos, e crises de saúde manifestadas for parâmetros de movimentação anômalos; e aplicativos mobile e web de acesso aos resultados decorrentes desse monitoramento. 
Os produtos existentes no mercado com maior semelhança à proposta são: câmeras de vigilância para monitoramento de pessoas em ambiente doméstico; sistemas para prevenção contra afogamento; e monitoramento de atividades por sensores. 

As câmeras de vigilância são conectadas à internet por meio de rede wifi ou celular, alguns modelos possuem visão noturna, controle de posição, e canal de áudio bidirecional. Dispõem ainda de aplicativo que permite controle de direcionamento, visualização de vídeo capturado em tempo real, ativação de microfone e alto falante embutidos, permitindo interação por áudio com o ambiente monitorado. Há soluções que oferecem também serviço de armazenamento de vídeo em nuvem mediante assinatura. Tais sistemas não fazem qualquer tipo de análise automática do vídeo capturado. A reação a situações de risco é feita, via de regra, tardiamente a partir da revisão do vídeo registrado em nuvem, salvo se alguém estiver observando o ambiente monitorado no instante de ocorrência, ou se a pessoa monitorada tiver possibilidade de se comunicar pelo canal de áudio da própria câmera. 

Por sua vez, as soluções para prevenção contra afogamentos em piscinas consistem em: câmeras, para monitorar tanto fora como embaixo d’água; sistemas de hardware e software para análise de vídeo; e interface de visualização/alerta. As imagens capturadas são analisadas em tempo real, procurando detectar duas situações: pessoas que estejam sem movimento; pessoas cuja movimentação seja incompatível com o que se poderia chamar de nado. Nessas duas situações, alertas sonoros e visuais são enviados para um equipamento que fica com o guarda-vidas em piscinas públicas ou responsáveis e familiares para piscinas em residências.

O monitoramento de atividades por sensores pode ser realizado mediante a instalação de sensores sem fio no ambiente e/ou objetos de interesse. Por exemplo, a verificação da regularidade de uso de medicação pode ser inferida por sensor em caixa/gaveta de medicamentos que registra sua abertura ou não o que pode ser comparado com um cronograma. Da mesma forma, pode-se identificar se a rotina diária de atividades foi alterada mediante sensores posicionados em controle remoto da TV, cama, porta, xícara, etc. Uma variante deste sistema é o uso de vestíveis, ou seja, sensores em calçados, camisas, calças, etc., que permitiriam registrar a movimentação de indivíduos que estivessem portanto as peças de roupa monitoradas. O principal inconveniente dessa abordagem é a necessária adesão do indivíduo ao uso dos objetos e roupas contendo sensores, o que pode ser um problema em pessoas já idosas com problemas relacionados à memória.

Assim, o sistema proposto concentra-se em etapas de processamento e análise que são posteriores à aquisição de vídeo, fazendo uso das câmeras de vigilância de ambientes domiciliares já existentes no mercado, mas agregando inteligência ao monitoramento. Diferente da prevenção de afogamentos, a detecção de quedas, acidentes domésticos e crises de saúde apresenta maior desafio técnico em razão da maior complexidade do ambiente monitorado e das atividades de interesse. A abordagem por visão computacional, agregando áudio e infravermelho ao vídeo convencional dispensa o uso de roupas e objetos específicos, podendo ser utilizada em um número maior de situações sem que haja necessidade de aumentar o número de equipamentos para realizar o monitoramento. Os autores da proposta acreditam que a solução em software proposta, apesar de fazer uso de dispositivos de prateleira para aquisição, processamento e armazenamento de vídeo, aporta inovação importante no mercado crescente (tanto em âmbito nacional quanto mundial) de sistemas de promoção saúde, proteção e bem-estar direcionados para terceira idade. Até o presente momento não se tem conhecimento de sistema comercial com as mesmas características da proposta apresentada.

Não obstante haver trabalhos científicos recentes sobre reconhecimento automático de atividades voltado para a aplicação pretendida nesta proposta,  os autores deste projeto se propõem a avançar o estado da arte na área no que diz respeito ao compromisso  qualidade versus complexidade computacional para viabilizar o produto em contextos práticos.

% grau de maturidade tecnológica atual, pretendido e os meios para chegar lá
% ADAPTAR PARA O PROJETO EM QUESTÃO
\subsection{Maturidade tecnológica}
\label{ssec:trl}
Os trabalhos científicos publicados sobre sistemas de reconhecimento automático de atividades humanas baseados em visão computacional e aprendizado profundo que poderiam ser usados para detectar situações de risco à integridade física de idosos e pessoas com limitações motoras (por exemplo, trabalhos em predição/detecção de quedas, estimação de parâmetros de marcha, e monitoramento de atividades diárias simples) reportam testes de desempenho obtidos em ambiente laboratorial, configurando nível de maturidade tecnológica 4 (TRL 4). A proposta envolve o reconhecimento multimodal de atividades, onde o vídeo obtido por câmeras convencionais (faixa visível) é acrescido de informações de áudio e de câmeras térmicas, visando melhoria de desempenho em situações práticas de operação. 
Pretende-se chegar à demonstração de protótipo do sistema proposto em ambiente operacional, ou seja, atingir nível de maturidade tecnológica 7 (TRL 7). Para tanto, faz-se necessário percorrer uma sequência de etapas para realizar a transição de maturidade tecnológica almejada.
A primeira dessas etapas é a reprodução do estado da arte em reconhecimento de atividades humanas usando uma rede genérica pré-treinada, cujos parâmetros deverão ser reajustados mediante conjunto de treinamento específico para as atividades de interesse. Para a formação de tal conjunto de dados, deve-se selecionar clips (trechos curtos de vídeo) a partir de bases disponíveis gratuitamente na internet, e em repositórios públicos como o Youtube. A principal entrega nesta etapa é um rede neural profunda com desempenho superior ao da rede genérica original em relação à acurácia no reconhecimento das atividades de interesse, e que servirá de base para os próximos desenvolvimentos. 
Em seguida, ainda usando repositórios e recursos disponíveis gratuitamente na internet, será acrescentado na rede de reconhecimento um ramo destinado ao processamento de sinal de áudio. A inserção do áudio permitirá que o sistema leve em consideração os sons do ambiente no processo de reconhecimento, combinando-os com as informações visuais. A entrega desta etapa será uma arquitetura que tratará e combinará informações de vídeo convencional e áudio para classificação de atividades. 
Na terceira etapa, agrega-se a informação de câmeras térmicas. Nesse caso, haverá necessidade de formação de conjunto de dados próprio pela indisponibilidade desse tipo de informação associada a sinais de áudio e vídeo em repositórios públicos. O ambiente considerado aqui ainda deve ser controlado, porém com características próximas às do ambiente operacional (ambiente relevante), para que se possa avaliar como melhor agregar a imagem térmica ao sistema de classificação já utilizado. Essa é uma etapa que está no caminho crítico da execução da proposta e que depende da aprovação de procedimento de coleta de dados por comitê de ética em pesquisa, e também da aquisição de equipamentos com recursos do projeto. Nessa etapa, duas são as entregas: o próprio conjunto de dados que será único e, portanto, conterá grande valor científico agregado; e uma nova arquitetura de rede profunda para reconhecimento de atividades humanas, cujas entradas compreendem vídeo convencional, áudio e imagem térmica.
A última etapa é semelhante à terceira, com a diferença que o sistema de captura deverá ser instalado em ambiente operacional, ou seja, realizando monitoramento de pessoas idosas em seu domicílio. Esta etapa, na verdade, deve ocorrer em paralelo à etapa anterior: as coletas em ambiente relevante e operacional podem acontecer simultaneamente para evitar atrasos no desenvolvimento do projeto. Uma vez que se tenha uma arquitetura que integre adequadamente a imagem térmica com ganhos de desempenho em ambiente relevante, o sistema pode ser exposto a situações reais para efetuar os reajustes devidos de seus parâmetros e sua arquitetura à nova situação de operação. Novamente, o procedimento experimental de coleta deverá ser aprovado por comitê de ética em pesquisa. A coleta de dados dependerá de voluntários e será impulsionada de maneira importante com o apoio de instituições de amparo ao idoso nesta etapa do projeto. Entregas previstas: conjunto de dados, contendo atividades monitoradas em ambiente operacional (mais uma vez, de grande valor científico agregado); e sistema de reconhecimento de atividades humanas, cujas entradas são sinais de vídeo convencional, imagem térmica e sinais de áudio, funcionando em situação prática. 
Os aplicativos mobile e web para armazenamento e visualização dos resultados obtidos pelo sistema de monitoramento podem ser desenvolvidos por grupo independente da equipe de aprendizado de máquina, responsável pelo sistema de reconhecimento de atividades. A empresa participante da proposta atua na área de produção de software, possuindo equipe especializada em desenvolvimento ágil e qualidade em software. A integração dos aplicativos com a base de dados gerada pelo sistema de aquisição será uma tarefa executada ao longo de todo o período de desenvolvimento do projeto.

% resumo da equipe executora
\subsection{Proponentes}
\label{ssec:equipe}
\begin{enumerate}
	\item Ronaldo de Freitas Zampolo é graduado em Engenharia Elétrica pela Universidade Federal do Pará (1995), obteve os títulos de mestre e doutor em Engenharia Elétrica na Universidade Federal de Santa Catarina (em 1998 e 2003, respectivamente), e realizou estágio de pós-doutoramento na Polytech Nantes (Escola de Engenharia da Universidade de Nantes, França) em 2016. Atualmente é professor associado III da Universidade Federal do Pará, membro do Laboratório de Processamento de Sinais, e docente da Faculdade de Engenharia da Computação e Telecomunicações. É também docente permanente do Programa de Pós-Graduação em Ciências Forenses da Universidade Federal do Sul e Sudeste do Pará (UNIFESSPA). Seus interesses e estudos recentes concentram-se, principalmente, nos temas: avaliação de qualidade visual, modelos de atenção visual, sistemas de rastreamento ocular, análise de vídeo, e reconhecimento de atividades humanas. 

Dentre os projetos em que tem atuado, aqueles de maior afinidade com o tema desta proposta, citamos: 
		\begin{itemize}
			\item ``Sistemas e tecnologias assistivas e de apoio à reabilitação: implementação de sistemas para análise de marcha 2020'': projeto em curso, aprovado no edital PIBEX 2020 (PROEX-UFPA) com recurso para financiamento de bolsa para aluno de graduação; utilização de sistema baseado em redes neurais profundas (OpenPose)\parencite{cao-2019} para localização automática de pontos anatômicos (articulações de quadril, joelho, tornozelo e ponta do pé) por vídeo, para estimação de parâmetros de marcha sem uso de marcadores (atuação como coordenador); 
			\item ``Soluções em `Indústria 4.0' aplicadas à Mineração'': encerrado em 2019, tratava da aplicação de visão computacional na inspeção automática de componentes em vagões ferroviários; projeto financiado pelo ITV-DS, SENAI-PA e CNPq (atuação como membro de equipe de pesquisadores);
			\item ``Desenvolvimento de um sistema de monitoramento e diagnóstico on-line de descargas parciais nos enrolamentos estatóricos de hidrogeradores'': projeto encerrado em 2015, versava sobre o uso de redes neurais para identificação de descargas parciais, contava com financiamento da Eletronorte (atuação como membro de equipe de pesquisadores).
		\end{itemize}
A seguir, segue uma relação de trabalhos publicados em alinhamento com a presente  proposta:
		\begin{itemize}
			\item GONÇALVES, C. L. A. et al. Improving the performance of a SVM+HOG classifier for detection and tracking of wagon components by using geometric constraints. In: WORKSHOP DE APLICAÇÕES INDUSTRIAIS - CONFERENCE ON GRAPHICS, PATTERNS AND IMAGES (SIBGRAPI), 32., 2019, Rio de Janeiro. Anais [...]. Porto Alegre: Sociedade Brasileira de Computação, 2019. p. 230-236. DOI: https://doi.org/10.5753/sibgrapi.est.2019.8336. 
			\item GONÇALVES, L. A., ZAMPOLO, R. F., BARROS, F. B.. A multi-stream network with different receptive fields to assess visual quality. In: Symposium on Knowledge Discovery, Mining and Learning (KDMiLe) (KDMile'19), 2019, Fortaleza (CE). Proceedings of Symposium on Knowledge Discovery, Mining and Learning (KDMiLe) (KDMile'19), 2019. 
			\item OLIVEIRA, R. M. S., ARÁUJO, R. C. F., BARROS, F. J. B., SEGUNDO, A. P., ZAMPOLO, R. F. , FONSECA, W. , DIMITRIEV, V., BRASIL, F. S.. A system based on artificial neural networks for automatic classification of hydro-generator stator windings partial discharges. In: Journal of Microwaves, Optoelectronics and Electromagnetic Applications, vol. 16, n. 3, September, 2017.
			\item GONÇALVES, C. L. A. , ZAMPOLO, R. F.. Classificador SLFN-ELM aplicado à verificação de adulteração de imagens baseada em padrão CFA. In: XXXIV Simpósio Brasileiro de Telecomunicações, 2016, Santarém (PA). Anais do XXXIV Simpósio Brasileiro de Telecomunicações, 2016.
		\end{itemize}	

\item Adriana Rosa Garcez Castro possui graduação pela Universidade Federal do Pará-UFPA (1992), mestrado em Engenharia Elétrica pela Universidade Federal do Pará (1995) e doutorado em Engenharia Elétrica pela Faculdade de Engenharia da Universidade do Porto, Portugal (2004). Atualmente é professora Associada IV da Faculdade de Engenharias Elétrica e Biomédica da UFPA, e vem atuando como pesquisadora e docente permanente no Programa de Pós-graduação em Engenharia Elétrica da UFPA, sendo suas áreas de interesse: Sistemas de Energia, Controle de Processos Eletrônicos e Inteligência Computacional Aplicada.

\end{enumerate}

% apresentação do laboratório
\subsection{Grupos de pesquisa}
\label{ssec:grppesq}
O Laboratório de Processamento de Sinais (LaPS) é um grupo de pesquisa associado à Faculdade de Engenharia da Computação e Telecomunicações,  à Faculdade de Engenharias Elétrica e Biomédica, e ao Programa de Pós-Graduação de Engenharia Elétrica, todos do Instituto de Tecnologia da UFPA. O LaPS desenvolve pesquisa em diversos ramos do processamento de sinais, com ênfase nas seguintes áreas: a) Sistemas de comunicação 5G e além: modelagem de canal de propagação em ondas milimétricas utilizando traçado de raios (ray-tracing), arquiteturas híbridas para separação espacial de usuários (hybrid-beamforming) para sistemas MIMO (múltiplas-entradas, múltiplas-saídas), técnicas de comunicação para sistemas MIMO massivos; b) Análise de sinais bioelétricos em estados normais e alterados (epilepsia, autismos, TDAH): eletroencefalograma, eletromiograma, eletrocardiograma. Sistemas de Neurofeedback. Neuroengenharia; e c) Processamento de imagem e visão computacional: avaliação de qualidade visual, reconhecimento de atividades humanas, inspeção automática de componentes, sistemas de eye-tracking e modelos de atenção visual. Dentre os parceiros do LaPS, destacamos a Coordenação de Aperfeiçoamento de Pessoal de Nível Superior (CAPES), o Conselho Nacional de Desenvolvimento Científico e Tecnológico (CNPq), o Instituto SENAI de Inovação - Tecnologias Minerais (ISI-TM), o Abrigo Especial Calabriano, e a Jambu Tecnologia.

%Dentre as atividades desenvolvidas pelo LaPS diretamente relacionadas ao tema da chamada, podemos citar os projetos “Sistemas e tecnologias assistivas e de apoio à reabilitação: implementação de sistemas para análise de marcha 2020” (em curso), “Dmóvel: Ferramenta móvel para fomento à autonomia de PcDs – 2020” (em curso) e “Implementação de Laboratório de Análise de Marcha de Baixo-Custo para Auxílio na Reabilitação de Crianças Portadoras de Encefalopatia Não-Evolutiva” (encerrado em 2012). Os dois primeiros foram recentemente aprovados em editais de extensão da UFPA, sendo contemplados com bolsa para alunos de graduação. O terceiro, por sua vez, foi aprovado em edital da FAPESPA (Fundação de Apoio à Pesquisa do Estado do Pará). Tais projetos contam com instituição parceira de assistência e saúde voltada a pessoas com deficiência, cujo quadro profissional é formado por médicos, psicólogos, terapeutas ocupacionais, e fisioterapia com longa experiência na área. Apesar das aplicações diferirem daquela desta chamada, a experiência em visão computacional e redes neurais artificiais do LaPS pode ser verificada pela participação em outros dois projetos: “Soluções em “Indústria 4.0” aplicadas à Mineração” (encerrado em 2019, aplicação de visão computacional na inspeção automática de componentes em vagões ferroviários, com parcerias do Instituto Tecnológico Vale, e Instituto SENAI de Inovação, e financiamento pelo ITV-DS, SENAI-PA e CNPq) e “Desenvolvimento de um sistema de monitoramento e diagnóstico on-line de descargas parciais nos enrolamentos estatóricos de hidrogeradores” (encerrado em 2015, usa redes neurais para identificação de descargas parciais, com financiamento pela Eletronorte).

% resumo do orçamento
\subsection{Resumo do orçamento}
\label{ssec:orca}

% impactos: tecnológico, econômico, ambiental, social,
\subsection{Impactos previstos}
\label{ssec:impact}

