%the sources and quality of evidences consulted

% justificativa: relevância do problema
\subsection{Justificativa}
\label{ssec:just}

% metodologia: como 
% procedimentos experimentais
% data gathered 
% controls
% statistical methods
\subsection{Metodologia}
\label{ssec:metod}


% risco tecnológico
\subsection{Análise preliminar de riscos}
\label{ssec:risco}

% grau de inovação
\subsection{Grau de inovação}
\label{ssec:inova}

% grau de maturidade tecnológica atual, pretendido e os meios para chegar lá
\subsection{Maturidade tecnológica}
\label{ssec:trl}

% resumo da equipe executora
\subsection{Resumo da equipe executora}
\label{ssec:equipe}
\begin{enumerate}
	\item Ronaldo de Freitas Zampolo é graduado em Engenharia Elétrica pela Universidade Federal do Pará (1995), obteve os títulos de mestre e doutor em Engenharia Elétrica na Universidade Federal de Santa Catarina (em 1998 e 2003, respectivamente), e realizou estágio de pós-doutoramento na Polytech Nantes (Escola de Engenharia da Universidade de Nantes, França) em 2016. Atualmente é professor associado III da Universidade Federal do Pará, membro do Laboratório de Processamento de Sinais, e docente da Faculdade de Engenharia da Computação e Telecomunicações. É também docente permanente do Programa de Pós-Graduação em Ciências Forenses da Universidade Federal do Sul e Sudeste do Pará (UNIFESSPA). Seus interesses e estudos recentes concentram-se, principalmente, nos temas: avaliação de qualidade visual, modelos de atenção visual, sistemas de rastreamento ocular, análise de vídeo, e reconhecimento de atividades humanas. 

Dentre os projetos recentes em que tem atuado, aqueles de maior afinidade com o tema desta proposta de pesquisa e desenvolvimento, citamos: 
		\begin{itemize}
			\item ``Sistemas e tecnologias assistivas e de apoio à reabilitação: implementação de sistemas para análise de marcha 2020'': projeto em curso, aprovado no edital PIBEX 2020 (PROEX-UFPA) com recurso para financiamento de bolsa para discente de graduação; utilização do sistema baseado em redes neurais profundas (OpenPose) para localização automática de pontos anatômicos de interesse por vídeo, para estimação de parâmetros de marcha sem uso de marcadores (atuação como coordenador); 
			\item ``Soluções em `Indústria 4.0' aplicadas à Mineração'': encerrado em 2019, tratava da aplicação de visão computacional na inspeção automática de componentes em vagões ferroviários; projeto financiado ITV-DS, SENAI-PA e CNPq (atuação como membro de equipe de pesquisadores);
			\iteme ``Desenvolvimento de um sistema de monitoramento e diagnóstico on-line de descargas parciais nos enrolamentos estatóricos de hidrogeradores'': projeto encerrado em 2015, versava sobre o uso de redes neurais para identificação de descargas parciais, contava financiamento da Eletronorte (atuação como membro de equipe de pesquisadores).

A seguir, segue uma relação de trabalhos publicados relacionados com o tema desta proposta:
		\begin{itemize}
			\item GONÇALVES, C. L. A. et al. Improving the performance of a SVM+HOG classifier for detection and tracking of wagon components by using geometric constraints. In: WORKSHOP DE APLICAÇÕES INDUSTRIAIS - CONFERENCE ON GRAPHICS, PATTERNS AND IMAGES (SIBGRAPI), 32., 2019, Rio de Janeiro. Anais [...]. Porto Alegre: Sociedade Brasileira de Computação, 2019. p. 230-236. DOI: https://doi.org/10.5753/sibgrapi.est.2019.8336. 
			\item GONÇALVES, L. A., ZAMPOLO, R. F., BARROS, F. B.. A multi-stream network with different receptive fields to assess visual quality. In: Symposium on Knowledge Discovery, Mining and Learning (KDMiLe) (KDMile'19), 2019, Fortaleza (CE). Proceedings of Symposium on Knowledge Discovery, Mining and Learning (KDMiLe) (KDMile'19), 2019. 
			\item OLIVEIRA, R. M. S., ARÁUJO, R. C. F., BARROS, F. J. B., SEGUNDO, A. P., ZAMPOLO, R. F. , FONSECA, W. , DIMITRIEV, V., BRASIL, F. S.. A system based on artificial neural networks for automatic classification of hydro-generator stator windings partial discharges. In: Journal of Microwaves, Optoelectronics and Electromagnetic Applications, vol. 16, n. 3, September, 2017.
			\item GONÇALVES, C. L. A. , ZAMPOLO, R. F.. Classificador SLFN-ELM aplicado à verificação de adulteração de imagens baseada em padrão CFA. In: XXXIV Simpósio Brasileiro de Telecomunicações, 2016, Santarém (PA). Anais do XXXIV Simpósio Brasileiro de Telecomunicações, 2016.
		\end{itemize}	

\item Adriana Rosa Garcez Castro: Adriana Rosa Garcez Castro possui graduação pela Universidade Federal do Pará -UFPA (1992), mestrado em Engenharia Elétrica pela Universidade Federal do Pará (1995) e doutorado em Engenharia Elétrica pela Faculdade de Engenharia da Universidade do Porto, Portugal (2004). Atualmente é professora Associada IV da Faculdade de Engenharia Elétrica e Biomédica da UFPA, e vem atuando como pesquisadora e docente permanente no Programa de Pós-graduação em Engenharia Elétrica da UFPA, sendo suas áreas de interesse: Sistemas de Energia, Controle de Processos Eletrônicos e Inteligência Computacional Aplicada.

\end{enumerate}

% apresentação do laboratório
\subsection{Grupos de pesquisa}
\label{ssec:grppesq}
O Laboratório de Processamento de Sinais (LaPS) é um grupo de pesquisa associado à Faculdade de Engenharia da Computação e Telecomunicações,  à Faculdade de Engenharias Elétrica e Biomédica, e ao Programa de Pós-Graduação de Engenharia Elétrica, todos do Instituto de Tecnologia da UFPA. O LaPS desenvolve pesquisa em diversos ramos do processamento de sinais, com ênfase nas seguintes áreas: a) Sistemas de comunicação 5G e além: modelagem de canal de propagação em ondas milimétricas utilizando traçado de raios (ray-tracing), arquiteturas híbridas para separação espacial de usuários (hybrid-beamforming) para sistemas MIMO (múltiplas-entradas, múltiplas-saídas), técnicas de comunicação para sistemas MIMO massivos; b) Análise de sinais bioelétricos em estados normais e alterados (epilepsia, autismos, TDAH): eletroencefalograma, eletromiograma, eletrocardiograma. Sistemas de Neurofeedback. Neuroengenharia; e c) Processamento de imagem/ visão computacional: avaliação de qualidade visual, reconhecimento de atividades humanas, inspeção automática de componentes, sistemas de eye-tracking e modelos de atenção visual. Dentre os parceiros recentes do LaPS, relacionamos: CAPES, CNPq, Instituto SENAI de Inovação - Tecnologias Minerais, Abrigo Especial Calabriano, e Jambu Tecnologia.

%Dentre as atividades desenvolvidas pelo LaPS diretamente relacionadas ao tema da chamada, podemos citar os projetos “Sistemas e tecnologias assistivas e de apoio à reabilitação: implementação de sistemas para análise de marcha 2020” (em curso), “Dmóvel: Ferramenta móvel para fomento à autonomia de PcDs – 2020” (em curso) e “Implementação de Laboratório de Análise de Marcha de Baixo-Custo para Auxílio na Reabilitação de Crianças Portadoras de Encefalopatia Não-Evolutiva” (encerrado em 2012). Os dois primeiros foram recentemente aprovados em editais de extensão da UFPA, sendo contemplados com bolsa para alunos de graduação. O terceiro, por sua vez, foi aprovado em edital da FAPESPA (Fundação de Apoio à Pesquisa do Estado do Pará). Tais projetos contam com instituição parceira de assistência e saúde voltada a pessoas com deficiência, cujo quadro profissional é formado por médicos, psicólogos, terapeutas ocupacionais, e fisioterapia com longa experiência na área. Apesar das aplicações diferirem daquela desta chamada, a experiência em visão computacional e redes neurais artificiais do LaPS pode ser verificada pela participação em outros dois projetos: “Soluções em “Indústria 4.0” aplicadas à Mineração” (encerrado em 2019, aplicação de visão computacional na inspeção automática de componentes em vagões ferroviários, com parcerias do Instituto Tecnológico Vale, e Instituto SENAI de Inovação, e financiamento pelo ITV-DS, SENAI-PA e CNPq) e “Desenvolvimento de um sistema de monitoramento e diagnóstico on-line de descargas parciais nos enrolamentos estatóricos de hidrogeradores” (encerrado em 2015, usa redes neurais para identificação de descargas parciais, com financiamento pela Eletronorte).

% resumo do orçamento
\subsection{Resumo do orçamento}
\label{ssec:orca}

% impactos: tecnológico, econômico, ambiental, social,
\subsection{Impactos previstos}
\label{ssec:impact}

