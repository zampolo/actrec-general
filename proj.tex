%the sources and quality of evidences consulted

% justificativa: relevância do problema
\subsection{Justificativa}
\label{ssec:just}

% metodologia: como 
% procedimentos experimentais
% data gathered 
% controls
% statistical methods
\subsection{Metodologia}
\label{ssec:metod}


% risco tecnológico
\subsection{Análise preliminar de riscos}
\label{ssec:risco}

% grau de inovação
\subsection{Grau de inovação}
\label{ssec:inova}

% grau de maturidade tecnológica atual, pretendido e os meios para chegar lá
\subsection{Maturidade tecnológica}
\label{ssec:trl}

% resumo da equipe executora
\subsection{Resumo da equipe executora}
\label{ssec:equipe}
\begin{enumerate}
\item Ronaldo de Freitas Zampolo: graduado em Engenharia Elétrica pela Universidade Federal do Pará (1995), obteve os títulos de mestre e doutor em Engenharia Elétrica na Universidade Federal de Santa Catarina (1998 e 2003, respectivamente), e realizou estágio de pós-doutoramento na Polytech Nantes (Escola de Engenharia da Universidade de Nantes, França, 2016). Atualmente é professor associado III da Universidade Federal do Pará, onde faz parte do corpo docente da Faculdade de Engenharia da Computação e Telecomunicações. Seus interesses e estudos recentes concentram-se, principalmente, nos temas: avaliação de qualidade visual, modelos de atenção visual, sistemas de rastreamento ocular, análise de vídeo, e reconhecimento de atividades humanas. Dentre atividades recentes, relacionadas com o tema da proposta, está sua atuação como coordenador dos projetos ``Sistemas e tecnologias assistivas e de apoio à reabilitação: implementação de sistemas para análise de marcha  2020'' e ``Dmóvel: Ferramenta móvel para fomento à autonomia de PcDs - 2020'', ambos aprovados em editais de extensão na UFPA com financiamento de bolsa para discente.
% incluir publicações (revista em redes neurais), atividades e produção com o isi

%Suas principais contribuições serão na aplicação de técnicas de Processamento de Imagens Digitais e Algoritmos de Visão Computacional para a coleta dos dados de imagens e vídeos necessários, além de auxiliar na escolha das técnicas de processamento baseadas em Machine Learning necessárias.
\end{enumerate}

% apresentação do laboratório
\subsection{Grupos de pesquisa}
\label{ssec:grppesq}


% resumo do orçamento
\subsection{Resumo do orçamento}
\label{ssec:orca}

% impactos: tecnológico, econômico, ambiental, social,
\subsection{Impactos previstos}
\label{ssec:impact}

