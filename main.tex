\documentclass[a4paper,10pt]{article}

%=== preambulo === 
\usepackage[brazilian]{babel}
\usepackage{color,enumerate}
\usepackage{graphicx}
%\usepackage[toc,page]{appendix}
%\usepackage[table]{xcolor}
%\usepackage{booktabs}
\usepackage[T1]{fontenc}
\usepackage[utf8]{inputenc}
\usepackage{subfigure}
\usepackage{fullpage}
\usepackage[affil-it,auth-sc]{authblk}
\usepackage[backend=biber, style=authoryear, doi=false, isbn=false, url=false]{biblatex}
%\addbibresource{bib/?.bib}

\renewcommand\Authand{ e }
\renewcommand\Authands{, e }

\date{Julho de 2020}


\title{Sistemas de visão computacional baseados em aprendizagem profunda para reconhecimento de atividades humanas}

\author[1]{Ronaldo de Freitas Zampolo}
\author[2]{Adriana Rosa Garcez Castro}

\affil[1]{Laboratório de Processamento de Sinais\\
Instituto de Tecnologia\\
Universidade Federal do Pará\\
66075-110 Belém, PA}

\begin{document}
\maketitle

\begin{table}[!th]
\begin{tabular}{|l p{0.7\textwidth}|}
\hline
\textbf{Pesquisadores}        & Ronaldo de Freitas Zampolo \\
			      & Adriana Rosa Garcez Castro\\
%\textbf{Cargo atual}          & Professor associado\\
\textbf{Instituição}          & Universidade Federal do Pará (UFPA)\\
%\textbf{Data de nascimento}   & 10 de fevereiro de 1973\\
%\textbf{Local de nascimento}  & São Paulo, SP, Brasil\\
\textbf{Endereço profissional}& Laboratório de Processamento de Sinais\\
                              & Anexo do LEEC, altos, sala 32\\
                              & Av. Augusto Correa, 1\\
                              & 66070-110 Guamá Belém, PA, Brasil\\
%\textbf{\slang{Endereço residencial}{Home address}}  & Av. Pedro Miranda, 1929\\
%                                		     & Ed. Pollux, ap 602\\
%                                		     & 66085-024 Pedreira Belém, PA, \slang{Brasil}{Brazil}\\
\textbf{Telefone}             & +55 91 3201-7674 \\
%                              & +55 91 3244-3578 (\slang{res.)}{home)}\\
%                              & +55 91 8119-8840 (celular)\\
\textbf{Emails}               & zampolo@ufpa.br\\
			      & adcastro@ufpa.br\\
%	                      & zampolo@ieee.org\\
\textbf{Título do projeto}    & Sistemas de visão computacional baseados em aprendizagem profunda para reconhecimento de atividades humanas\\
\textbf{Área}                 & Reconhecimento de atividades humanas. Visão computacional. Aprendizagem profunda. Redes neurais convolucionais.\\
\textbf{Período de execução}  & 24 meses\\
%\textbf{Supervisor}             & Prof.~Patrick Le Callet\\
%                                & École Polytechnique de Nantes\\
%                                & Université de Nantes\\
%                                & IRCCYN/IVC\\
%                                & Rue Christian Pauc La Chantrerie\\
%                                & BP 50609 44306 Nantes Cedex 3 \slang{França}{France}\\
\hline
\end{tabular}
\end{table}
\newpage
%----------------------------------------------------------------------------

\tableofcontents\newpage


%\begin{abstract}
%reconhecer ações e objetivos
mediante observações dos "agentes" e do ambiente

resposta personalizada em diferentes aplicações

conexões com medicina, psicologia, interação humano-computador, sociologia, segurança, etc.

tipos:
  - baseado em sensores, usuário único: rede de sensores + mineração de dados + aprendizado de máquina, smartphones, consumo diário de energia, atividade ou sedentarismo, covid-19: estimativas de isolamento social,
  - baseado em sensores, multi-usuário: coleta de informações simultâneas de vários usuários por meio de sensores (normalmente vestíveis), covid- 19 (agora sim)
  - baseado em sensores, grupo (qual a diferença em relação ao anterior?): uma possível resposta à pergunta: reconhecer o comportamento de um grupo como entidade, ao invés de identificar o comportamento individual dos membros de um grupo, comportamento do grupo x comportamento dos indivíduos; principais desafios: modelar o papel do comportamento individual na dinâmica coletiva; comportamento de multidões (Círio de Nazaré), gerenciamento de multidões (crowd management) e respostta a situações de emergência.

Abordagens:
  - Através de lógica e reasoning
    - explicações consistentes das ações observadas
    - kautz's : complexidade exponencial no pior caso
    - parece um sistema baseado em regras, tipo árvore de decisão
    - dificuldade para representar incerteza, incapacidade de aprendizado ao longo do tempo
  - Através de inferência prbabilística
    - identificação de atividades diárias, usando sensores (lembrar das discussões sobre monitoramento de idosos, usando vestíveis e câmeras)
    - sensores: RFID, GPS (HMM, Dynamic Bayesian networks)
  - Baseado em mineração de dados:
    - problema de reconhecimento de padrões
    - definição de classes de atividade
  - Baseado em GPS

Uso de sensores:
  - Baseados em visão computacional
    - multi-câmera
    - fluxo-optico, filtro de Kalman, hmm, 
    - câmera única, stereo, ir
    - rastreamento de pedestre, rastreamento de grupo, deteção de objetos largados
    - câmeras rgbd
    - deep learning techniques: classificação de vídeo, detecção de início e fim de atividades, localização espaço-temporal de atividades
    - integração com commonsense reasoning and commonsense knowledge
    - Níveis:
      - detecção de pessoas
      - rastreamento de pessoas
      - reconhecimento de atividades humanas
      - avaliação de atividade de alto-nível.
    - Markov Networks, CNN e LSTM
    - Reconhecimento por marcha, análise de marcha,

- caracterizar contexto, e não mais informação específica quadro a quadro
- o que é e porque é difícil
  - classificação por múltiplos quadros
  - ImageNet
  - Dificuldades:
    - Custo computacional elevado:
      - 3 a 4 dias para treinar uma rede na base ucf101 e aproximadamente dois meses para a base sports-1m
    - Captura de longo contexto:
      - capturar contexto espaço-temporal ao longo dos quadros
    - Projeto de arquiteturas para classificação
      - são muitas as opções
      - a avaliação de cada uma das opções é custosa
    - Não há bechmark padronizado
      - ucf101 e sports1m: old datasets ?
      - one can use youtube?
      - kinectics data set?
      
- abordagens
  - Single stream network: 
  - Two stream network:
    - uma rede para contexto espacial
    - uma rede para contexto de movimento
    - características de movimento são modeladas por vetores de fluxo optico empilhados

    - problemas: false label assignment, ou seja rotular a base é sempre complicado. 
    - segundo o material consultado: o treinamento de ambas as redes era separado. E  ainda havia uma longa estrada para treinamento "end-to-end on-the-go"

- artigos/Redes:
  - LRCN: Long-term Recurrent Convolutional Networks for visual recognition and description
    - the final prediction of each clip (16 frames) is the average of predictions across each time step, and the final prediction at video level is average of predictions from each clip
    - fluxo óptico não pode ser calculado on the fly?
    - desvantagem apontada: inabilidade de capturar informação temporal de faixa longa... por quê? a lstm usada não supriria isso, uma vez que a informação é realimentada? A não ser que a realimentação seja resetada a cada clip. 
    - aumento do número de frames que compõem um clip parece melhorar o desempenho (informação temporal de faixa longa)
  
  - C3D: Learning spatiotemporal features with 3D convolutional networks
    - 3d convolutional nerworks as feature extractors
    - using deconvolutional layers to interpret model decision
    - a simple linear classifier like svm on top of ensemble of extracted features worked better than the sota algorithms, performed wven better if hand crafted featureswere used
    - findings: the net focussed on spatial appearance in first few frames and thacked the motion in the subsequente frames.

  - Conv3D \& Attention
    - Describing videos by exploiting temporal structure
    - among the contributions of the paper: use of an **attention mechanism** within a cnn-rnn encoder-decoder framework to capture global context
    - 3d cnn + lstm: base architecture for video description task.
    - Algorith:
      - 3d cnn feature maps for a clip: concatenation with stacked 2d feature maps for the same set of frames
      - 2d and 3d cnn are pre-trained
      - weighted average is used to combine the temporal features
      - attention weights are decided based on lstm output at every time step.

  - TwoStreamFusion
    - Convolutional two-stream network fusion for video action recognition
      - multi-level fused architecture ?
      - fusion of spatial and temporal streams (how and when) for task discrimination: at early stages, rather than at the end of the pipeline.
      - long-term dependencies are also modeled.
      * The authors stablished the **supremacy** of the **TwoStreamFusion** method: improved performance withou extra parameters (in comparison with C3D)

  - TSN
    - Temporal segments networks: towards good practices for deep action recognition
      - long range temporal modeling effective solution
      - usage of batch normalisation, dropout, and pre-training
      - sparse clip sampling across the video: to better model long range temporal signal
      - final prediction at video-level: 
        - combine scores from temporal and spatial streams
        - Fusing score of final spatial and temporal scores using weighted average and applying softmax over all classes
      - main challenges addressed: overfitting due to small sizes; and long range modeling

  - ActionVlad
    - ActionVlad: learning spatio-temporal aggregation for action classification
      - learnable video-level aggregation of features
      - end-to-end trainable model to capture long term dependency
      - main contribution: vlad (video-level aggregation ...?) rather than normal aggregation (maxpool or avgpool, for instance)
      - VLAD: effective way of pooling ! sota in early 2017

  - HiddenTwoStream
    - Hidden two-stream convolutional networks for action recognition
      - generation of optical flow input on-the-fly (by using a separate network)
      - the paper advocates the usage of an unsupervised architecture to generate optical flow for a stack of frames.
      - optical flow: image reconstruction problem
      - temporal stream:
        - it has an optical flow generation net on the top
        - the input is subsequent frames rather than preprocessed optical flow
      
  - I3D
    - Quo vadis, action recognition? A new model and the Kinetics database
      - 3d models combined into two stream achitecture 
      - two different 3d networks for both streams (spatial and temporal)
      - 2d pre trained weights are repeated in 3d !!!
      - major contribution: benefit of using pre trained 2D conv nets

  - T3D
    - Temporal 3d convnets: new architecture and transfer learning for video classification
      - combine temporal information across variable depth
      - new technique to supervise transfer learning between 2d pre-trained net to 3d net
      - multi-depth temporal pooling layer
Todas as abordagens acima são variações/combinações das redes básicas a seguir:
  - LSTM: long shot term memory
  - 3D-ConvNet
  - Two-stream
  - 3D-Fused Two-Stream
  - Two-Stream 3D-ConvNet



%\end{abstract}
%===

\section{Breve descrição do problema}
\label{sec:descr}
% verificar os desafios do último artigo lido para incluir aqui (se necessário)
% uma aplicação:
% - segurança em ambientes industriais (é o que dá maiores possibilidades para financiamento):
%   - deteção de pessoas;
%   - contagem de pessoas;
%   - presença de EPI;
%   - identificação de ações (quais?)
%
% Desafios: 
% - Variação antropométricas
% - Variação de ângulos de visão para diferentes câmeras no ambiente monitorado
% - Diferentes condições de iluminaçao
% - Baixa qualidade dos vídeos ? qualidade vs taxas de reconhecimento
% - Oclusão
% - Variação de iluminação e presença de sombras
% - Dados insuficientes
% - Variações climáticas
%
% Abordagem:
% -
% 
Sistemas para reconhecimento automático de ações apresentam desempenho cada vez mais próximo da capacidade humana de interpretação visual, graças à popularização das técnicas de aprendizado de máquina, em especial de redes neurais profundas. 
%

O reconhecimento de atividades humanas visa identificar o tipo de ação executada por uma ou mais pessoas em uma cena registrada em vídeo não exibido anteriormente ao sistema.
%
A tarefa apresenta desafios técnicos relevantes, cujo grau de dificuldade depende de diversos elementos, tais como: número de pessoas envolvidas; complexidade da ação de interesse; características do ambiente em que ocorre a ação; grau de interação entre componentes (pessoas e objetos) da atividade; e qualidade do vídeo sob análise. 
%

Para fins de definição de escopo do projeto e atendimento a demandas de setores produtivos locais, optou-se pela aplicação em sistemas de apoio à proteção e segurança em ambientes industriais. Dentre as funcionalidades previstas, destacam-se: detecção e contagem de pessoas nos ambientes monitorados; verificação de presença e utilização correta de equipamento de proteção individual obrigatório; alerta de comportamentos anômalos; e identificação de ações potencialmente perigosas para integridade de trabalhadores e da operação.
%

A presente proposta de pesquisa concentra-se na investigação de novas técnicas em reconhecimento e predição de ações humanas, mediante o emprego de estratégias baseadas em visão computacional e aprendizagem profunda, bem como na implementação de provas de conceito que demonstrem a viabilidade dos sistemas concebidos. 
%
Em redes neurais profundas, duas abordagens principais se consolidaram para tratar o problema em questão: uso de redes \emph{multi-stream} e uso de redes convolucionais 3D. Ambas procuram combinar análise espacial (intra-quadro) e dependências temporais de eventos (inter-quadros) na caracterização das ações monitoradas. 
%

As contribuições científicas visadas pelo estudo proposto concentram-se em três desafios atuais da área: o uso de dados de entrada multi-modais para melhorar o desempenho do reconhecimento visual; a exploraração de mecanismos de interpretação dos sinais de entrada para identificar atributos que permitam reconhecer ações no curto prazo; e o desenvolvimento de abordagens para uso de redes pré-treinadas e o treinamento com conjunto de dados parcialmente rotulado.
%
%Apesar de seus resultados representarem o estado da arte, são conhecidos os problemas da aprendizagem profunda com impacto direto na capacidade de generalização das redes: a necessidade de grande poder computacional e de grande volume de dados rotulados na fase de treinamento (aprendizado supervisionado). Como alternativa, a fim de contornar tais restrições, pode-se empregar o uso de apredizado semi-supervisionado com base de treinamento parcialmente anotada, bem como proceder ao ajuste de parâmetros de redes previamente treinadas (transferência de conhecimento).
Outros ganhos esperados são: formação de recursos humanos especializados em projeto e implementação de sistemas baseados em aprendizado de máquina; prevenção de acidentes em espaços industriais que comprometam a saúde de colaboradores ou possuam efeito contaminante para o ambiente; e aumento de competitividade dos parceiros envolvidos pela inovação em produtos e processos advinda da pesquisa.


\newpage
%===                                          

\section{Revisão da literatura científica}
\label{sec:rev}
\nocite{jegham-2020, hussain-2020,yao-2019,kongr-2018,herath-2017}
%#Elements:
%1- Short and precise **overview**; 
%  - Concentuação/Taxonomia: o que é o reconhecimento de ações (ações ou atividades?, interação, )
%  - Tipos de sistemas. Diferentes classificações: número de pessoas envolvidas, ambiente, interação com pessoas e/ou objetos, predição/reconhecimento, uso de hadcrafted features ou extração automática de features/atributos usando cnns, ou ainda abordagens híbridas
Este projeto trata do reconhecimento automático de atividades humanas mediante o uso de visão computacional baseada em redes profundas. Em razão da falta de consenso sobre os termos usados nas área envolvidas, é interessante que algumas definições sejam postas a fim de caracterizar o escopo deste projeto bem como o objeto de pesquisa. 
%
Assim, a primeira distinção a ser feita é entre \emph{ação} e \emph{atividade}. Sem maiores aprofundamentos, e a despeito das controvérsias, considera-se neste texto que uma \emph{ação} é um evento primitivo, simples, fácil de caracterizar, e que é realizado por uma única pessoa (por exemplo, \emph{homem levanta o braço}). Por outro lado, a sucessão de ações diferentes forma \emph{atividades}, essas mais ou menos complexas (por exemplo, \emph{preparar café}). 
%
Outro aspecto a ser considerado, diz respeito à quantidade de elementos participantes de uma cena. Pode-se ter situações em que apenas uma pessoa está presente; em outras, pode-se ter dois ou mais indivíduos interagindo; e ainda, o(s) indivíduo(s) pode(m) interagir com objetos. As possibilidades de variação aqui são inúmeras, influenciando no grau de complexidade da cena.
%
Cenas não são apenas compostas pelos atores principais que desempenham interações e/ou atividades de interesse, mas também pelo pelo contexto, simplificadamente denominado de \emph{fundo}, que pode variar entre \emph{estático} em ambientes controlados (em um estúdio ou laboratório, por exemplo) ou altamente dinâmicos, como em ruas de grandes cidades ou áreas com a presença de diversas pessoas realizando diferentes atividades. A dificuldade do reconhecimento aumenta com a dinamicidade do fundo, pois além da tarefa de estimar a atividade sendo desempenhada, o sistema precisa diferenciar entre movimentação dos objetos de interesse e das modificações típicas do fundo.
%
O reconhecimento automático de atividades é uma das áreas de maior interesse em visão computacional com um leque de possíveis aplicações abrangendo áreas como ambientes inteligentes, classificação e busca automática em repositórios de imagem e vídeo, e avaliação do comportamento de multidões, dentre outras.
%
%2- **Present state** of the research; 
% Abordagens:
%% Redes multi-streams
%% Redes 3D
%% Soluções híbridas.
Atualmente, a tendência e o estado da arte nessa área compreendem técnicas baseadas em redes neurais profundas. AUTOT-REF aponta como principal desafio a representação da dependência temporal entre ações na caracterização de uma atividade complexa. Tal desafio, em grande parte, tem orientado os esforços da comunidade científica nos avanços da área. 
%
Grosso modo, são duas as principais abordagens mais bem sucedidas na busca de soluções para a representação temporal. A primeira delas, denominada multi-fluxo (multi-stream), procura caracterizar informações temporais e espaciais em dois ramos separados na rede neural profunda. O ramo temporal geralmente tem como entrada o fluxo óptico (ou estimativas deste) de cada quadro de vídeo, ao passo que o ramo espacial recebe o quadro propriamente dito. Cada uma das entradas passa por etapas de extração automática de atributos nas camadas convolucionais da rede, cujos resultados são processados e fundidos oportunamente para fins da classificação final. O uso de fluxo óptico, contudo, tem suas desvantagens uma vez que seu cálculo requer quadros adjacentes ao quadro de interesse, resultando em aumento de complexidade computacional e, consequentemente, dificuldade de implementação em tempo real. A outra abordagem justamente prescinde do fluxo óptico, e compreende o uso de redes profundas 3D que poderiam reunir em um único fluxo informações temporais e espaciais simultaneamente. 
%
Ainda motivado pelas inconveniências do uso do fluxo óptico e pela necessidade de caracterização da informação temporal de longo e curto prazo no entendimento de atividades complexas, técnicas baseadas em redes profundas recursivas, tais como a LSTM (\emph{long short-term memory}) e outras REF, têm sido propostas, apresentando compromisso competitivo entre complexidade computacional e qualidade de resultados REF.

%3- **Immediate connection** with your own research; 
Particularizando para o contexto da aplicação visada neste projeto (reconhecimento de atividades humanas em áreas industriais), pode-se dizer que o ambiente típico a ser a ser monitorado apresenta alta complexidade. Avançando um pouco mais na caracterização desse tipo de ambiente, é importante considerar que outros fatores influenciarão no desempenho do reconhecimento, tais como: variação de luminosidade, de acordo com os horários de monitoramento; presença de mais de um indivíduo interagindo entre si e com objetos, de maneiras difíceis de prever na etapa de treinamento das redes; variação de aparência de uma determinada atividade, em função da distância e do ângulo entre objeto de interesse e câmeras; e ocorrência de oclusões (encobrimento de uma região de interesse por distratores).
%
Apesar bons resultados documentados em REFs, onde diversos dos elementos mencionados, típicos de uma situação realística, são considerados, observa-se acentuada degradação no desempenho dos sistemas de reconhecimento a medida em que o controle das variáveis ambientais é perdido (a expressão em inglês seria \emph{in the wild}). Ou seja, sistemas de reconhecimento automático de atividades humanas em contextos práticos constituem um desafio atual, de grande interesse, e estratégico em termos científicos e de inovação.

%4- Clear and logical **discussion of the theoretical framework**; 
Em grande medida, as limitações de desempenho em situações realísticas são devidas às grandes variações entre cenas apresentadas na fase de operação e cenas usadas na fase de treinamento. Tipicamente, redes profundas exigem  conjunto de dados de treinamento enorme a fim de apresentar boa capacidade de generalização, mesmo em condições controladas. Considerando sistemas de aprendizagem supervisionada (o mais comum), tal conjunto de treinamento requer dados previamente rotulados, exigindo considerável esforço para sua geração e manutenção. Nesse sentido, as tendências alternativas ao treinamento supervisionado completo de sistema de reconhecimento (\emph{from scratch}) que vêm se consolidado na área REF são as seguintes: transferência de conhecimento, usando redes de genéricas pré-treinadas, mediante treinamento supervisionado expondo tais redes a exemplos rotulados típicos da aplicação pretendida, procedimento conhecido pelo termo \emph{fine-tuning}; estratégias para treinamento de conjunto de dados parcialmente rotulados, quando se dispõe de grande quantidade de exemplos no conjunto de dados, mas a atribuição de rótulos possui custo elevado.
%5- **Methodological implications**;
%6- Indicate an **open problem**; 
À medida em que a área avança, outras perspectivas de uso começaram a surgir. Dentre elas, está a que pode ser considerada o próximo passo em visão computacional nessa área: a predição de atividades humanas. Em linhas gerais, pode-se considerar que a distinção entre reconhecimento e predição está na localização temporal pretendida da estimativa de atividade fornecida pelo sistema REF. Para um mesmo conjunto de entrada, o reconhecimento visa identificar a ação ocorrida ou caracterizada nesse conjunto de entrada. Por sua vez, a predição fornecerá uma estimativa da atividade que ocorrerá nos quadros de vídeo seguintes ao conjunto de entrada. O reconhecimento olha para o passado, enquanto que a predição considera o passado para projetar uma aposta sobre o que virá em seguida. Claramente, o problema da predição é bem mais desafiador, mas possui um apelo bastante evidente em direção autônoma e ambientes inteligentes voltados para segurança e prevenção de acidentes, onde melhor que remediar é previnir REF. No aspecto técnico e de implementação dos sistemas, predição e reconhecimento são semelhantes REF. Contudo, em relação aos resultados, a qualidade da predição é acentuadamente menor que em reconhecimento, aumentando ainda mais com o distanciamento temporal entre conjunto de entrada e predição. A representação temporal e da sucessão de ações/atividades assume importância ainda maior na tarefa de predição.
%7- The **contribution**  to the area
Alinhada com o estado da arte, problemas em aberto e tendências em reconhecimento automático de atividades humanas, a presente proposta de pesquisa estabelece como tópicos prioritários de investigação: o uso de dados de entrada  multimodais, ou seja outros tipos de entrada além do vídeo, para melhorar o desempenho do sistema de reconhecimento em ambientes de alta complexidade; a exploração de mecanismos de interpretação do sinais de entrada para identificar atributos que permitam reconhecer ações de curto prazo, fator chave para melhorar a qualidade de sistemas de predição visando segurança e proteção em ambientes industriais; o aperfeiçoamento de abordagens para transferência efetiva de conhecimento, mediante o uso de redes pré-treinadas; e o desenvolvimento de esquema de treinamento com conjunto de dados parcialmente rotulados.

\newpage
%===                                          

\section{Trabalho preparatório}
\label{sec:prep}
%\input{preparatorio}
\newpage
%===                                          

\section{Objetivos do projeto de pesquisa}
\label{sec:obj}
% objetivos.tex
% 
% Aims that you want to achieve:
%   Academical,
%   Social,
%   Political, etc.
% It needs to show the research is important;
% Show the theoretical and practical significance of the project
%

\subsection{Objetivo geral}
O objetivo geral desta proposta de pesquisa consiste no desenvolvimento de novas técnicas em reconhecimento de atividades humanas aplicado ao monitoramento de áreas industriais, usando aprendizagem profunda, com ênfase na solução de problemas locais e na implementação de provas de conceito que demonstrem a viabilidade dos sistemas concebidos.

\subsection{Objetivos específicos} %valeria a pena transformar cada item em uma subsubseção ?? ou simplesmente colocar ":" e desenvolver o tópico?
\begin{enumerate}
    \item Aumento de relevância da produção científica local por meio de publicações em períodicos qualidade reconhecida;
    %formação de recursos humanos
    \item Formação de recursos humanos especializados em projeto, pesquisa e implementação de sistemas de visão computacional e aprendizado de máquina, com foco em Indústria 4.0 e transformação digital; 
    % arranjo produtivo local
    \item Estabelecimento de rede de parcerias envolvendo a academia, a indústria e o setor produtor de \emph{software}, para que as soluções concebidas no projeto não permaneçam apenas em âmbito acadêmico, mas possam ser comercializadas e implantadas; 
    % econômicos
    \item Contribuição para o aumento de competitividade da indústria regional, em razão do aumento de capital social, da redução de custos decorrente do uso de tecnologia desenvolvida pelo arranjo local, e da diminuição de acidentes na operação;  
    \item Manutenção de cultura em inovação, tendo como eixo diretor a busca de soluções a demandas do setor industrial com potencial para resultar na concessão de patentes ou registro de \emph{softwares};
    % pesquisa
    \item Maior visibilidade aos parceiros do projeto em razão da divulgação por meio de publicações e participação em eventos em que o projeto se possa representar; 
    % segurança e proteção
    \item Aumento de segurança e redução de acidentes nos ambientes industriais monitorados;
    % qualidade de vida
    \item Contribuição à melhoria da qualidade de vida do trabalhador industrial;   
    % ambiente
    \item Reforço às medidas de proteção ambiental já implantadas no setor industrial;
    \item Subsídio à prevenção de acidentes de efeito contaminante para o ambiente;
    % social
    \item Apoio aos mecanismos de manutenção da saúde e modo de vida das populações tradicionais em torno de instalações industriais.
\end{enumerate}

\newpage
%===

\section{Esboço do projeto}
\label{sec:proj}
%the sources and quality of evidences consulted

% justificativa: relevância do problema
\subsection{Justificativa}
\label{ssec:just}
% qual a relevância do problema
A razão de ser desta proposta apresenta-se sob dois enfoques. Um deles, acadêmico: redes neurais profundas aplicadas à interpretação automática de imagens ou vídeos é um tópico de grande atualidade e interesse na comunidade científica. Mais especificamente, o reconhecimento de atividades humanas em ambientes reais ganha cada vez mais espaço em conferências e publicações especializadas em decorrência dos resultados promissores alcançados mediante proposições de novas arquiteturas, estratégias de treinamento e transferência de conhecimento em aprendizagem profunda, e disponibilidade cada vez maior tanto de sistemas de captura com recursos de acesso a redes de comunicação, quanto de unidades de armazenamento de alta capacidade. Considerável esforço tem sido empregado na modelagem temporal de ações sucessivas que formam atividades complexas, na caracterização da interação entre pessoas e objetos em uma cena, e no aumento de acurácia de classificação de ações quando o ambiente monitorado é complexo e dinâmico.

O segundo enfoque, contudo, possui grande relevância e constitui a principal motivação deste projeto: segurança e proteção em ambientes industriais. O tema possui importância evidente, pois impacta em um número muito grande de elementos, dentre os quais, a integridade física e psicológica dos colaboradores da indústria, produtividade das empresas, e pegada ambiental da atividade industrial. Apesar da adoção de políticas internas induzidas pela legislação específica, estima-se que mais de R\$ 20 bilhões da Previdência Social foram gastos com benefícios (indenizações e tratamentos de saúde) para trabalhadores acidentados entre 2012 e 2016. A Associação Nacional dos Procuradores do Trabalho (ANPT), indica que em 2015 o Brasil registrou aproximadamente 704.000 acidentes de trabalho e 3.000 mortes associadas. Tais acidentes trazem consequências negativas para a economia em geral, para a empresa implicada, e, principalmente, para os trabalhadores e seus familiares. Em setores de grande concorrência, os custos associados a acidentes laborais pode acarretar expressiva redução do potencial competitivo da empresa.  

Os sistemas de reconhecimento de atividades humanas fazem uso de técnicas de inteligência computacional e oferecem uma camada de proteção a mais aos protocolos já implementados pela equipe de segurança de uma empresa. Tais sistemas podem ser usados no apoio às inspeções de segurança regulares, e em ações de conscientização sobre a importância do uso de equipamento de proteção individual (EPI) e coletiva (EPC), e manuseio correto de maquinário. Se o sistema de monitoramento dispor de câmeras com sensores térmicos, avaliações automáticas de temperatura podem ser programadas, auxiliando na prevenção contra incêncios. Uma tendência recente com crescente popularidade, é o uso de câmeras embarcadas em drones. O arranjo permite a verificação remota de áreas, equipamentos e estruturas de difícil acesso ou de grande risco aos trabalhadores.

A integração de recursos de inteligência computacional na atividade industrial alinha-se com as proposições de renovação tecnológica da Indústria 4.0 e da transformação digital. Economia de insumos, aumento da produtividade e competitividade, prevenção ativa contra acidentes, e redução do número de paradas na operação, seja para manutenção reativa ou reposição de peças e equipamentos por uso incorreto são alguns dos benefícios advindos da implantação de um sistema de monitoramento inteligente. 

\subsection{Metodologia}
\label{ssec:metod}
% metodologia: como 
% procedimentos experimentais
% data gathered 
% controls
% statistical methods

% formação de equipe
% uso de bases de dados públicas gratuitas para testes iniciais
% montagem de um ambiente/setup experimental para realização de simulações de cenas e testes das estratégias/técnicas utilizadas: computadores, sistemas de aquisição de vídeo (câmeras, sistema de armazenamento), software
% integração com o sistema de monitoramento em ambiente real
% testes iniciais em ambiente real
% 
% Python pelo conjunto variado de módulos (bibliotecas) em processamento de sinais, visão computacional e aprendizado de máquina disponíveis, ser uma linguagem de alto nível com resultados na redução do tempo de codificação e obtenção de resultados. na fase de desenvolvimento, mas haverá adaptação ao contexto de software usado na empresa
% Estratégias de desenvolvimento ágil, em que versões funcionais do produto são lançadas regularmente ao longo do período de desenvolvimento do projeto, facilitando correções e adaptações de escopo
%
% Prazo de dois anos: ao fim do primeiro ano, primeiro release já instalado na empresa. Atualizações de funcionalidade periódicas, mês a mês até o final do projeto
% Relatórios de acompanhamento técnico-finaceiro semestrais
% Reuniões de revisão abertas a todos os interessados (stackeholders)
%
% As técnicas propostas serão avaliadas segundo os procedimentos padrão da área de classificação: treinamento: data augmentation, e validação cruzada; monitoramento da função custo e acurácia de classificação, matriz de confusão, curvas ROC (em caso de discriminadores ou seja lassificação binária) e curvas ROC multi-classes (adaptação para classificação não-binária), complexidade computacional em fase de treinamento e produção, 

% Esses tópicos devem ser transferidos ou replicados para a seção de indicadores...
% Dadas as orientações, é necessário verificar se as normas estão sendo seguidas. Tendo isso em mente, crie métricas para medir o desempenho dos profissionais quanto ao uso dos equipamentos de proteção e adoção das medidas de segurança.
% Alguns dos indicadores que podem ser utilizados para fins de avaliação e divulgação são:
%    número de incidentes;
%    despesas com multas;
%    quantidade de funcionários afastados por motivo de acidentes de trabalho;
%    número de irregularidades encontradas em inspeções.


% risco tecnológico
\subsection{Análise preliminar de riscos}
\label{ssec:risco}
A seguir são apontados os riscos antevistos e as estratégias para mitigá-los: 
\begin{enumerate}
	\item Indisponibilidade de conjuntos de dados adequados para treinamento do sistema de reconhecimento de atividades: apesar da disponibilidade de conjunto de dados públicos, o número de clips contendo situações específicas de interesse para a aplicação pretendida não é suficiente para o treinamento adequado do sistema de reconhecimento. Dessa forma, deverão ser produzidos novos vídeos que retratem mais fielmente situações e características ambientais típicas da aplicação pretendida. A indisponibilidade de conjunto de dados apropriado tem impacto negativo na acurácia do reconhecimento e poderá inviabilizar o atingimento dos objetivos do projeto. Para mitigar: prioridade na realização de setup experimental piloto, e início de coleta de dados ainda no primeiro semestre de execução do projeto. Nesse sentido, desde o primeiro momento deve se dar atenção especial à aquisição dos equipamentos necessários à coleta e ao armazenamento de dados, e à obtenção das permissões requeridas (comitê de ética em pesquisa, administração da operação, orgãos governamentais, etc.).
	\item Perda de dados experimentais: um dos elementos de maior valor agregado no projeto é o conjunto de dados contendo os vídeos produzidos e utilizados para treinamento do sistema de reconhecimento de atividades. Isso se deve ao esforço necessário para coleta de dados, rotulação de atividades e edição de vídeo. Para mitigar: uma estratégia automática e estrita de backup, com uso de redundâncias em relação ao número e frequência de cópias, deve ser estabelecida para manter probabilidade de perda de dados sob controle.
	\item Perda de código produzido: o produto mesmo do projeto consiste em softwares (sistema de reconhecimento de atividades, aplicativo mobile e aplicativo web). A perda de código provocada por fatores acidentais (queima de dispositivos de armazenamento por surtos de energia) ou intencionais (ataques cibernéticos) pode acarretar atrasos nas entregas do projeto ou mesmo inviabilizar a concretização de seu objetivo. Para mitigar: uso de equipamentos de nobreak (sistemas de proteção a surtos) e armazenamento de código em repositórios distribuídos (local e em nuvem) com controle de acesso para evitar uso não autorizado.
	\item Indisponibilidade de recursos de software e hardware suficientes para o desenvolvimento da proposta: a alta cambial do Dólar em relação ao Real pode levar à impossibilidade de aquisição de hardware e software especificados inicialmente na etapa de elaboração da proposta. Para mitigar: considerar a variação do câmbio nos últimos 6 meses por ocasião da elaboração do orçamento do projeto; priorização e presteza nas atividades de compra de equipamentos e programas de computador, tão logo haja liberação de recursos; e contratação de recursos de computação em nuvem no início da etapa de desenvolvimento.
	\item Violação de confidencialidade: os vídeos adquiridos pelo sistema de reconhecimento, seja na fase de desenvolvimento ou de operação, podem ser hackeados revelando informações estratégicas sobre as empresas apoiadoras da proposta. Para mitigar: todo e qualquer material em vídeo será transmitido e armazenado em redes protegidas, consoante aos procedimentos de segurança e com concordância expressa das empresas apoiadoras. 
\end{enumerate}


% grau de inovação
\subsection{Grau de inovação}
\label{ssec:inova}
O sistema proposto é composto por: software para reconhecimento automático de atividades humanas, voltado para a detecção de situações de risco à integridade física de pessoas em ambientes industriais monitorados, e aplicativos mobile e web de acesso e análise dos resultados decorrentes desse monitoramento. 
Os produtos existentes no mercado com maior semelhança à proposta são: câmeras de videomonitoramento; sistemas para prevenção contra afogamento; e monitoramento de atividades por sensores. 

As câmeras de videomonitoramento atuais podem ser conectadas à internet ou permanecer em circuito fechado. Alguns modelos possuem visão noturna, visão térmica integrada, controle de posição e zoom (câmeras PTZ), canal de áudio bidirecional, e sistema de proteção anti-vandalismo. Dispõem ainda de aplicativo de controle, possibilidade de visualização em tempo real e armazenamento de vídeo local ou em nuvem, e interação por áudio com o ambiente monitorado. Alguns sistemas têm inteligência embarcada para detecção de situções de interesse simples, tais como identificação de modelos e cor de automóveis, reconhecimento de placas, estatísticas de fluxo (carros e pessoas), deteção de fogo, medição de temperatura, e cerca virtual. A reação a situações mais complexas, contudo, é feita tardiamente a partir da revisão do vídeo registrado, salvo se alguém estiver observando o ambiente monitorado no instante de ocorrência, ou se a pessoa monitorada tiver possibilidade de se comunicar pelo canal de áudio da própria câmera. 

Por sua vez, as soluções para prevenção contra afogamentos em piscinas consistem em: câmeras, para monitorar tanto fora como embaixo d’água; sistemas de hardware e software para análise de vídeo; e interface de visualização/alerta. As imagens capturadas são analisadas em tempo real, procurando detectar duas situações: pessoas que estejam sem movimento; pessoas cuja movimentação seja incompatível com o que se poderia chamar de nado. Nessas duas situações, alertas sonoros e visuais são enviados para um equipamento que fica com o guarda-vidas em piscinas públicas ou responsáveis e familiares para piscinas em residências.

O monitoramento de atividades por sensores tem sido usado comumente em ambientes domésticos. Sua implementação pressupõe o uso de sensores sem fio no ambiente e/ou objetos de interesse. Por exemplo, a regularidade no uso de medicação por idosos pode ser inferida por sensor posicionado na caixa ou gaveta de medicamentos, cujos horários de abertura podem ser comparados com cronograma previamente definido. Da mesma forma, pode-se identificar se a rotina diária de atividades foi alterada mediante sensores posicionados em controle remoto da TV, cama, porta, xícara, etc. Uma variante deste sistema é o uso de vestíveis, ou seja, sensores em calçados, camisas, calças, etc., que permitiriam registrar a movimentação de indivíduos que estivessem portanto as peças de roupa monitoradas. O principal inconveniente dessa abordagem é a necessária adesão do indivíduo ao uso dos objetos e roupas contendo sensores, o que pode ser problemático em certas situações.

Assim, o sistema proposto concentra-se em etapas de processamento e análise que são posteriores à aquisição de vídeo, fazendo uso das câmeras de videomonitoramento já existentes no mercado ou previamente instaladas no ambiente a ser monitorado, mas agregando inteligência ao monitoramento. Detecção de acesso a ambientes por pessoas ou em horários não autorizados, ausência de EPI obrigatório, localização de comportamento anômalo, e avaliação de fluxo de pessoas são algumas das possibilidades de uso do sistema proposto. Em comparação ao uso de sensores, a abordagem por visão computacional dispensa o uso de roupas e objetos específicos, podendo ser utilizada em um número maior de situações sem que haja necessidade de aumentar o número de equipamentos para realizar o monitoramento. Os autores da proposta acreditam que a  proposta aporta inovação importante na área de  monitoramento de ambientes industriais, promovendo maior segurança a colaboradores e contribuindo para prevenção contra acidentes industriais. Até o presente momento não se tem conhecimento de sistema comercial com as mesmas características da proposta apresentada.

Os proponentes deste projeto acompanham os trabalhos científicos recentes sobre reconhecimento automático de atividades e se propõem a avançar o estado da arte na área no que diz respeito a viabilizar a implantação do sistema em ambiente operacional.

% grau de maturidade tecnológica atual, pretendido e os meios para chegar lá
\subsection{Maturidade tecnológica}
\label{ssec:trl}
Os trabalhos científicos publicados sobre sistemas de reconhecimento automático de atividades humanas baseados em visão computacional e aprendizado profundo com potencial para detectar situações de risco em ambientes industriais reportam testes de desempenho obtidos em ambiente laboratorial, configurando nível de maturidade tecnológica 4 (TRL 4). A proposta envolve o reconhecimento multimodal de atividades, onde o vídeo obtido por câmeras convencionais (faixa visível) é acrescido de informações de áudio e de câmeras térmicas, visando melhoria de desempenho em situações práticas de operação.

Pretende-se chegar à demonstração de protótipo do sistema proposto em ambiente operacional, ou seja, atingir nível de maturidade tecnológica 7 (TRL 7). Para tanto, faz-se necessário percorrer uma sequência de etapas para realizar a transição de maturidade tecnológica almejada.

A primeira dessas etapas é a reprodução do estado da arte em reconhecimento de atividades humanas usando uma rede genérica pré-treinada, cujos parâmetros deverão ser reajustados mediante conjunto de treinamento específico para as atividades de interesse. Para a formação de tal conjunto de dados, deve-se selecionar clips (trechos curtos de vídeo) a partir de bases disponíveis gratuitamente na internet, e em repositórios públicos como o Youtube. A principal entrega nesta etapa é um rede neural profunda com desempenho superior ao da rede genérica original em relação ao reconhecimento das atividades de interesse, e que servirá de base para os próximos desenvolvimentos.

Em seguida, ainda usando repositórios e recursos disponíveis gratuitamente na internet, será acrescentado na rede de reconhecimento um ramo destinado ao processamento de sinal de áudio. A inserção do áudio permitirá que o sistema leve em consideração os sons do ambiente no processo de reconhecimento, combinando-os com as informações visuais. A entrega desta etapa será uma arquitetura que tratará e combinará informações de vídeo convencional e áudio para classificação de atividades. 

Na terceira etapa, agrega-se a informação de sensores térmicos. Nesse caso, haverá necessidade de formação de conjunto de dados próprio pela indisponibilidade desse tipo de informação associada a sinais de áudio e vídeo em repositórios públicos. O ambiente considerado aqui ainda deve ser controlado, porém com características próximas às do ambiente operacional (ambiente relevante), para que se possa avaliar como melhor agregar a imagem térmica ao sistema de classificação já utilizado. Essa é uma etapa que está no caminho crítico da execução da proposta e que depende da aquisição de equipamentos com recursos do projeto. Nessa etapa, duas são as entregas: o próprio conjunto de dados que será único e, portanto, conterá grande valor científico agregado; e uma nova arquitetura de rede profunda para reconhecimento de atividades humanas, cujas entradas compreendem vídeo convencional, áudio e imagem térmica.

A última etapa é semelhante à terceira, com a diferença que o sistema de captura deverá ser instalado em ambiente operacional. Esta etapa, na verdade, deve ocorrer em paralelo à etapa anterior: as coletas em ambiente relevante e operacional podem acontecer simultaneamente para evitar atrasos no desenvolvimento do projeto. Uma vez que se tenha uma arquitetura que integre adequadamente a imagem térmica com ganhos de desempenho em ambiente relevante, o sistema pode ser exposto a situações reais para efetuar os ajustes devidos de seus parâmetros e sua arquitetura à nova situação de operação. Entregas previstas: conjunto de dados, contendo atividades monitoradas em ambiente operacional (mais uma vez, de grande valor científico agregado); e sistema de reconhecimento de atividades humanas, cujas entradas são sinais de vídeo convencional, imagem térmica e sinais de áudio, funcionando em situação prática.

Os aplicativos mobile e web para armazenamento e visualização dos resultados obtidos pelo sistema de monitoramento podem ser desenvolvidos por grupo independente da equipe de aprendizado de máquina, responsável pelo sistema de reconhecimento de atividades. A integração dos aplicativos com a base de dados gerada pelo sistema de aquisição será uma tarefa executada ao longo de todo o período de desenvolvimento do projeto.

% resumo da equipe executora
\subsection{Proponentes}
\label{ssec:equipe}
\begin{enumerate}
	\item Ronaldo de Freitas Zampolo é graduado em Engenharia Elétrica pela Universidade Federal do Pará (1995), obteve os títulos de mestre e doutor em Engenharia Elétrica na Universidade Federal de Santa Catarina (em 1998 e 2003, respectivamente), e realizou estágio de pós-doutoramento na Polytech Nantes (Escola de Engenharia da Universidade de Nantes, França) em 2016. Atualmente é professor associado III da Universidade Federal do Pará, membro do Laboratório de Processamento de Sinais, e docente da Faculdade de Engenharia da Computação e Telecomunicações. É também docente permanente do Programa de Pós-Graduação em Ciências Forenses da Universidade Federal do Sul e Sudeste do Pará (UNIFESSPA). Seus interesses e estudos recentes concentram-se, principalmente, nos temas: avaliação de qualidade visual, modelos de atenção visual, sistemas de rastreamento ocular, análise de vídeo, e reconhecimento de atividades humanas. 

Dentre os projetos em que tem atuado, aqueles de maior afinidade com o tema desta proposta, citamos: 
		\begin{itemize}
			\item ``Sistemas e tecnologias assistivas e de apoio à reabilitação: implementação de sistemas para análise de marcha 2020'': projeto em curso, aprovado no edital PIBEX 2020 (PROEX-UFPA) com recurso para financiamento de bolsa para aluno de graduação; utilização de sistema baseado em redes neurais profundas (OpenPose)\parencite{cao-2019} para localização automática de pontos anatômicos (articulações de quadril, joelho, tornozelo e ponta do pé) por vídeo, para estimação de parâmetros de marcha sem uso de marcadores (atuação como coordenador); 
			\item ``Soluções em `Indústria 4.0' aplicadas à Mineração'': encerrado em 2019, tratava da aplicação de visão computacional na inspeção automática de componentes em vagões ferroviários; projeto financiado pelo ITV-DS, SENAI-PA e CNPq (atuação como membro de equipe de pesquisadores);
			\item ``Desenvolvimento de um sistema de monitoramento e diagnóstico on-line de descargas parciais nos enrolamentos estatóricos de hidrogeradores'': projeto encerrado em 2015, versava sobre o uso de redes neurais para identificação de descargas parciais, contava com financiamento da Eletronorte (atuação como membro de equipe de pesquisadores).
		\end{itemize}
A seguir, segue uma relação de trabalhos publicados em alinhamento com a presente  proposta:
		\begin{itemize}
			\item GONÇALVES, C. L. A. et al. Improving the performance of a SVM+HOG classifier for detection and tracking of wagon components by using geometric constraints. In: WORKSHOP DE APLICAÇÕES INDUSTRIAIS - CONFERENCE ON GRAPHICS, PATTERNS AND IMAGES (SIBGRAPI), 32., 2019, Rio de Janeiro. Anais [...]. Porto Alegre: Sociedade Brasileira de Computação, 2019. p. 230-236. DOI: https://doi.org/10.5753/sibgrapi.est.2019.8336. 
			\item GONÇALVES, L. A., ZAMPOLO, R. F., BARROS, F. B.. A multi-stream network with different receptive fields to assess visual quality. In: Symposium on Knowledge Discovery, Mining and Learning (KDMiLe) (KDMile'19), 2019, Fortaleza (CE). Proceedings of Symposium on Knowledge Discovery, Mining and Learning (KDMiLe) (KDMile'19), 2019. 
			\item OLIVEIRA, R. M. S., ARÁUJO, R. C. F., BARROS, F. J. B., SEGUNDO, A. P., ZAMPOLO, R. F. , FONSECA, W. , DIMITRIEV, V., BRASIL, F. S.. A system based on artificial neural networks for automatic classification of hydro-generator stator windings partial discharges. In: Journal of Microwaves, Optoelectronics and Electromagnetic Applications, vol. 16, n. 3, September, 2017.
			\item GONÇALVES, C. L. A. , ZAMPOLO, R. F.. Classificador SLFN-ELM aplicado à verificação de adulteração de imagens baseada em padrão CFA. In: XXXIV Simpósio Brasileiro de Telecomunicações, 2016, Santarém (PA). Anais do XXXIV Simpósio Brasileiro de Telecomunicações, 2016.
		\end{itemize}	

\item Adriana Rosa Garcez Castro possui graduação pela Universidade Federal do Pará-UFPA (1992), mestrado em Engenharia Elétrica pela Universidade Federal do Pará (1995) e doutorado em Engenharia Elétrica pela Faculdade de Engenharia da Universidade do Porto, Portugal (2004). Atualmente é professora Associada IV da Faculdade de Engenharias Elétrica e Biomédica da UFPA, e vem atuando como pesquisadora e docente permanente no Programa de Pós-graduação em Engenharia Elétrica da UFPA, sendo suas áreas de interesse: Sistemas de Energia, Controle de Processos Eletrônicos e Inteligência Computacional Aplicada.

\end{enumerate}

% apresentação do laboratório
\subsection{Grupos de pesquisa}
\label{ssec:grppesq}
O Laboratório de Processamento de Sinais (LaPS) é um grupo de pesquisa associado à Faculdade de Engenharia da Computação e Telecomunicações,  à Faculdade de Engenharias Elétrica e Biomédica, e ao Programa de Pós-Graduação de Engenharia Elétrica, todos do Instituto de Tecnologia da UFPA. O LaPS desenvolve pesquisa em diversos ramos do processamento de sinais, com ênfase nas seguintes áreas: a) Sistemas de comunicação 5G e além: modelagem de canal de propagação em ondas milimétricas utilizando traçado de raios (ray-tracing), arquiteturas híbridas para separação espacial de usuários (hybrid-beamforming) para sistemas MIMO (múltiplas-entradas, múltiplas-saídas), técnicas de comunicação para sistemas MIMO massivos; b) Análise de sinais bioelétricos em estados normais e alterados (epilepsia, autismos, TDAH): eletroencefalograma, eletromiograma, eletrocardiograma. Sistemas de Neurofeedback. Neuroengenharia; e c) Processamento de imagem e visão computacional: avaliação de qualidade visual, reconhecimento de atividades humanas, inspeção automática de componentes, sistemas de eye-tracking e modelos de atenção visual. Dentre os parceiros do LaPS, destacamos a Coordenação de Aperfeiçoamento de Pessoal de Nível Superior (CAPES), o Conselho Nacional de Desenvolvimento Científico e Tecnológico (CNPq), o Instituto SENAI de Inovação - Tecnologias Minerais (ISI-TM), o Abrigo Especial Calabriano, e a Jambu Tecnologia.

%Dentre as atividades desenvolvidas pelo LaPS diretamente relacionadas ao tema da chamada, podemos citar os projetos “Sistemas e tecnologias assistivas e de apoio à reabilitação: implementação de sistemas para análise de marcha 2020” (em curso), “Dmóvel: Ferramenta móvel para fomento à autonomia de PcDs – 2020” (em curso) e “Implementação de Laboratório de Análise de Marcha de Baixo-Custo para Auxílio na Reabilitação de Crianças Portadoras de Encefalopatia Não-Evolutiva” (encerrado em 2012). Os dois primeiros foram recentemente aprovados em editais de extensão da UFPA, sendo contemplados com bolsa para alunos de graduação. O terceiro, por sua vez, foi aprovado em edital da FAPESPA (Fundação de Apoio à Pesquisa do Estado do Pará). Tais projetos contam com instituição parceira de assistência e saúde voltada a pessoas com deficiência, cujo quadro profissional é formado por médicos, psicólogos, terapeutas ocupacionais, e fisioterapia com longa experiência na área. Apesar das aplicações diferirem daquela desta chamada, a experiência em visão computacional e redes neurais artificiais do LaPS pode ser verificada pela participação em outros dois projetos: “Soluções em “Indústria 4.0” aplicadas à Mineração” (encerrado em 2019, aplicação de visão computacional na inspeção automática de componentes em vagões ferroviários, com parcerias do Instituto Tecnológico Vale, e Instituto SENAI de Inovação, e financiamento pelo ITV-DS, SENAI-PA e CNPq) e “Desenvolvimento de um sistema de monitoramento e diagnóstico on-line de descargas parciais nos enrolamentos estatóricos de hidrogeradores” (encerrado em 2015, usa redes neurais para identificação de descargas parciais, com financiamento pela Eletronorte).

% resumo do orçamento
\subsection{Resumo do orçamento}
\label{ssec:orca}

% impactos: tecnológico, econômico, ambiental, social,
\subsection{Indicadores de impacto}
\label{ssec:impact}
\begin{enumerate}
	\item Impacto tecnológico e científico:
		%Apresentar indicadores voltados à área tecnológica, tais como desenvolvimento de produtos ou processos, obtenção de patentes, entre outros.
		\begin{enumerate}
			\item Desenvolvimento de 2 produtos: software para reconhecimento de atividades; e aplicativos web/mobile para consulta a relatórios e análise dos dados coletados.
			\item Registro de 2 softwares: 1) Aplicativo mobile e 2) sistema web para visualização de relatórios para monitoramento remoto de atividades; 
			%\item Potencial para patente: Técnica para reconhecimento automático de atividades humanas baseado em visão computacional e redes neurais profundas.
			\item Publicação de artigos em eventos e periódicos de grande relevância científica.
		\end{enumerate}
	\item Impacto econômico:
		%Apresentar indicadores voltados à área econômica, em termos da transferência dos resultados do projeto e sua incorporação pelos setores de produção industrial, serviços e governo, tais como redução de custos, investimentos e retorno financeiro.
		\begin{enumerate}
			\item Aumento de economia, nas áreas de saúde, materiais e serviços, decorrente da redução de acidentes nos ambientes monitorados.
			\item Maior aproveitamento de recursos, face à diminuição deações de reposição ou manutenção corretiva de equipamentos por danos advindos de manuseio incorreto ou roubo.
			\item Incremento na competitividade das empresas apoiadoras em consequência da redução do custo de operaçao.
		\end{enumerate}
	\item Impacto ambiental:
		%Apresentar indicadores voltados à área ambiental, em termos de sua influência nos níveis de qualidade da água, ar e solos, da preservação da diversidade biológica ou recuperação de degradação, entre outros.
		\begin{enumerate}
			\item Não há risco ambiental direto associado ao desenvolvimento ou ao uso do sistema proposto, uma vez que não há produção de substâncias tóxicas ou de efeito contaminante para o ambiente.
			\item Diminuição do risco de contaminação ambiental associada à ocorrência de acidentes na operação.  
		\end{enumerate}
	\item Impacto social:
		%Apresentar indicadores voltados à área social, em termos de sua influência nos níveis de qualidade de vida das populações afetadas, em âmbito regional ou local, tais como emprego, renda, saúde, educação, habitação, saneamento, entre outros
		\begin{enumerate}
			\item Melhoria da qualidade de vida de colaboradores.
			\item Aumento do capital social, evidenciado pelo investimento na segurança da operação, preocupação com a integridade física e saúde dos colaboradores, e preservação do modo de vida das populações locais.
		\end{enumerate}
\end{enumerate}

\newpage

\section{Cronograma}
\label{sec:cronos}
% cronos.tex
\subsection{Atividades e metas}
\begin{itemize}
	\item Atividade A1: revisão bibliográfica e atualização do estado-da-arte
	\begin{itemize}
		\item Descrição: a fim de entregar o máximo de valor agregado, a equipe executora acompanhará com regularidade ao longo de todo o período de execução do projeto o estado-da-arte na área de videomonitoramento inteligente através do estudo de artigos publicados em períódicos e eventos técnico-científicos de comprovada qualidade.
		\item Metas associadas: texto de revisão da literatura científica e coleção de artigos e sites selecionados; primeira versão de ambos (texto e coleção) no terceiro mês de execução do projeto (M1a), e versões atualizadas nos meses 6 (M1b), 12 (M1c), 18 (M1d) e 24 (M1e).
		\item Meses: todos.
	\end{itemize}
	\item Atividade A2: aquisição de equipamento.
	\begin{itemize}
		\item Descrição: a aquisição de equipamento de videomonitoramento para montagem de setup experimental é essencial para o sucesso do projeto, sendo particularmente importante nas etapas de seleção e teste de técnicas candidatas; a atividade compreende a definição de empresas fornecedoras, e realização de procedimentos de compra, acompanhamento e recebimento de equipamento de videomonitoramento. 
		\item Meses: 1 a 3.
		\item Metas associadas: equipamento disponível para equipe executora com documentação pertinente (M2).
	\end{itemize}
	\item Atividade A3: instalação, teste e treinamento de equipamento adquirido.
	\begin{itemize}
		\item Descrição: montagem, instalação e teste de equipamento adquirido; deve ser realizado pela equipe executora ou empresa especializada, dependendo do tipo de equipamento; se necessário a equipe de executora deverá realizar treinamento para uso do equipamento; em caso de dano de transporte ou de fábrica, acionar a garantia do produto.
		\item Meses: 3 e 4.
		\item Metas associadas: equipamento instalado e operacional, pronto para ser usado (M3).
	\end{itemize}
	\item Atividade A4: investigação de técnicas candidatas.
	\begin{itemize}
		\item Descrição: implementar e testar em ambiente laboratorial abordagens com potencial para atender aos objetivos do projeto.
		\item Meses: 1 a 12.
		\item Metas associadas: setup experimental (M4a), protótipo funcional (M4b), implementação funcional em Python da técnica selecionada armazenada em repositório do projeto (M4c), documento justificando a técnica selecionada (M4d).
	\end{itemize}
	\item Atividade A5: realização de testes em ambiente relevante.
	\begin{itemize}
		\item Descrição: ambiente relevante significa um contexto de operação intermediário entre o ambiente laboratorial, em que as variáveis do experimento estão sob controle, e o ambiente operacional, onde o sistema deverá funcionar na prática. A realização de teste do sistema de videomonitoramento em ambiente relevante permitirá observar o desempenho e comportamento do sistema em situações não previstas inicialmente, bem como subsidiará as correções necessárias para seu correto funcionamento.
		\item Meses: 13 a 18.
		\item Metas associadas: protótipo funcional (M5a), nova versão funcional em Python do sistema armazenado em repositório do projeto (M5b), documentação com testes de desempenho, adaptações e correções feitas (M5c).
	\end{itemize}
	\item Atividade A6: integração do sistema em ambiente operacional (4o semestre).
	\begin{itemize}
		\item Descrição: compreende a adaptação do sistema testado em ambiente relevante para o local de operação real e integração com o sistema de monitoramento da empresa demandadora/parceira; inclui ainda ações de transferência de conhecimento/tecnologia ao corpo técnico da empresa.
		\item Meses: 19 a 24.
		\item Metas associadas: sistema proposto integrado ao aparato de monitoramento da empresa demandadora/parceira e operando em ambiente real (M6a); manual de operação do sistema (M6b); eventualmente, realização de minicurso/seminário apresentando o sistema implementado (M6c).
	\end{itemize}
	\item Atividade A7: redação de relatório técnico-financeiro (a cada 6 meses).
	\begin{itemize}
		\item Descrição: consolidação de registros técnicos e financeiros em relatórios redigidos para prestar contas dos recursos utilizados, subsidiar a redação de artigos técnicos e informar a quem de direito sobre o andamento geral do projeto.
		\item Meses: 6, 12, 18, e 24.
		\item Metas associadas: relatórios parciais (M7a, M7b, e M7c) e relatório final (M7d).
	\end{itemize}
	\item  Atividade A8: redação de artigos e participação em eventos técnico-científicos
	\begin{itemize}
		\item Descrição: a participação nesses eventos deve ocorrer, de preferência, condicionada à aprovação de trabalhos de autoria da equipe executora em tema associado ao objeto de estudo do projeto; cumpre diversas finalidades: avaliação externa da qualidade do trabalho sendo desenvolvido, divulgação do projeto e parceiros; atualização sobre o estado-da-arte no tema da pesquisa; estabelecimento de contatos para possíveis futuras parcerias.
		\item Meses: não há mês definido \emph{a priori}, contudo, estima-se que o esforço de redação deve ser realizado ao final do segundo e terceiro semestres, ou seja, meses 11 e 12, e meses 17 e 18.
		\item Metas associadas: publicação de pelo menos dois artigos (M8a, e M8b) em eventos técnico-científicos (um nacional e outro internacional) durante a vigência do projeto.
	\end{itemize}
	\item Atividade A9: redação de artigo para publicação em periódico.
	\begin{itemize}
		\item Descrição: a redação de artigo com estrutura e qualidade para publicação em periódicos especializados com revisão por pares indexado no Qualis da CAPES
		\item Meses: 19 a 24.
		\item Metas associadas: publicação de, pelo menos, um artigo em periódico com Qualis A (M9).
	\end{itemize}
\end{itemize}

\newpage
\subsection{Cronograma}
Na Tabela \ref{tab:crono}, as atividades e metas estão dispostas ao longo dos meses de desenvolvimento do projeto.
\begin{table}[h!]
	\caption{Tabela de cronograma (as colunas numeradas referem-se aos meses de desenvolvimento do projeto)}
\rowcolors{2}{lightgray}{}%lightgray
\begin{tabular}{ p{0.35\textwidth} cccccccccccc}
\toprule
%\hline\hline
   \rowcolor{lgray}
   Atividades -- ano 01                                                            & 01 & 02 & 03 & 04 & 05 & 06 & 07 & 08 & 09 & 10 & 11 & 12 \\
\midrule
   Atividade A1: revisão bibliográfica e atualização do estado-da-arte             &  x & x  & M1a& x  & x  & M1b& x  & x  & x  & x  & x  & M1c\\
   Atividade A2: aquisição de equipamento                                          &  x & x  & M2 &    &    &    &    &    &    &    &    &    \\
   Atividade A3: instalação, teste e treinamento de equipamento adquirido          &    &    & x  & M3 &    &    &    &    &    &    &    &    \\
   Atividade A4: investigação de técnicas candidatas                               &  x & x  & x  & x  & x  & x  & x  & x  & x  & x  & x  & M4 \\
   Atividade A7: redação de relatório técnico-financeiro (a cada 6 meses)          &    &    &    &    &    & M7a&    &    &    &    &    & M7b\\
   Atividade A8: redação de artigos e participação em eventos técnico-científicos &    &    &    &    &    &    &    &    &    &    & x  & M8a\\
	\midrule
%   \rowcolor{lightblue}%lgray}
   \rowcolor{lgray}
   Atividades -- ano 02                                                            & 13 & 14 & 15 & 16 & 17 & 18 & 19 & 20 & 21 & 22 & 23 & 24 \\
\midrule
   Atividade A1: revisão bibliográfica e atualização do estado-da-arte             & x  & x  & x  & x  & x  & M1d& x  & x  & x  & x  & x  & M1e\\
   Atividade A5: realização de testes em ambiente relevante                        & x  & x  & x  & x  & x  & M5 &    &    &    &    &    &    \\
   Atividade A6: integração do sistema em ambiente operacional                     &    &    &    &    &    &    & x  & x  & x  & x  & x  & M6 \\
   Atividade A7: redação de relatório técnico-financeiro (a cada 6 meses)          &    &    &    &    &    & M7c&    &    &    &    &    & M7d\\
   Atividade A8: redação de artigos e participação em eventos técnico-científicos &    &    &    &    & x  & M8b&    &    &    &    &    &    \\
   Atividade A9: redação de artigo para publicação em periódico                    &    &    &    &    &    &    & x  & x  & x  & x  & x  & M9 \\
\bottomrule
%\hline\hline   
\end{tabular}
	\label{tab:crono}
\end{table}
\clearpage

\newpage

\section{Orçamento}
\label{sec:orc}
%%orcamento.tex
\subsection{Recursos humanos}
A equipe de execução foi concebida como sendo composta por dois alunos de graduação (IC), dois mestrandos (MT), um doutorando (DR), dois pesquisadores seniors (PQ) e um coordenador. Tal configuração foi elaborada no sentido de obter um bom compromisso entre custo e maturidade técnica suficiente para executar o projeto no prazo estipulado. Os valores de bolsa para IC, MT e DR têm como referência os valores pagos pelo CNPq. A Tabela~\ref{tab:rhmes} mostra as despesas mensais totais e por tipo de bolsa nos primeiro e segundo anos de projeto.
A Tabela~\ref{tab:rhano} apresenta os valores anuais totais de desembolso com recursos humanos do projeto.
Por sua vez, a Tabela~\ref{tab:rhtipo} especifica o investimento total em recursos humanos por tipo.

\begin{table}[h!]
\scriptsize
	\caption{Recursos humanos -- desembolso mensal por ano}
\rowcolors{2}{lightgray}{}%lightgray
\begin{tabular}{ lrrrrrrrrrrrr}
\toprule
%\hline\hline
   \rowcolor{lgray}
   Tipo      & 01      & 02      & 03      & 04      & 05      & 06      & 07      & 08      & 09      & 10      & 11      & 12      \\
\midrule
   IC-01     &   400,00&   400,00&   400,00&   400,00&   400,00&   400,00&   400,00&   400,00&   400,00&   400,00&   400,00&   400,00\\
   IC-02     &   400,00&   400,00&   400,00&   400,00&   400,00&   400,00&   400,00&   400,00&   400,00&   400,00&   400,00&   400,00\\
   MT-01     & 1.500,00& 1.500,00& 1.500,00& 1.500,00& 1.500,00& 1.500,00& 1.500,00& 1.500,00& 1.500,00& 1.500,00& 1.500,00& 1.500,00\\
   MT-02     & 1.500,00& 1.500,00& 1.500,00& 1.500,00& 1.500,00& 1.500,00& 1.500,00& 1.500,00& 1.500,00& 1.500,00& 1.500,00& 1.500,00\\
   DR        & 2.200,00& 2.200,00& 2.200,00& 2.200,00& 2.200,00& 2.200,00& 2.200,00& 2.200,00& 2.200,00& 2.200,00& 2.200,00& 2.200,00\\
   PQ-01     & 1.500,00& 1.500,00& 1.500,00& 1.500,00& 1.500,00& 1.500,00& 1.500,00& 1.500,00& 1.500,00& 1.500,00& 1.500,00& 1.500,00\\
   PQ-02     & 1.500,00& 1.500,00& 1.500,00& 1.500,00& 1.500,00& 1.500,00& 1.500,00& 1.500,00& 1.500,00& 1.500,00& 1.500,00& 1.500,00\\
   CO        & 2.000,00& 2.000,00& 2.000,00& 2.000,00& 2.000,00& 2.000,00& 2.000,00& 2.000,00& 2.000,00& 2.000,00& 2.000,00& 2.000,00\\
	\midrule
   Total     & 5.200,00& 5.200,00& 5.200,00& 5.200,00& 5.200,00& 5.200,00& 5.200,00& 5.200,00& 5.200,00& 5.200,00& 5.200,00& 5.200,00\\
   
%   \rowcolor{lightblue}%lgray}
\midrule
\midrule
   \rowcolor{lgray}
   Tipo      & 13      & 14      & 15      & 16      & 17      & 18      & 19      & 20      & 21      & 22      & 23      & 24      \\
\midrule
   IC-01     &   400,00&   400,00&   400,00&   400,00&   400,00&   400,00&   400,00&   400,00&   400,00&   400,00&   400,00&   400,00\\
   IC-02     &   400,00&   400,00&   400,00&   400,00&   400,00&   400,00&   400,00&   400,00&   400,00&   400,00&   400,00&   400,00\\
   MT-01     & 1.500,00& 1.500,00& 1.500,00& 1.500,00& 1.500,00& 1.500,00& 1.500,00& 1.500,00& 1.500,00& 1.500,00& 1.500,00& 1.500,00\\
   MT-02     & 1.500,00& 1.500,00& 1.500,00& 1.500,00& 1.500,00& 1.500,00& 1.500,00& 1.500,00& 1.500,00& 1.500,00& 1.500,00& 1.500,00\\
   DR        & 2.200,00& 2.200,00& 2.200,00& 2.200,00& 2.200,00& 2.200,00& 2.200,00& 2.200,00& 2.200,00& 2.200,00& 2.200,00& 2.200,00\\
   PQ-01     & 1.500,00& 1.500,00& 1.500,00& 1.500,00& 1.500,00& 1.500,00& 1.500,00& 1.500,00& 1.500,00& 1.500,00& 1.500,00& 1.500,00\\
   PQ-02     & 1.500,00& 1.500,00& 1.500,00& 1.500,00& 1.500,00& 1.500,00& 1.500,00& 1.500,00& 1.500,00& 1.500,00& 1.500,00& 1.500,00\\
   CO        & 2.000,00& 2.000,00& 2.000,00& 2.000,00& 2.000,00& 2.000,00& 2.000,00& 2.000,00& 2.000,00& 2.000,00& 2.000,00& 2.000,00\\
	\midrule
   Total     & 5.200,00& 5.200,00& 5.200,00& 5.200,00& 5.200,00& 5.200,00& 5.200,00& 5.200,00& 5.200,00& 5.200,00& 5.200,00& 5.200,00\\
\bottomrule
%\hline\hline   
\end{tabular}
	\label{tab:rhmes}
\end{table}
%====================

\begin{table}[!h]
\centering
%\scriptsize
	\caption{Recursos humanos por ano do projeto}
%\rowcolors{2}{lightgray}{}%lightgray
\begin{tabular}{ lr}
\toprule
%\hline\hline
%   \rowcolor{lgray}
   Ano       & Total (R\$)  \\
\midrule
   Ano 01    &   132.000,00 \\
   Ano 02    &   132.000,00 \\
\midrule
 %  \rowcolor{lgray}
   Total     &   264.000,00 \\
\bottomrule
\end{tabular}
	\label{tab:rhano}
\end{table}
%=======================

\begin{table}[!h]
\centering
%\scriptsize
	\caption{Recursos humanos por tipo}
%\rowcolors{2}{lightgray}{}%lightgray
\begin{tabular}{ lr}
\toprule
%\hline\hline
%   \rowcolor{lgray}
   Tipo  & Total (R\$) \\
\midrule
   IC    &   19.200,00 \\
   MT    &   72.000,00 \\
   DR    &   52.800,00 \\
   PQ    &   72.000,00 \\
   CO    &   48.000,00 \\
\midrule
 %  \rowcolor{lgray}
   Total     &   264.000,00 \\
\bottomrule
\end{tabular}
	\label{tab:rhtipo}
\end{table}
 
\newpage
\subsection{Serviços de terceiros}
A título de justificativa dos itens relacionados na Tabela~\ref{tab:ter}: 
\begin{itemize}
	\item A \emph{instalação de sistema de câmeras} é normalmente oferecida ou indicada pela empresa que vende esse tipo de sistema e é recomendável fazer uso de tal serviço profissional, tanto para não se perder a garantia dos equipamentos, quanto para se poder utilizar o mais rapidamente possível o sistema de aquisição de vídeo. Esse último aspecto é de grande importância para o atendimento dos prazos do projeto;
	\item O item \emph{computação em nuvem} tem por objetivo resgardar os prazos de entrega nas diferentes fases da execução contra atrasos na aquisição de computadores de alto desempenho, depreciação cambial (o que impediria a aquisição de computadores com a qualidade/preço inicialmente cotados), e eventuais panes nos sistemas de processamento;
	\item As \emph{taxas open access} visam prover recurso para o pagamento referente à publicação em periódicos de acesso livre de prestígio internacional de artigos técnicos relacionados à produção do projeto.
\end{itemize}
\begin{table}[!h]
\centering
%\scriptsize
	\caption{Serviços de terceiros por mês}
%\rowcolors{2}{lightgray}{}%lightgray
\begin{tabular}{clr}
\toprule
%\hline\hline
%   \rowcolor{lgray}
	Mês   & Descrição & Custo (R\$) \\
	\midrule
	04    & Instalação de sistema de câmeras   & 2.000,00 \\
	07    & Computação em nuvem                & 3.000,00 \\
	08    & Computação em nuvem                & 3.000,00 \\
	09    & Computação em nuvem                & 3.000,00 \\
	10    & Computação em nuvem                & 3.000,00 \\
	11    & Computação em nuvem                & 3.000,00 \\
	12    & Computação em nuvem                & 3.000,00 \\
	13    & Computação em nuvem                & 3.000,00 \\
	14    & Computação em nuvem                & 3.000,00 \\
	15    & Computação em nuvem                & 3.000,00 \\
	16    & Computação em nuvem                & 3.000,00 \\
	17    & Computação em nuvem                & 3.000,00 \\
	17    & Computação em nuvem                & 3.000,00 \\
	24    & Taxas Open Access                  &10.500,00 \\
\midrule
 %  \rowcolor{lgray}
	      & Total                              &48.500,00 \\
\bottomrule
\end{tabular}
	\label{tab:ter}
\end{table}
\newpage

\subsection{Custeio}
Os itens de custeio apresentados na Tabela~\ref{tab:custeio} referem-se principalmente a material de escritório a ser utilizado necessariamente em atividades relacionadas à execução do projeto de pesquisa.
\begin{table}[!h]
\centering
%\scriptsize
	\caption{Custeio por mês}
%\rowcolors{2}{lightgray}{}%lightgray
\begin{tabular}{clrcr}
\toprule
%\hline\hline
%   \rowcolor{lgray}
	Mês   & Descrição              & Custo (R\$) & Quantidade & Total \\
	\midrule
	01    & Toner para impressora  & 400,00      & 01         & 400,00 \\
	01    & Resma de papel A4      &  25,00      & 10         & 250,00 \\
	13    & Toner para impressora  & 400,00      & 01         & 400,00 \\
	13    & Resma de papel A4      &  25,00      & 10         & 250,00 \\
\midrule
 %  \rowcolor{lgray}
	      &                        &             & Total      &1.300,00 \\
\bottomrule
\end{tabular}
	\label{tab:custeio}
\end{table}
 
\subsection{Capital}
Os itens de capital relacionados na Tabela~\ref{tab:capital} consistem em equipamentos para aquisição (câmera de vigilância, gravador digital, e hd) e processamento (desktop, workstation, e notebook) de vídeo digital. Os desktops e workstation possuem configuração para suportar as fases de treinamento e produção de sistemas baseados em aprendizado de máquina que lidem com grande volume de dados, o que é típico em aplicaçãoes de reconhecimento de atividades por vídeo e redes neurais convolucionais. Os notebooks são de alto desempenho e devem apoiar a realização de testes em campo do sistema de reconhecimento de atividades.
\begin{table}[!h]
\centering
\scriptsize
	\caption{Capital por mês}
%\rowcolors{2}{lightgray}{}%lightgray
\begin{tabular}{clp{0.35\textwidth}rcr}
\toprule
%\hline\hline
%   \rowcolor{lgray}
	Mês & Equipamento & Descrição               & Custo (R\$) & Quantidade & Total    \\
	\midrule
	01  & Câmera de vigilância  & Speed dome; Compactação de vídeo: H.264 (MPEG-4 Parte 10/AVC) Motion JPEG; Resoluções: HDTV 720p 1280x720 a 320x180; Taxa de quadros: H.264: Até 30/25 fps (60/50 Hz) em todas as resoluções Motion JPEG: Até 30/25 fps (60/50 Hz) em todas as resoluções & 6.000,00    & 02         &  12.000,00 \\
	01  & Gravador digital      & Gravador digital para sistema de câmeras de vigilância, 4 canais                     & 3.000,00    & 01         &   3.000,00 \\
	01  & HD                    & HD para gravador digital 8 TB                                                        & 2.500,00    & 01         &   2.500,00 \\
	01  & Desktop               & iAMD Ryzen 7 3700X (8 Núcleos e 16 Threads, 3.6GHz, Turbo até 4.4GHz, Cache de 32MB); Nvidia GeforceTM GTX 1650 Super 4GB 1280 cuda cores; RAM 16GB DDR4; SSD 1TB Workstation; Placa de rede Wireless Dual 802.11 AC;bMouse e teclado com fio; Monitor: 23.8" (1920x1080) (HDMI, DP); Frete: Transportadora com seguro - Grátis (CIF)                                                                                          & 10.757,34   & 04         &  43.029,36 \\
	13  & Workstation           & AMD Ryzen Threadripper 3970X (32 Núcleos e 64 Threads, 3.7GHz, Turbo até 4.5GHz, Cache de 144MB); Nvidia GeforceTM RTX 2080 TI 11GB 3584 cuda cores; 64GB DDR4 3200MHz (4x16GB); SSD M.2 PCIe X4 NVMe 1TB Workstation Class; HDD 2 TB 7200RPM 128MB SATA III Enterprise Class; Refrigeração: Refrigeração líquida dupla; Rede: Integrada 10/100/1000;Frete: Transportadora com seguro - Grátis (CIF)                           & 48.070,55   & 01         &  48.070,55 \\
	13  & Notebook              & Intel® Core™ i7 1065G7 1,3 GHz, 8 MB Cache, 16 GB (8 GB Onboard + 8 GB Offboard), 512 GB SSD PCIe NVME M2, Tela > 15'', Placa de Vídeo GeForce MX330 com 2GB de memória GDDR5; wifi 802.11ac, câmera frontal, Bluetooth, saídas: 1x HDMI 1.4 1x 3.5mm Combo Audio Jack 2x USB 2.0 Type-A 1x USB 3.2 Gen 1 Type-A 1x USB 3.2 Gen 1 Type-C,                                                                                      & 8.000,00    & 02         &  16.000,00 \\
\midrule
 %  \rowcolor{lgray}
             &                      &                                                                                      &             & Total      & 124.599,91 \\
\bottomrule
\end{tabular}
	\label{tab:capital}
\end{table}
\newpage

\subsection{Passagens e diárias}
Na Tabela~\ref{tab:viagem} estão especificados recursos para viabilizar a participação de pesquisadores em eventos técnicos de abrangência nacional e internacional. A finalidade de tais participações seria a atualização técnica da equipe e a apresentação de trabalhos produzidos no âmbito do projeto e aprovados por comitê de revisores. Face ao tempo para execução do projeto, estima-se a possibilidade a participação em dois eventos nacionais e um evento internacional. Para os eventos nacionais, é prevista a participação de dois membros do projeto, com cinco diárias de manutenção cada. Para o evento internacional, prevê-se a participação de um pesquisador da equipe e sete diárias. Para os valores de diárias nacionais e internacionais, utilizou-se como referência os valores correspondentes pagos pelo CNPq.
\begin{table}[!h]
\centering
%\scriptsize
	\caption{Passagens e diárias por mês}
%\rowcolors{2}{lightgray}{}%lightgray
\begin{tabular}{lcccrcr}
\toprule
%\hline\hline
%   \rowcolor{lgray}
	Tipo                         & Mês & Localidade & Finalidade               & Custo (R\$) & Quantidade & Total    \\
	\midrule
	Passagem aérea nacional      & 12  & a definir  & Apresentação de trabalho & 3.300,00    & 02         &  6.600,00 \\
	Diárias (nacional)           & 12  & a definir  & Manutenção em viagem     &   320,00    & 10         &  3.200,00 \\
	Passagem aérea nacional      & 24  & a definir  & Apresentação de trabalho & 3.300,00    & 02         &  6.600,00 \\
	Diárias (nacional)           & 24  & a definir  & Manutenção em viagem     &   320,00    & 10         &  3.200,00 \\
	Passagem aérea internacional & 24  & a definir  & Apresentação de trabalho & 7.000,00    & 01         &  7.000,00 \\
	Diárias (internacional)      & 24  & a definir  & Manutenção em viagem     & 2.220,00    & 07         & 15.540,00 \\
\midrule
 %  \rowcolor{lgray}
	                             &     &            &                          &             & Total      & 42.140,00 \\
\bottomrule
\end{tabular}
	\label{tab:viagem}
\end{table}
 
\subsection{Geral}
Nesta seção, apresentamos um resumo do orçamento. Na Tabela~\ref{tab:geral-mes-tipo} encontra-se o desembolso mensal para cada tipo de item de orçamento.
A Tabela~\ref{tab:geral-ano-rubrica}, por sua vez, exibe o investimento anual por rubrica. Nesse caso, a coluna \emph{Custeio} reúne \emph{Terceiros}, \emph{Consumo}, e \emph{Passagens/Diárias}.
\begin{table}[!h]
\scriptsize
	\caption{Desembolso mensal por tipo}
\rowcolors{2}{lightgray}{}%lightgray
\centering
\begin{tabular}{lcrrrrrr}
\toprule
%\hline\hline
   \rowcolor{lgray}
	Ano  & Mês      & Bolsas  & Terceiros  & Consumo & Pass/Diárias & Capital    & Valor \\
\midrule
       Ano 1 & 01       &  11.000,00 &      0,00 & 650,00  &     0,00     & 60.529,36 &  72.179,36\\
             & 02       &  11.000,00 &      0,00 &   0,00  &     0,00     &      0,00 &   11.000,00\\
             & 03       &  11.000,00 &      0,00 &   0,00  &     0,00     &      0,00 &   11.000,00\\
             & 04       &  11.000,00 &  2.000,00 &   0,00  &     0,00     &      0,00 &   13.000,00\\
             & 05       &  11.000,00 &      0,00 &   0,00  &     0,00     &      0,00 &   11.000,00\\
             & 06       &  11.000,00 &      0,00 &   0,00  &     0,00     &      0,00 &   11.000,00\\
             & 07       &  11.000,00 &  3.000,00 &   0,00  &     0,00     &      0,00 &   14.000,00\\
             & 08       &  11.000,00 &  3.000,00 &   0,00  &     0,00     &      0,00 &   14.000,00\\
             & 09       &  11.000,00 &  3.000,00 &   0,00  &     0,00     &      0,00 &   14.000,00\\
             & 10       &  11.000,00 &  3.000,00 &   0,00  &     0,00     &      0,00 &   14.000,00\\
             & 11       &  11.000,00 &  3.000,00 &   0,00  &     0,00     &      0,00 &   14.000,00\\
             & 12       &  11.000,00 &  3.000,00 &   0,00  & 9.800,00     &      0,00 &   23.800,00\\
\midrule
	     & Subtotal & 132.000,00 & 20.000,00 & 650,00  & 9.800,00     & 60.529,36 & 222.979,36\\
%   \rowcolor{lightblue}%lgray}
\midrule
\midrule
   \rowcolor{lgray}
	Ano  & Mês      & Bolsas  & Terceiros  & Consumo & Pass/Diárias & Capital    & Valor \\
\midrule
       Ano 2 & 13       &  11.000,00 &  3.000,00 & 650,00  &      0,00   & 64.070,55 &  78.720,55\\
             & 14       &  11.000,00 &  3.000,00 &   0,00  &      0,00   &      0,00 &  14.000,00\\
             & 15       &  11.000,00 &  3.000,00 &   0,00  &      0,00   &      0,00 &  14.000,00\\
             & 16       &  11.000,00 &  3.000,00 &   0,00  &      0,00   &      0,00 &  14.000,00\\
             & 17       &  11.000,00 &  3.000,00 &   0,00  &      0,00   &      0,00 &  14.000,00\\
             & 18       &  11.000,00 &  3.000,00 &   0,00  &      0,00   &      0,00 &  14.000,00\\
             & 19       &  11.000,00 &      0,00 &   0,00  &      0,00   &      0,00 &   11.000,00\\
             & 20       &  11.000,00 &  2.000,00 &   0,00  &      0,00   &      0,00 &   11.000,00\\
             & 21       &  11.000,00 &      0,00 &   0,00  &      0,00   &      0,00 &   11.000,00\\
             & 22       &  11.000,00 &      0,00 &   0,00  &      0,00   &      0,00 &   11.000,00\\
             & 23       &  11.000,00 &      0,00 &   0,00  &      0,00   &      0,00 &   11.000,00\\
             & 24       &  11.000,00 & 10.500,00 &   0,00  & 32.340,00   &      0,00 &  53.840,00\\
\midrule
	     & Subtotal & 132.000,00 & 28.500,00 & 650,00  & 32.340,00   & 64.070,55 & 257.560,55\\
%   \rowcolor{lgray}
\bottomrule
%\hline\hline   
\end{tabular}
	\label{tab:geral-mes-tipo}
\end{table}

\begin{table}[!h]
%\scriptsize
	\caption{Desembolso anual por rubrica}
%\rowcolors{2}{lightgray}{}%lightgray
\centering
\begin{tabular}{crrrr}
\toprule
%\hline\hline
   %\rowcolor{lgray}
	Ano   & RH         & Custeio   & Capital    & Total \\
\midrule
	Ano 1 & 132.000,00  & 30.450,00 & 60.529,36  & 222.979,36\\
	Ano 2 & 132.000,00  & 61.490,00 & 64.070,55  & 257.560,55\\
\midrule
	Total & 264.000,00 & 91.940,00 & 124.599,91 & 480.539,91\\
\bottomrule
%\hline\hline   
\end{tabular}
	\label{tab:geral-ano-rubrica}
\end{table}
\clearpage
%

\newpage

%\section{Referências}
%\label{sec:ref}
\printbibliography
%===

\end{document}
