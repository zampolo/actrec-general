\documentclass[a4paper,10pt]{article}

%=== preambulo === 
\usepackage[brazilian]{babel}
\usepackage{color,enumerate}
\usepackage{graphicx}
%\usepackage[toc,page]{appendix}
%\usepackage[table]{xcolor}
%\usepackage{booktabs}
\usepackage[T1]{fontenc}
\usepackage[utf8]{inputenc}
\usepackage{subfigure}
\usepackage{fullpage}
\usepackage[affil-it,auth-sc]{authblk}
\usepackage[backend=biber, style=authoryear, doi=false, isbn=false, url=false]{biblatex}
%\addbibresource{bib/?.bib}

\renewcommand\Authand{ e }
\renewcommand\Authands{, e }

\date{Julho de 2020}


\title{Sistemas de visão computacional baseados em aprendizagem profunda para reconhecimento e análise de atividades humanas}

\author[1]{Adriana Rosa Garcez Castro}
\author[2]{Ronaldo de Freitas Zampolo}

\affil[2]{Laboratório de Processamento de Sinais\\
Instituto de Tecnologia\\
Universidade Federal do Pará\\
66075-110 Belém, PA}

\begin{document}
\maketitle

\begin{table}[!th]
\begin{tabular}{|l p{0.7\textwidth}|}
\hline
\textbf{Pesquisadores}        & Adriana Rosa Garcez Castro\\
	                      & Ronaldo de Freitas Zampolo \\
%\textbf{Cargo atual}          & Professor associado\\
\textbf{Instituição}          & Universidade Federal do Pará (UFPA)\\
%\textbf{Data de nascimento}   & 10 de fevereiro de 1973\\
%\textbf{Local de nascimento}  & São Paulo, SP, Brasil\\
\textbf{Endereço profissional}& Laboratório de Processamento de Sinais\\
                              & Anexo do LEEC, altos, sala 32\\
                              & Av. Augusto Correa, 1\\
                              & 66070-110 Guamá Belém, PA, Brasil\\
%\textbf{\slang{Endereço residencial}{Home address}}  & Av. Pedro Miranda, 1929\\
%                                		     & Ed. Pollux, ap 602\\
%                                		     & 66085-024 Pedreira Belém, PA, \slang{Brasil}{Brazil}\\
\textbf{Telefone}             & +55 91 3201-7674 \\
%                              & +55 91 3244-3578 (\slang{res.)}{home)}\\
%                              & +55 91 8119-8840 (celular)\\
\textbf{Emails}               & adcastro@ufpa.br\\
	                      & zampolo@ufpa.br\\
%                              & zampolo@ieee.org\\
\textbf{Título do projeto}    & Sistemas de visão computacional baseados em aprendizagem profunda para reconhecimento e análise de atividades humanas\\
\textbf{Área}                 & Reconhecimento de atividades humanas. Visão computacional. Aprendizagem profunda. Redes neurais convolucionais.\\
\textbf{Período de execução}  & 24 meses\\
%\textbf{Supervisor}             & Prof.~Patrick Le Callet\\
%                                & École Polytechnique de Nantes\\
%                                & Université de Nantes\\
%                                & IRCCYN/IVC\\
%                                & Rue Christian Pauc La Chantrerie\\
%                                & BP 50609 44306 Nantes Cedex 3 \slang{França}{France}\\
\hline
\end{tabular}
\end{table}
\newpage
%----------------------------------------------------------------------------

\tableofcontents\newpage


%\begin{abstract}
%reconhecer ações e objetivos
mediante observações dos "agentes" e do ambiente

resposta personalizada em diferentes aplicações

conexões com medicina, psicologia, interação humano-computador, sociologia, segurança, etc.

tipos:
  - baseado em sensores, usuário único: rede de sensores + mineração de dados + aprendizado de máquina, smartphones, consumo diário de energia, atividade ou sedentarismo, covid-19: estimativas de isolamento social,
  - baseado em sensores, multi-usuário: coleta de informações simultâneas de vários usuários por meio de sensores (normalmente vestíveis), covid- 19 (agora sim)
  - baseado em sensores, grupo (qual a diferença em relação ao anterior?): uma possível resposta à pergunta: reconhecer o comportamento de um grupo como entidade, ao invés de identificar o comportamento individual dos membros de um grupo, comportamento do grupo x comportamento dos indivíduos; principais desafios: modelar o papel do comportamento individual na dinâmica coletiva; comportamento de multidões (Círio de Nazaré), gerenciamento de multidões (crowd management) e respostta a situações de emergência.

Abordagens:
  - Através de lógica e reasoning
    - explicações consistentes das ações observadas
    - kautz's : complexidade exponencial no pior caso
    - parece um sistema baseado em regras, tipo árvore de decisão
    - dificuldade para representar incerteza, incapacidade de aprendizado ao longo do tempo
  - Através de inferência prbabilística
    - identificação de atividades diárias, usando sensores (lembrar das discussões sobre monitoramento de idosos, usando vestíveis e câmeras)
    - sensores: RFID, GPS (HMM, Dynamic Bayesian networks)
  - Baseado em mineração de dados:
    - problema de reconhecimento de padrões
    - definição de classes de atividade
  - Baseado em GPS

Uso de sensores:
  - Baseados em visão computacional
    - multi-câmera
    - fluxo-optico, filtro de Kalman, hmm, 
    - câmera única, stereo, ir
    - rastreamento de pedestre, rastreamento de grupo, deteção de objetos largados
    - câmeras rgbd
    - deep learning techniques: classificação de vídeo, detecção de início e fim de atividades, localização espaço-temporal de atividades
    - integração com commonsense reasoning and commonsense knowledge
    - Níveis:
      - detecção de pessoas
      - rastreamento de pessoas
      - reconhecimento de atividades humanas
      - avaliação de atividade de alto-nível.
    - Markov Networks, CNN e LSTM
    - Reconhecimento por marcha, análise de marcha,

- caracterizar contexto, e não mais informação específica quadro a quadro
- o que é e porque é difícil
  - classificação por múltiplos quadros
  - ImageNet
  - Dificuldades:
    - Custo computacional elevado:
      - 3 a 4 dias para treinar uma rede na base ucf101 e aproximadamente dois meses para a base sports-1m
    - Captura de longo contexto:
      - capturar contexto espaço-temporal ao longo dos quadros
    - Projeto de arquiteturas para classificação
      - são muitas as opções
      - a avaliação de cada uma das opções é custosa
    - Não há bechmark padronizado
      - ucf101 e sports1m: old datasets ?
      - one can use youtube?
      - kinectics data set?
      
- abordagens
  - Single stream network: 
  - Two stream network:
    - uma rede para contexto espacial
    - uma rede para contexto de movimento
    - características de movimento são modeladas por vetores de fluxo optico empilhados

    - problemas: false label assignment, ou seja rotular a base é sempre complicado. 
    - segundo o material consultado: o treinamento de ambas as redes era separado. E  ainda havia uma longa estrada para treinamento "end-to-end on-the-go"

- artigos/Redes:
  - LRCN: Long-term Recurrent Convolutional Networks for visual recognition and description
    - the final prediction of each clip (16 frames) is the average of predictions across each time step, and the final prediction at video level is average of predictions from each clip
    - fluxo óptico não pode ser calculado on the fly?
    - desvantagem apontada: inabilidade de capturar informação temporal de faixa longa... por quê? a lstm usada não supriria isso, uma vez que a informação é realimentada? A não ser que a realimentação seja resetada a cada clip. 
    - aumento do número de frames que compõem um clip parece melhorar o desempenho (informação temporal de faixa longa)
  
  - C3D: Learning spatiotemporal features with 3D convolutional networks
    - 3d convolutional nerworks as feature extractors
    - using deconvolutional layers to interpret model decision
    - a simple linear classifier like svm on top of ensemble of extracted features worked better than the sota algorithms, performed wven better if hand crafted featureswere used
    - findings: the net focussed on spatial appearance in first few frames and thacked the motion in the subsequente frames.

  - Conv3D \& Attention
    - Describing videos by exploiting temporal structure
    - among the contributions of the paper: use of an **attention mechanism** within a cnn-rnn encoder-decoder framework to capture global context
    - 3d cnn + lstm: base architecture for video description task.
    - Algorith:
      - 3d cnn feature maps for a clip: concatenation with stacked 2d feature maps for the same set of frames
      - 2d and 3d cnn are pre-trained
      - weighted average is used to combine the temporal features
      - attention weights are decided based on lstm output at every time step.

  - TwoStreamFusion
    - Convolutional two-stream network fusion for video action recognition
      - multi-level fused architecture ?
      - fusion of spatial and temporal streams (how and when) for task discrimination: at early stages, rather than at the end of the pipeline.
      - long-term dependencies are also modeled.
      * The authors stablished the **supremacy** of the **TwoStreamFusion** method: improved performance withou extra parameters (in comparison with C3D)

  - TSN
    - Temporal segments networks: towards good practices for deep action recognition
      - long range temporal modeling effective solution
      - usage of batch normalisation, dropout, and pre-training
      - sparse clip sampling across the video: to better model long range temporal signal
      - final prediction at video-level: 
        - combine scores from temporal and spatial streams
        - Fusing score of final spatial and temporal scores using weighted average and applying softmax over all classes
      - main challenges addressed: overfitting due to small sizes; and long range modeling

  - ActionVlad
    - ActionVlad: learning spatio-temporal aggregation for action classification
      - learnable video-level aggregation of features
      - end-to-end trainable model to capture long term dependency
      - main contribution: vlad (video-level aggregation ...?) rather than normal aggregation (maxpool or avgpool, for instance)
      - VLAD: effective way of pooling ! sota in early 2017

  - HiddenTwoStream
    - Hidden two-stream convolutional networks for action recognition
      - generation of optical flow input on-the-fly (by using a separate network)
      - the paper advocates the usage of an unsupervised architecture to generate optical flow for a stack of frames.
      - optical flow: image reconstruction problem
      - temporal stream:
        - it has an optical flow generation net on the top
        - the input is subsequent frames rather than preprocessed optical flow
      
  - I3D
    - Quo vadis, action recognition? A new model and the Kinetics database
      - 3d models combined into two stream achitecture 
      - two different 3d networks for both streams (spatial and temporal)
      - 2d pre trained weights are repeated in 3d !!!
      - major contribution: benefit of using pre trained 2D conv nets

  - T3D
    - Temporal 3d convnets: new architecture and transfer learning for video classification
      - combine temporal information across variable depth
      - new technique to supervise transfer learning between 2d pre-trained net to 3d net
      - multi-depth temporal pooling layer
Todas as abordagens acima são variações/combinações das redes básicas a seguir:
  - LSTM: long shot term memory
  - 3D-ConvNet
  - Two-stream
  - 3D-Fused Two-Stream
  - Two-Stream 3D-ConvNet



%\end{abstract}
%===

\section{Breve descrição do problema}
\label{sec:descr}
% verificar os desafios do último artigo lido para incluir aqui (se necessário)
% uma aplicação:
% - segurança em ambientes industriais (é o que dá maiores possibilidades para financiamento):
%   - deteção de pessoas;
%   - contagem de pessoas;
%   - presença de EPI;
%   - identificação de ações (quais?)
%
% Desafios: 
% - Variação antropométricas
% - Variação de ângulos de visão para diferentes câmeras no ambiente monitorado
% - Diferentes condições de iluminaçao
% - Baixa qualidade dos vídeos ? qualidade vs taxas de reconhecimento
% - Oclusão
% - Variação de iluminação e presença de sombras
% - Dados insuficientes
% - Variações climáticas
%
% Abordagem:
% -
% 
Sistemas para reconhecimento automático de ações apresentam desempenho cada vez mais próximo da capacidade humana de interpretação visual, graças à popularização das técnicas de aprendizado de máquina, em especial de redes neurais profundas. 
%

O reconhecimento de atividades humanas visa identificar o tipo de ação executada por uma ou mais pessoas em uma cena registrada em vídeo não exibido anteriormente ao sistema.
%
A tarefa apresenta desafios técnicos relevantes, cujo grau de dificuldade depende de diversos elementos, tais como: número de pessoas envolvidas; complexidade da ação de interesse; características do ambiente em que ocorre a ação; grau de interação entre componentes (pessoas e objetos) da atividade; e qualidade do vídeo sob análise. 
%

Para fins de definição de escopo do projeto e atendimento a demandas de setores produtivos locais, optou-se pela aplicação em sistemas de apoio à proteção e segurança em ambientes industriais. Dentre as funcionalidades previstas, destacam-se: detecção e contagem de pessoas nos ambientes monitorados; verificação de presença e utilização correta de equipamento de proteção individual obrigatório; alerta de comportamentos anômalos; e identificação de ações potencialmente perigosas para integridade de trabalhadores e da operação.
%

A presente proposta de pesquisa concentra-se na investigação de novas técnicas em reconhecimento e predição de ações humanas, mediante o emprego de estratégias baseadas em visão computacional e aprendizagem profunda, bem como na implementação de provas de conceito que demonstrem a viabilidade dos sistemas concebidos. 
%
Em redes neurais profundas, duas abordagens principais se consolidaram para tratar o problema em questão: uso de redes \emph{multi-stream} e uso de redes convolucionais 3D. Ambas procuram combinar análise espacial (intra-quadro) e dependências temporais de eventos (inter-quadros) na caracterização das ações monitoradas. 
%

As contribuições científicas visadas pelo estudo proposto concentram-se em três desafios atuais da área: o uso de dados de entrada multi-modais para melhorar o desempenho do reconhecimento visual; a exploraração de mecanismos de interpretação dos sinais de entrada para identificar atributos que permitam reconhecer ações no curto prazo; e o desenvolvimento de abordagens para uso de redes pré-treinadas e o treinamento com conjunto de dados parcialmente rotulado.
%
%Apesar de seus resultados representarem o estado da arte, são conhecidos os problemas da aprendizagem profunda com impacto direto na capacidade de generalização das redes: a necessidade de grande poder computacional e de grande volume de dados rotulados na fase de treinamento (aprendizado supervisionado). Como alternativa, a fim de contornar tais restrições, pode-se empregar o uso de apredizado semi-supervisionado com base de treinamento parcialmente anotada, bem como proceder ao ajuste de parâmetros de redes previamente treinadas (transferência de conhecimento).
Outros ganhos esperados são: formação de recursos humanos especializados em projeto e implementação de sistemas baseados em aprendizado de máquina; prevenção de acidentes em espaços industriais que comprometam a saúde de colaboradores ou possuam efeito contaminante para o ambiente; e aumento de competitividade dos parceiros envolvidos pela inovação em produtos e processos advinda da pesquisa.


\newpage
%===                                          

\section{Revisão da literatura científica}
\label{sec:rev}
\nocite{jegham-2020, hussain-2020,yao-2019,kongr-2018,herath-2017}
%#Elements:
%1- Short and precise **overview**; 
%  - Concentuação/Taxonomia: o que é o reconhecimento de ações (ações ou atividades?, interação, )
%  - Tipos de sistemas. Diferentes classificações: número de pessoas envolvidas, ambiente, interação com pessoas e/ou objetos, predição/reconhecimento, uso de hadcrafted features ou extração automática de features/atributos usando cnns, ou ainda abordagens híbridas
Este projeto trata do reconhecimento automático de atividades humanas mediante o uso de visão computacional baseada em redes profundas. Em razão da falta de consenso sobre os termos usados nas área envolvidas, é interessante que algumas definições sejam postas a fim de caracterizar o escopo deste projeto bem como o objeto de pesquisa. 
%
Assim, a primeira distinção a ser feita é entre \emph{ação} e \emph{atividade}. Sem maiores aprofundamentos, e a despeito das controvérsias, considera-se neste texto que uma \emph{ação} é um evento primitivo, simples, fácil de caracterizar, e que é realizado por uma única pessoa (por exemplo, \emph{homem levanta o braço}). Por outro lado, a sucessão de ações diferentes forma \emph{atividades}, essas mais ou menos complexas (por exemplo, \emph{preparar café}). 
%
Outro aspecto a ser considerado, diz respeito à quantidade de elementos participantes de uma cena. Pode-se ter situações em que apenas uma pessoa está presente; em outras, pode-se ter dois ou mais indivíduos interagindo; e ainda, o(s) indivíduo(s) pode(m) interagir com objetos. As possibilidades de variação aqui são inúmeras, influenciando no grau de complexidade da cena.
%
Cenas não são apenas compostas pelos atores principais que desempenham interações e/ou atividades de interesse, mas também pelo pelo contexto, simplificadamente denominado de \emph{fundo}, que pode variar entre \emph{estático} em ambientes controlados (em um estúdio ou laboratório, por exemplo) ou altamente dinâmicos, como em ruas de grandes cidades ou áreas com a presença de diversas pessoas realizando diferentes atividades. A dificuldade do reconhecimento aumenta com a dinamicidade do fundo, pois além da tarefa de estimar a atividade sendo desempenhada, o sistema precisa diferenciar entre movimentação dos objetos de interesse e das modificações típicas do fundo.
%
O reconhecimento automático de atividades é uma das áreas de maior interesse em visão computacional com um leque de possíveis aplicações abrangendo áreas como ambientes inteligentes, classificação e busca automática em repositórios de imagem e vídeo, e avaliação do comportamento de multidões, dentre outras.
%
%2- **Present state** of the research; 
% Abordagens:
%% Redes multi-streams
%% Redes 3D
%% Soluções híbridas.
Atualmente, a tendência e o estado da arte nessa área compreendem técnicas baseadas em redes neurais profundas. AUTOT-REF aponta como principal desafio a representação da dependência temporal entre ações na caracterização de uma atividade complexa. Tal desafio, em grande parte, tem orientado os esforços da comunidade científica nos avanços da área. 
%
Grosso modo, são duas as principais abordagens mais bem sucedidas na busca de soluções para a representação temporal. A primeira delas, denominada multi-fluxo (multi-stream), procura caracterizar informações temporais e espaciais em dois ramos separados na rede neural profunda. O ramo temporal geralmente tem como entrada o fluxo óptico (ou estimativas deste) de cada quadro de vídeo, ao passo que o ramo espacial recebe o quadro propriamente dito. Cada uma das entradas passa por etapas de extração automática de atributos nas camadas convolucionais da rede, cujos resultados são processados e fundidos oportunamente para fins da classificação final. O uso de fluxo óptico, contudo, tem suas desvantagens uma vez que seu cálculo requer quadros adjacentes ao quadro de interesse, resultando em aumento de complexidade computacional e, consequentemente, dificuldade de implementação em tempo real. A outra abordagem justamente prescinde do fluxo óptico, e compreende o uso de redes profundas 3D que poderiam reunir em um único fluxo informações temporais e espaciais simultaneamente. 
%
Ainda motivado pelas inconveniências do uso do fluxo óptico e pela necessidade de caracterização da informação temporal de longo e curto prazo no entendimento de atividades complexas, técnicas baseadas em redes profundas recursivas, tais como a LSTM (\emph{long short-term memory}) e outras REF, têm sido propostas, apresentando compromisso competitivo entre complexidade computacional e qualidade de resultados REF.

%3- **Immediate connection** with your own research; 
Particularizando para o contexto da aplicação visada neste projeto (reconhecimento de atividades humanas em áreas industriais), pode-se dizer que o ambiente típico a ser a ser monitorado apresenta alta complexidade. Avançando um pouco mais na caracterização desse tipo de ambiente, é importante considerar que outros fatores influenciarão no desempenho do reconhecimento, tais como: variação de luminosidade, de acordo com os horários de monitoramento; presença de mais de um indivíduo interagindo entre si e com objetos, de maneiras difíceis de prever na etapa de treinamento das redes; variação de aparência de uma determinada atividade, em função da distância e do ângulo entre objeto de interesse e câmeras; e ocorrência de oclusões (encobrimento de uma região de interesse por distratores).
%
Apesar bons resultados documentados em REFs, onde diversos dos elementos mencionados, típicos de uma situação realística, são considerados, observa-se acentuada degradação no desempenho dos sistemas de reconhecimento a medida em que o controle das variáveis ambientais é perdido (a expressão em inglês seria \emph{in the wild}). Ou seja, sistemas de reconhecimento automático de atividades humanas em contextos práticos constituem um desafio atual, de grande interesse, e estratégico em termos científicos e de inovação.

%4- Clear and logical **discussion of the theoretical framework**; 
Em grande medida, as limitações de desempenho em situações realísticas são devidas às grandes variações entre cenas apresentadas na fase de operação e cenas usadas na fase de treinamento. Tipicamente, redes profundas exigem  conjunto de dados de treinamento enorme a fim de apresentar boa capacidade de generalização, mesmo em condições controladas. Considerando sistemas de aprendizagem supervisionada (o mais comum), tal conjunto de treinamento requer dados previamente rotulados, exigindo considerável esforço para sua geração e manutenção. Nesse sentido, as tendências alternativas ao treinamento supervisionado completo de sistema de reconhecimento (\emph{from scratch}) que vêm se consolidado na área REF são as seguintes: transferência de conhecimento, usando redes de genéricas pré-treinadas, mediante treinamento supervisionado expondo tais redes a exemplos rotulados típicos da aplicação pretendida, procedimento conhecido pelo termo \emph{fine-tuning}; estratégias para treinamento de conjunto de dados parcialmente rotulados, quando se dispõe de grande quantidade de exemplos no conjunto de dados, mas a atribuição de rótulos possui custo elevado.
%5- **Methodological implications**;
%6- Indicate an **open problem**; 
À medida em que a área avança, outras perspectivas de uso começaram a surgir. Dentre elas, está a que pode ser considerada o próximo passo em visão computacional nessa área: a predição de atividades humanas. Em linhas gerais, pode-se considerar que a distinção entre reconhecimento e predição está na localização temporal pretendida da estimativa de atividade fornecida pelo sistema REF. Para um mesmo conjunto de entrada, o reconhecimento visa identificar a ação ocorrida ou caracterizada nesse conjunto de entrada. Por sua vez, a predição fornecerá uma estimativa da atividade que ocorrerá nos quadros de vídeo seguintes ao conjunto de entrada. O reconhecimento olha para o passado, enquanto que a predição considera o passado para projetar uma aposta sobre o que virá em seguida. Claramente, o problema da predição é bem mais desafiador, mas possui um apelo bastante evidente em direção autônoma e ambientes inteligentes voltados para segurança e prevenção de acidentes, onde melhor que remediar é previnir REF. No aspecto técnico e de implementação dos sistemas, predição e reconhecimento são semelhantes REF. Contudo, em relação aos resultados, a qualidade da predição é acentuadamente menor que em reconhecimento, aumentando ainda mais com o distanciamento temporal entre conjunto de entrada e predição. A representação temporal e da sucessão de ações/atividades assume importância ainda maior na tarefa de predição.
%7- The **contribution**  to the area
Alinhada com o estado da arte, problemas em aberto e tendências em reconhecimento automático de atividades humanas, a presente proposta de pesquisa estabelece como tópicos prioritários de investigação: o uso de dados de entrada  multimodais, ou seja outros tipos de entrada além do vídeo, para melhorar o desempenho do sistema de reconhecimento em ambientes de alta complexidade; a exploração de mecanismos de interpretação do sinais de entrada para identificar atributos que permitam reconhecer ações de curto prazo, fator chave para melhorar a qualidade de sistemas de predição visando segurança e proteção em ambientes industriais; o aperfeiçoamento de abordagens para transferência efetiva de conhecimento, mediante o uso de redes pré-treinadas; e o desenvolvimento de esquema de treinamento com conjunto de dados parcialmente rotulados.

\newpage
%===                                          

%\section{Balanço entre teoria e prática}
%\label{sec:teopratica}
%\input{teopratica}
%\newpage
%===                                          

%\section{Atividades práticas}
%\label{sec:atvprt}
%\input{estagio}
%===

%\section{Internacionalização}
%\label{sec:inter}
%\input{inter}

%\section{Acompanhamento de egressos}
%\label{sec:egresso}
%\input{egresso}

%\section{Discussão}
%\label{sec:disc}
%\input{result}
%===

%\section{Conclusão}
%\label{sec:conc}
%\input{conclusao}
%===

\section{Referências}
\label{sec:ref}
%===

\end{document}
