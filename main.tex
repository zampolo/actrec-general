\documentclass[a4paper,10pt]{article}

%=== preambulo === 
\usepackage[brazilian]{babel}
\usepackage{color,enumerate}
\usepackage{graphicx}
%\usepackage[toc,page]{appendix}
%\usepackage[table]{xcolor}
%\usepackage{booktabs}
\usepackage[T1]{fontenc}
\usepackage[utf8]{inputenc}
\usepackage{subfigure}
\usepackage{fullpage}
\usepackage[affil-it,auth-sc]{authblk}
\usepackage[backend=biber, style=authoryear, doi=false, isbn=false, url=false]{biblatex}
\addbibresource{bib/actrec.bib}

\renewcommand\Authand{ e }
\renewcommand\Authands{, e }

\date{Agosto de 2020}%Julho 


\title{Sistemas de visão computacional baseados em aprendizagem profunda para reconhecimento de atividades humanas}

\author[1]{Ronaldo de Freitas Zampolo}
\author[2]{Adriana Rosa Garcez Castro}
\author[2]{Agostinho Luiz da Silva Castro}

\affil[1]{Laboratório de Processamento de Sinais\\
Instituto de Tecnologia\\
Universidade Federal do Pará\\
66075-110 Belém, PA}

\affil[2]{Laboratório de Monitoramento Inteligente e Soluções em Telecomunicações\\
Instituto de Tecnologia\\
Universidade Federal do Pará\\
66075-110 Belém, PA}

\begin{document}
\maketitle

\begin{table}[!th]
\begin{tabular}{|l p{0.7\textwidth}|}
\hline
\textbf{Pesquisadores}        & Ronaldo de Freitas Zampolo \\
			      & Adriana Rosa Garcez Castro\\
			      & Agostinho Luiz da Silva Castro\\
%\textbf{Cargo atual}          & Professor associado\\
\textbf{Instituição}          & Universidade Federal do Pará (UFPA)\\
%\textbf{Data de nascimento}   & 10 de fevereiro de 1973\\
%\textbf{Local de nascimento}  & São Paulo, SP, Brasil\\
\textbf{Endereço profissional}& Laboratório de Processamento de Sinais\\
                              & Anexo do LEEC, altos, sala 32\\
                              & Av. Augusto Correa, 1\\
                              & 66070-110 Guamá Belém, PA, Brasil\\
%\textbf{\slang{Endereço residencial}{Home address}}  & Av. Pedro Miranda, 1929\\
%                                		     & Ed. Pollux, ap 602\\
%                                		     & 66085-024 Pedreira Belém, PA, \slang{Brasil}{Brazil}\\
\textbf{Telefone}             & +55 91 3201-7674 \\
%                              & +55 91 3244-3578 (\slang{res.)}{home)}\\
%                              & +55 91 8119-8840 (celular)\\
\textbf{Emails}               & zampolo@ufpa.br\\
%	                      & zampolo@ieee.org\\
\textbf{Título do projeto}    & Sistemas de visão computacional baseados em aprendizagem profunda para reconhecimento de atividades humanas\\
\textbf{Área}                 & Reconhecimento de atividades humanas. Visão computacional. Aprendizagem profunda. Redes neurais convolucionais.\\
\textbf{Período de execução}  & 24 meses\\
%\textbf{Supervisor}             & Prof.~Patrick Le Callet\\
%                                & École Polytechnique de Nantes\\
%                                & Université de Nantes\\
%                                & IRCCYN/IVC\\
%                                & Rue Christian Pauc La Chantrerie\\
%                                & BP 50609 44306 Nantes Cedex 3 \slang{França}{France}\\
\hline
\end{tabular}
\end{table}
\newpage
%----------------------------------------------------------------------------

\tableofcontents\newpage


%\begin{abstract}
%reconhecer ações e objetivos
mediante observações dos "agentes" e do ambiente

resposta personalizada em diferentes aplicações

conexões com medicina, psicologia, interação humano-computador, sociologia, segurança, etc.

tipos:
  - baseado em sensores, usuário único: rede de sensores + mineração de dados + aprendizado de máquina, smartphones, consumo diário de energia, atividade ou sedentarismo, covid-19: estimativas de isolamento social,
  - baseado em sensores, multi-usuário: coleta de informações simultâneas de vários usuários por meio de sensores (normalmente vestíveis), covid- 19 (agora sim)
  - baseado em sensores, grupo (qual a diferença em relação ao anterior?): uma possível resposta à pergunta: reconhecer o comportamento de um grupo como entidade, ao invés de identificar o comportamento individual dos membros de um grupo, comportamento do grupo x comportamento dos indivíduos; principais desafios: modelar o papel do comportamento individual na dinâmica coletiva; comportamento de multidões (Círio de Nazaré), gerenciamento de multidões (crowd management) e respostta a situações de emergência.

Abordagens:
  - Através de lógica e reasoning
    - explicações consistentes das ações observadas
    - kautz's : complexidade exponencial no pior caso
    - parece um sistema baseado em regras, tipo árvore de decisão
    - dificuldade para representar incerteza, incapacidade de aprendizado ao longo do tempo
  - Através de inferência prbabilística
    - identificação de atividades diárias, usando sensores (lembrar das discussões sobre monitoramento de idosos, usando vestíveis e câmeras)
    - sensores: RFID, GPS (HMM, Dynamic Bayesian networks)
  - Baseado em mineração de dados:
    - problema de reconhecimento de padrões
    - definição de classes de atividade
  - Baseado em GPS

Uso de sensores:
  - Baseados em visão computacional
    - multi-câmera
    - fluxo-optico, filtro de Kalman, hmm, 
    - câmera única, stereo, ir
    - rastreamento de pedestre, rastreamento de grupo, deteção de objetos largados
    - câmeras rgbd
    - deep learning techniques: classificação de vídeo, detecção de início e fim de atividades, localização espaço-temporal de atividades
    - integração com commonsense reasoning and commonsense knowledge
    - Níveis:
      - detecção de pessoas
      - rastreamento de pessoas
      - reconhecimento de atividades humanas
      - avaliação de atividade de alto-nível.
    - Markov Networks, CNN e LSTM
    - Reconhecimento por marcha, análise de marcha,

- caracterizar contexto, e não mais informação específica quadro a quadro
- o que é e porque é difícil
  - classificação por múltiplos quadros
  - ImageNet
  - Dificuldades:
    - Custo computacional elevado:
      - 3 a 4 dias para treinar uma rede na base ucf101 e aproximadamente dois meses para a base sports-1m
    - Captura de longo contexto:
      - capturar contexto espaço-temporal ao longo dos quadros
    - Projeto de arquiteturas para classificação
      - são muitas as opções
      - a avaliação de cada uma das opções é custosa
    - Não há bechmark padronizado
      - ucf101 e sports1m: old datasets ?
      - one can use youtube?
      - kinectics data set?
      
- abordagens
  - Single stream network: 
  - Two stream network:
    - uma rede para contexto espacial
    - uma rede para contexto de movimento
    - características de movimento são modeladas por vetores de fluxo optico empilhados

    - problemas: false label assignment, ou seja rotular a base é sempre complicado. 
    - segundo o material consultado: o treinamento de ambas as redes era separado. E  ainda havia uma longa estrada para treinamento "end-to-end on-the-go"

- artigos/Redes:
  - LRCN: Long-term Recurrent Convolutional Networks for visual recognition and description
    - the final prediction of each clip (16 frames) is the average of predictions across each time step, and the final prediction at video level is average of predictions from each clip
    - fluxo óptico não pode ser calculado on the fly?
    - desvantagem apontada: inabilidade de capturar informação temporal de faixa longa... por quê? a lstm usada não supriria isso, uma vez que a informação é realimentada? A não ser que a realimentação seja resetada a cada clip. 
    - aumento do número de frames que compõem um clip parece melhorar o desempenho (informação temporal de faixa longa)
  
  - C3D: Learning spatiotemporal features with 3D convolutional networks
    - 3d convolutional nerworks as feature extractors
    - using deconvolutional layers to interpret model decision
    - a simple linear classifier like svm on top of ensemble of extracted features worked better than the sota algorithms, performed wven better if hand crafted featureswere used
    - findings: the net focussed on spatial appearance in first few frames and thacked the motion in the subsequente frames.

  - Conv3D \& Attention
    - Describing videos by exploiting temporal structure
    - among the contributions of the paper: use of an **attention mechanism** within a cnn-rnn encoder-decoder framework to capture global context
    - 3d cnn + lstm: base architecture for video description task.
    - Algorith:
      - 3d cnn feature maps for a clip: concatenation with stacked 2d feature maps for the same set of frames
      - 2d and 3d cnn are pre-trained
      - weighted average is used to combine the temporal features
      - attention weights are decided based on lstm output at every time step.

  - TwoStreamFusion
    - Convolutional two-stream network fusion for video action recognition
      - multi-level fused architecture ?
      - fusion of spatial and temporal streams (how and when) for task discrimination: at early stages, rather than at the end of the pipeline.
      - long-term dependencies are also modeled.
      * The authors stablished the **supremacy** of the **TwoStreamFusion** method: improved performance withou extra parameters (in comparison with C3D)

  - TSN
    - Temporal segments networks: towards good practices for deep action recognition
      - long range temporal modeling effective solution
      - usage of batch normalisation, dropout, and pre-training
      - sparse clip sampling across the video: to better model long range temporal signal
      - final prediction at video-level: 
        - combine scores from temporal and spatial streams
        - Fusing score of final spatial and temporal scores using weighted average and applying softmax over all classes
      - main challenges addressed: overfitting due to small sizes; and long range modeling

  - ActionVlad
    - ActionVlad: learning spatio-temporal aggregation for action classification
      - learnable video-level aggregation of features
      - end-to-end trainable model to capture long term dependency
      - main contribution: vlad (video-level aggregation ...?) rather than normal aggregation (maxpool or avgpool, for instance)
      - VLAD: effective way of pooling ! sota in early 2017

  - HiddenTwoStream
    - Hidden two-stream convolutional networks for action recognition
      - generation of optical flow input on-the-fly (by using a separate network)
      - the paper advocates the usage of an unsupervised architecture to generate optical flow for a stack of frames.
      - optical flow: image reconstruction problem
      - temporal stream:
        - it has an optical flow generation net on the top
        - the input is subsequent frames rather than preprocessed optical flow
      
  - I3D
    - Quo vadis, action recognition? A new model and the Kinetics database
      - 3d models combined into two stream achitecture 
      - two different 3d networks for both streams (spatial and temporal)
      - 2d pre trained weights are repeated in 3d !!!
      - major contribution: benefit of using pre trained 2D conv nets

  - T3D
    - Temporal 3d convnets: new architecture and transfer learning for video classification
      - combine temporal information across variable depth
      - new technique to supervise transfer learning between 2d pre-trained net to 3d net
      - multi-depth temporal pooling layer
Todas as abordagens acima são variações/combinações das redes básicas a seguir:
  - LSTM: long shot term memory
  - 3D-ConvNet
  - Two-stream
  - 3D-Fused Two-Stream
  - Two-Stream 3D-ConvNet



%\end{abstract}
%===

\section{Resumo do projeto}%Breve descrição do problema}
\label{sec:descr}
% verificar os desafios do último artigo lido para incluir aqui (se necessário)
% uma aplicação:
% - segurança em ambientes industriais (é o que dá maiores possibilidades para financiamento):
%   - deteção de pessoas;
%   - contagem de pessoas;
%   - presença de EPI;
%   - identificação de ações (quais?)
%
% Desafios: 
% - Variação antropométricas
% - Variação de ângulos de visão para diferentes câmeras no ambiente monitorado
% - Diferentes condições de iluminaçao
% - Baixa qualidade dos vídeos ? qualidade vs taxas de reconhecimento
% - Oclusão
% - Variação de iluminação e presença de sombras
% - Dados insuficientes
% - Variações climáticas
%
% Abordagem:
% -
% 
Apesar das políticas públicas indutoras e do esforço das empresas, a ocorrência de acidentes industriais ainda é elevada e responde por grandes prejuízos. Dentre os quais, a perda de vidas, danos psicológicos em colaboradores e familiares, degradação ambiental, e despesas com indenizações e custos previdenciários. Nesse sentido, o investimento em procedimentos e sistemas para salvaguardar a segurança do colaborador industrial é estratégico, não só por evitar os prejuízos mencionados, mas também pelos reflexos positivos na competitividade da empresa.
 
Visando contribuir para o melhoramento de sistemas de apoio às boas práticas de segurança e proteção em ambientes industriais, este projeto propõe o desenvolvimento e a implementação de um sistema para reconhecimento automático de atividades humanas (HAR, \emph{human activity recognition}), baseado em estratégias de visão computacional e redes neurais profundas. Duas abordagens principais se consolidaram para tratar o problema em questão: uso de redes \emph{multi-stream} e uso de redes convolucionais 3D. Ambas procuram combinar análise espacial (intra-quadro) e dependências temporais de eventos (inter-quadros) na caracterização das ações monitoradas. O projeto pretende ainda usar informação multimodal (especificamente sinal térmico e de áudio) para aprimorar a qualidade da classificação de atividades em relação ao classificador baseado apenas em sinal de vídeo. Dentre as funcionalidades previstas, destacam-se: detecção e contagem de pessoas nos ambientes monitorados; verificação de presença e utilização correta de equipamento de proteção individual e coletiva obrigatórios; alerta de comportamentos anômalos; e identificação de ações potencialmente perigosas para integridade de trabalhadores e da operação. É previsto também o desenvolvimento de aplicativos web e mobile para consulta de resultados e recebimento de alertas emitidos pelo sistema HAR.

A metodologia de execução do projeto adota abordagens de desenvolvimento ágil, notadamente o \emph{scrum}, e prevê o lançamento periódico de protótipos funcionais do sistema de reconhecimento, visando ao final do projeto dispor de uma solução de  maturidade tecnológica nível 7 (TRL 7, technology readiness level 7).

 O problema a ser resolvido é de grande interesse e relevância científica. As contribuições técnicas visadas concentram-se em três desafios atuais da área: o uso de dados de entrada multimodais para melhorar o desempenho do reconhecimento visual; a exploraração de mecanismos de interpretação dos sinais de entrada para identificar atributos que permitam reconhecer ações no curto prazo; e o desenvolvimento de abordagens para melhor uso de redes pré-treinadas e o treinamento com conjunto de dados parcialmente rotulado. Outros ganhos esperados são: formação de recursos humanos especializados em visão computacional e aprendizado de máquina; prevenção de acidentes industriais; e aumento de competitividade dos parceiros envolvidos.


\newpage
%===                                          

\section{Revisão da literatura científica}
\label{sec:rev}
\nocite{jegham-2020, hussain-2020,yao-2019,kongr-2018,herath-2017}
%#Elements:
%1- Short and precise **overview**; 
%  - Concentuação/Taxonomia: o que é o reconhecimento de ações (ações ou atividades?, interação, )
%  - Tipos de sistemas. Diferentes classificações: número de pessoas envolvidas, ambiente, interação com pessoas e/ou objetos, predição/reconhecimento, uso de hadcrafted features ou extração automática de features/atributos usando cnns, ou ainda abordagens híbridas
Este projeto trata do reconhecimento automático de atividades humanas mediante o uso de visão computacional baseada em redes profundas. Em razão da falta de consenso sobre os termos usados na área, é interessante que algumas definições sejam postas a fim de caracterizar o escopo deste projeto bem como o objeto de pesquisa. 

% O que é e algumas definições
A primeira distinção a ser feita é entre dois termos encontrados muito frequentemente: \emph{ação} e \emph{atividade}. Sem maiores aprofundamentos, e a despeito de algumas controvérsias, o presente texto adota a direção apontada por~\textcite{herath-2017} que considera que uma \emph{ação} é um evento primitivo, simples, fácil de caracterizar, e que é realizado por uma única pessoa (por exemplo, \emph{indivíduo levanta o braço}). Por outro lado, a sucessão de ações diferentes forma \emph{atividades}, essas mais ou menos complexas (por exemplo, \emph{preparar café}).
%
A tarefa de reconhecimento de atividades é também referenciada na literatura técnica especializada como classificação de atividades e consiste em, dado um vídeo de entrada, atribuir-lhe um dentre os elementos de um conjunto de atividades previamente definidas. Outras tarefas relacionadas são: a detecção de atividades, onde o objetivo é precisar os quadros de início e término de uma atividade no vídeo de entrada (localização temporal); e a predição de atividades, cuja finalidade é estimar quais atividades ocorrerão em quadros futuros, baseando-se nas atividades presentes no vídeo de entrada~\parencite{yao-2019}.%pensar em AUC para diferentes valores de limiar temporal.

% A estratégia a ser utilizada e suas variações
O enfoque pretendido neste projeto compreende o uso de redes neurais artificiais profundas, que vêm consolidado sua presença em um grande número de aplicações em visão computacional, em razão de sua capacidade de selecionar automaticamente os atributos mais adequados dos sinais de entrada, dadas as características do conjunto de treinamento~\parencite{yao-2019}.

Além dos algoritmos de classificação em si, outros elementos compõem o sistema de reconhecimento de atividades e influenciam a qualidade do resultado a ser alcançado. Dentre eles, um que merece menção diz respeito à modalidade de aquisição do vídeo a ser analisado. As principais modalidades em relação às características espectrais do sinal de entrada são: vídeo convencional, colorido ou não, em que o registro compreende a faixa de frequências do espectro visível; vídeo com informação de profundidade, cada vez mais populares em razão da redução de curso de sensores como o Kinect; e vídeo obtido por sensores infravermelhos. Outra modalidade de grande importância, refere-se à quantidade de câmeras usadas para aquisição da cena, podendo os sistemas ser classificados como: de câmera única, em geral posicionada estaticamente em relação à cena; ou multi-câmera, em que há mais de uma câmera registrando a cena que será analisada (pontos de vista diferentes do mesmo evento)~\parencite{yao-2019}.
%
% aspecto relacionado à cena
Outro aspecto a ser considerado, diz respeito à quantidade de elementos participantes de uma cena. Pode-se ter situações em que apenas uma pessoa está presente; em outras, pode-se ter dois ou mais indivíduos interagindo; e ainda, o(s) indivíduo(s) pode(m) interagir com objetos. As possibilidades de variação aqui são inúmeras, influenciando no grau de complexidade da cena.
% aspecto relacionado à cena
Cenas não são apenas compostas pelos atores principais que desempenham interações e/ou atividades de interesse, mas também pelo pelo contexto, simplificadamente denominado de \emph{fundo}, que pode variar entre \emph{estático} em ambientes controlados (em um estúdio ou laboratório, por exemplo) ou altamente dinâmicos, como em ruas de grandes cidades ou áreas com a presença de diversas pessoas realizando diferentes atividades. A dificuldade do reconhecimento aumenta com a dinamicidade do fundo, pois além da tarefa de estimar a atividade sendo desempenhada, o sistema precisa diferenciar entre movimentação dos objetos de interesse e das modificações típicas do fundo.
% aplicações
O reconhecimento automático de atividades é uma das áreas de maior interesse em visão computacional com um leque de possíveis aplicações abrangendo áreas como ambientes inteligentes, classificação e busca automática em repositórios de imagem e vídeo, e avaliação do comportamento de multidões, dentre outras.
%
%2- **Present state** of the research; 
% Abordagens:
%% Redes multi-streams
%% Redes 3D
%% Soluções híbridas.
Atualmente, a tendência e o estado da arte nessa área compreendem técnicas baseadas em redes neurais profundas. \textcite{yao-2019} apontam como principal desafio a representação da dependência temporal entre ações na caracterização de uma atividade complexa. Tal desafio, em grande parte, tem orientado os esforços da comunidade científica nos avanços da área. 
%
Grosso modo, são duas as principais abordagens mais bem sucedidas na busca de soluções para a representação temporal. A primeira delas, denominada multi-fluxo (multi-stream), procura caracterizar informações temporais e espaciais em dois ramos separados na rede neural profunda. O ramo temporal geralmente tem como entrada o fluxo óptico (ou estimativas deste) de cada quadro de vídeo, ao passo que o ramo espacial recebe o quadro propriamente dito. Cada uma das entradas passa por etapas de extração automática de atributos nas camadas convolucionais da rede, cujos resultados são processados e fundidos oportunamente para fins da classificação final. O uso de fluxo óptico, contudo, tem suas desvantagens uma vez que seu cálculo requer quadros adjacentes ao quadro de interesse, resultando em aumento de complexidade computacional e, consequentemente, dificuldade de implementação em tempo real. Como abordagens que prescindem do fluxo óptico, tem-se o uso de vetores de movimento em vídeo codificado~\parencite{yao-2019} e o uso de redes profundas 3D que poderiam reunir em um único fluxo informações temporais e espaciais simultaneamente. 
%
Ainda motivado pelas inconveniências do uso do fluxo óptico e pela necessidade de caracterização da informação temporal de longo e curto prazo no entendimento de atividades complexas, técnicas baseadas em redes profundas recursivas, tais como a LSTM (\emph{long short-term memory})~\parencite{hochreiter-1997} e outras, têm sido propostas, apresentando compromisso competitivo entre complexidade computacional e qualidade de resultados~\parencite{donahue-2016, herath-2017, xia-2020}.

%3- **Immediate connection** with your own research; 
Particularizando para o contexto da aplicação visada neste projeto (reconhecimento de atividades humanas em áreas industriais), pode-se dizer que o ambiente típico a ser a ser monitorado apresenta alta complexidade. Avançando um pouco mais na caracterização desse tipo de ambiente, é importante considerar que outros fatores influenciarão no desempenho do reconhecimento, tais como: variação de luminosidade, de acordo com os horários de monitoramento; presença de mais de um indivíduo interagindo entre si e com objetos, de maneiras difíceis de prever na etapa de treinamento das redes; variação de aparência de uma determinada atividade, em função da distância e do ângulo entre objeto de interesse e câmeras; e ocorrência de oclusões (encobrimento de uma região de interesse por distratores).
%
Apesar bons resultados documentados em REFs, onde diversos dos elementos mencionados, típicos de uma situação realística, são considerados, observa-se acentuada degradação no desempenho dos sistemas de reconhecimento a medida em que o controle das variáveis ambientais é perdido (a expressão em inglês seria \emph{in the wild}). Ou seja, sistemas de reconhecimento automático de atividades humanas em contextos práticos constituem um desafio atual, de grande interesse, e estratégico em termos científicos e de inovação.

%4- Clear and logical **discussion of the theoretical framework**; 
Em grande medida, as limitações de desempenho em situações realísticas são devidas às grandes variações entre cenas apresentadas na fase de operação e cenas usadas na fase de treinamento. Tipicamente, redes profundas exigem  conjunto de dados de treinamento enorme a fim de apresentar boa capacidade de generalização, mesmo em condições controladas. Considerando sistemas de aprendizagem supervisionada (o mais comum), tal conjunto de treinamento requer dados previamente rotulados, exigindo considerável esforço para sua geração e manutenção. Nesse sentido, as tendências alternativas ao treinamento supervisionado completo de sistema de reconhecimento (\emph{from scratch}) que vêm se consolidado na área~\parencite{herath-2017} são as seguintes: transferência de conhecimento, usando redes de genéricas pré-treinadas, mediante treinamento supervisionado expondo tais redes a exemplos rotulados típicos da aplicação pretendida, procedimento conhecido pelo termo \emph{fine-tuning}; estratégias para treinamento de conjunto de dados parcialmente rotulados, quando se dispõe de grande quantidade de exemplos no conjunto de dados, mas a atribuição de rótulos possui custo elevado; e técnicas para ampliação artificial de conjunto de dados (data augmentation)~\parencite{wang-2015}.

% Falar sobre datasets
O desempenho dos sistemas de reconhecimento de atividades, como qualquer outra aplicação que esteja baseada em redes neurais artificiais, sobre forte influência não apenas da arquitetura da rede, da estratégia de treinamento/validação/teste, mas também do conjunto de dados utizado. Dessa forma, a avaliação comparativa entre diferentes soluções, passa necessariamente pela padronização dos procedimentos de treinamento e teste, bem como pela padronização dos conjuntos de dados.  Ao longo do desenvolvimento da área, acompanhando as mudanças na tecnologia empregada para implementar os sistemas de reconhecimento, foram propostos diferentes conjuntos de dados, variando em número de vídeos, em interação entre elementos, e em complexidade da cena~\parencite{jegham-2020, kongr-2018}. 

%5- **Methodological implications**;
%6- Indicate an **open problem**; 
À medida em que a área avança, outras perspectivas de uso começaram a surgir. Dentre elas, está a que pode ser considerada o próximo passo em visão computacional nessa área: a predição de atividades humanas. Em linhas gerais, pode-se considerar que a distinção entre reconhecimento e predição está na localização temporal pretendida da estimativa de atividade fornecida pelo sistema~\parencite{kongr-2018}. Para um mesmo conjunto de entrada, o reconhecimento visa identificar a ação ocorrida ou caracterizada nesse conjunto de entrada. Por sua vez, a predição fornecerá uma estimativa da atividade que ocorrerá nos quadros de vídeo seguintes ao conjunto de entrada. O reconhecimento olha para o passado, enquanto que a predição considera o passado para projetar uma aposta sobre o que virá em seguida. Claramente, o problema da predição é bem mais desafiador, mas possui um apelo bastante evidente em direção autônoma e ambientes inteligentes voltados para segurança e prevenção de acidentes, onde melhor que remediar é previnir. No aspecto técnico e de implementação dos sistemas, predição e reconhecimento são semelhantes. Contudo, em relação aos resultados, a qualidade da predição é acentuadamente menor que em reconhecimento, aumentando ainda mais com o distanciamento temporal entre conjunto de entrada e predição. A representação temporal e da sucessão de ações/atividades assume importância ainda maior na tarefa de predição.
%7- The **contribution**  to the area
% uma outra contribuição pode ser a criação de um dataset que ficará disponível? Segredo industrical em risco !!
Alinhada com o estado da arte, problemas em aberto e tendências em reconhecimento automático de atividades humanas, a presente proposta de pesquisa estabelece como tópicos prioritários de investigação: o uso de dados de entrada  multimodais, ou seja outros tipos de entrada além do vídeo, para melhorar o desempenho do sistema de reconhecimento em ambientes de alta complexidade; a exploração de mecanismos de interpretação do sinais de entrada para identificar atributos que permitam reconhecer ações de curto prazo, fator chave para melhorar a qualidade de sistemas de predição visando segurança e proteção em ambientes industriais; o aperfeiçoamento de abordagens para transferência efetiva de conhecimento, mediante o uso de redes pré-treinadas; e o desenvolvimento de esquema de treinamento com conjunto de dados parcialmente rotulados.

\newpage
%===                                          

%\section{Trabalho preparatório}
%\label{sec:prep}
%\input{preparatorio}
%\newpage
%===                                          

\section{Objetivos}% do projeto de pesquisa}
\label{sec:obj}
% objetivos.tex
% 
% Aims that you want to achieve:
%   Academical,
%   Social,
%   Political, etc.
% It needs to show the research is important;
% Show the theoretical and practical significance of the project
%

\subsection{Objetivo geral}
O objetivo geral desta proposta de pesquisa consiste no desenvolvimento de novas técnicas em reconhecimento de atividades humanas aplicado ao monitoramento de áreas industriais, usando aprendizagem profunda, com ênfase na solução de problemas locais e na implementação de provas de conceito que demonstrem a viabilidade dos sistemas concebidos.

\subsection{Objetivos específicos} %valeria a pena transformar cada item em uma subsubseção ?? ou simplesmente colocar ":" e desenvolver o tópico?
\begin{enumerate}
    \item Aumento de relevância da produção científica local por meio de publicações em períodicos qualidade reconhecida;
    %formação de recursos humanos
    \item Formação de recursos humanos especializados em projeto, pesquisa e implementação de sistemas de visão computacional e aprendizado de máquina, com foco em Indústria 4.0 e transformação digital; 
    % arranjo produtivo local
    \item Estabelecimento de rede de parcerias envolvendo a academia, a indústria e o setor produtor de \emph{software}, para que as soluções concebidas atinjam maturidade tecnológica nível 7 (TRL, Technology Readness Level), ou seja, demonstração de protótipo em ambiente operacional; %no projeto não permaneçam apenas em âmbito acadêmico, mas possam ser comercializadas e implantadas; 
    % econômicos
    \item Contribuição para o aumento de competitividade da indústria regional, em razão do aumento de capital social, da redução de custos decorrente do uso de tecnologia desenvolvida pelo arranjo local, e da diminuição de acidentes na operação;  
    \item Manutenção de cultura em inovação, tendo como eixo diretor a busca de soluções a demandas do setor industrial com potencial para resultar na concessão de patentes ou registro de \emph{softwares};
    % pesquisa
    \item Maior visibilidade aos parceiros do projeto em razão da divulgação por meio de publicações e participação em eventos em que o projeto se possa representar; 
    % segurança e proteção
    \item Aumento de segurança e redução de acidentes nos ambientes industriais monitorados;
    % qualidade de vida
    \item Contribuição à melhoria da qualidade de vida do trabalhador industrial;   
    % ambiente
    \item Reforço às medidas de proteção ambiental já implantadas no setor industrial;
    \item Subsídio à prevenção de acidentes de efeito contaminante para o ambiente;
    % social
    \item Apoio aos mecanismos de manutenção da saúde e modo de vida das populações tradicionais em torno de instalações industriais.
\end{enumerate}

\newpage
%===

\section{Elementos do projeto}
\label{sec:proj}
%procedimentos experimentais
%the sources and quality of evidences consulted
%data gathered 
%controls
%statistical methods

% justificativa: relevância do problema
% metodologia: como 
% risco tecnológico
% grau de inovação
% grau de maturidade tecnológica atual, pretendido e os meios para chegar lá
% resumo da equipe executora
% apresentação do laboratório
% resumo do orçamento
% impactos: tecnológico, econômico, ambiental, social,
% orçamento/financeiro (merece uma seção em uma seção em separado)


\newpage

\section{Atividades, metas e cronograma}
\label{sec:cronos}
% cronos.tex
\subsection{Atividades e metas}



\newpage

\section{Orçamento}
\label{sec:orc}
%orcamento.tex
\subsection{Recursos humanos}
A equipe de execução foi concebida como sendo composta por dois alunos de graduação (IC), dois mestrandos (MT), um doutorando (DR), dois pesquisadores seniors (PQ) e um coordenador. Tal configuração foi elaborada no sentido de obter um bom compromisso entre custo e maturidade técnica suficiente para executar o projeto no prazo estipulado. Os valores de bolsa para IC, MT e DR têm como referência os valores pagos pelo CNPq. A Tabela~\ref{tab:rhmes} mostra as despesas mensais totais e por tipo de bolsa nos primeiro e segundo anos de projeto.
A Tabela~\ref{tab:rhano} apresenta os valores anuais totais de desembolso com recursos humanos do projeto.
Por sua vez, a Tabela~\ref{tab:rhtipo} especifica o investimento total em recursos humanos por tipo.

\begin{table}[h!]
\scriptsize
	\caption{Recursos humanos -- desembolso mensal por ano}
\rowcolors{2}{lightgray}{}%lightgray
\begin{tabular}{ lrrrrrrrrrrrr}
\toprule
%\hline\hline
   \rowcolor{lgray}
   Tipo      & 01      & 02      & 03      & 04      & 05      & 06      & 07      & 08      & 09      & 10      & 11      & 12      \\
\midrule
   IC-01     &   400,00&   400,00&   400,00&   400,00&   400,00&   400,00&   400,00&   400,00&   400,00&   400,00&   400,00&   400,00\\
   IC-02     &   400,00&   400,00&   400,00&   400,00&   400,00&   400,00&   400,00&   400,00&   400,00&   400,00&   400,00&   400,00\\
   MT-01     & 1.500,00& 1.500,00& 1.500,00& 1.500,00& 1.500,00& 1.500,00& 1.500,00& 1.500,00& 1.500,00& 1.500,00& 1.500,00& 1.500,00\\
   MT-02     & 1.500,00& 1.500,00& 1.500,00& 1.500,00& 1.500,00& 1.500,00& 1.500,00& 1.500,00& 1.500,00& 1.500,00& 1.500,00& 1.500,00\\
   DR        & 2.200,00& 2.200,00& 2.200,00& 2.200,00& 2.200,00& 2.200,00& 2.200,00& 2.200,00& 2.200,00& 2.200,00& 2.200,00& 2.200,00\\
   PQ-01     & 1.500,00& 1.500,00& 1.500,00& 1.500,00& 1.500,00& 1.500,00& 1.500,00& 1.500,00& 1.500,00& 1.500,00& 1.500,00& 1.500,00\\
   PQ-02     & 1.500,00& 1.500,00& 1.500,00& 1.500,00& 1.500,00& 1.500,00& 1.500,00& 1.500,00& 1.500,00& 1.500,00& 1.500,00& 1.500,00\\
   CO        & 2.000,00& 2.000,00& 2.000,00& 2.000,00& 2.000,00& 2.000,00& 2.000,00& 2.000,00& 2.000,00& 2.000,00& 2.000,00& 2.000,00\\
	\midrule
   Total     & 5.200,00& 5.200,00& 5.200,00& 5.200,00& 5.200,00& 5.200,00& 5.200,00& 5.200,00& 5.200,00& 5.200,00& 5.200,00& 5.200,00\\
   
%   \rowcolor{lightblue}%lgray}
\midrule
\midrule
   \rowcolor{lgray}
   Tipo      & 13      & 14      & 15      & 16      & 17      & 18      & 19      & 20      & 21      & 22      & 23      & 24      \\
\midrule
   IC-01     &   400,00&   400,00&   400,00&   400,00&   400,00&   400,00&   400,00&   400,00&   400,00&   400,00&   400,00&   400,00\\
   IC-02     &   400,00&   400,00&   400,00&   400,00&   400,00&   400,00&   400,00&   400,00&   400,00&   400,00&   400,00&   400,00\\
   MT-01     & 1.500,00& 1.500,00& 1.500,00& 1.500,00& 1.500,00& 1.500,00& 1.500,00& 1.500,00& 1.500,00& 1.500,00& 1.500,00& 1.500,00\\
   MT-02     & 1.500,00& 1.500,00& 1.500,00& 1.500,00& 1.500,00& 1.500,00& 1.500,00& 1.500,00& 1.500,00& 1.500,00& 1.500,00& 1.500,00\\
   DR        & 2.200,00& 2.200,00& 2.200,00& 2.200,00& 2.200,00& 2.200,00& 2.200,00& 2.200,00& 2.200,00& 2.200,00& 2.200,00& 2.200,00\\
   PQ-01     & 1.500,00& 1.500,00& 1.500,00& 1.500,00& 1.500,00& 1.500,00& 1.500,00& 1.500,00& 1.500,00& 1.500,00& 1.500,00& 1.500,00\\
   PQ-02     & 1.500,00& 1.500,00& 1.500,00& 1.500,00& 1.500,00& 1.500,00& 1.500,00& 1.500,00& 1.500,00& 1.500,00& 1.500,00& 1.500,00\\
   CO        & 2.000,00& 2.000,00& 2.000,00& 2.000,00& 2.000,00& 2.000,00& 2.000,00& 2.000,00& 2.000,00& 2.000,00& 2.000,00& 2.000,00\\
	\midrule
   Total     & 5.200,00& 5.200,00& 5.200,00& 5.200,00& 5.200,00& 5.200,00& 5.200,00& 5.200,00& 5.200,00& 5.200,00& 5.200,00& 5.200,00\\
\bottomrule
%\hline\hline   
\end{tabular}
	\label{tab:rhmes}
\end{table}
%====================

\begin{table}[!h]
\centering
%\scriptsize
	\caption{Recursos humanos por ano do projeto}
%\rowcolors{2}{lightgray}{}%lightgray
\begin{tabular}{ lr}
\toprule
%\hline\hline
%   \rowcolor{lgray}
   Ano       & Total (R\$)  \\
\midrule
   Ano 01    &   132.000,00 \\
   Ano 02    &   132.000,00 \\
\midrule
 %  \rowcolor{lgray}
   Total     &   264.000,00 \\
\bottomrule
\end{tabular}
	\label{tab:rhano}
\end{table}
%=======================

\begin{table}[!h]
\centering
%\scriptsize
	\caption{Recursos humanos por tipo}
%\rowcolors{2}{lightgray}{}%lightgray
\begin{tabular}{ lr}
\toprule
%\hline\hline
%   \rowcolor{lgray}
   Tipo  & Total (R\$) \\
\midrule
   IC    &   19.200,00 \\
   MT    &   72.000,00 \\
   DR    &   52.800,00 \\
   PQ    &   72.000,00 \\
   CO    &   48.000,00 \\
\midrule
 %  \rowcolor{lgray}
   Total     &   264.000,00 \\
\bottomrule
\end{tabular}
	\label{tab:rhtipo}
\end{table}
 
\newpage
\subsection{Serviços de terceiros}
A título de justificativa dos itens relacionados na Tabela~\ref{tab:ter}: 
\begin{itemize}
	\item A \emph{instalação de sistema de câmeras} é normalmente oferecida ou indicada pela empresa que vende esse tipo de sistema e é recomendável fazer uso de tal serviço profissional, tanto para não se perder a garantia dos equipamentos, quanto para se poder utilizar o mais rapidamente possível o sistema de aquisição de vídeo. Esse último aspecto é de grande importância para o atendimento dos prazos do projeto;
	\item O item \emph{computação em nuvem} tem por objetivo resgardar os prazos de entrega nas diferentes fases da execução contra atrasos na aquisição de computadores de alto desempenho, depreciação cambial (o que impediria a aquisição de computadores com a qualidade/preço inicialmente cotados), e eventuais panes nos sistemas de processamento;
	\item As \emph{taxas open access} visam prover recurso para o pagamento referente à publicação de artigos técnicos relacionados à produção do projeto  em periódicos de acesso livre de prestígio internacional.
\end{itemize}
\begin{table}[!h]
\centering
%\scriptsize
	\caption{Serviços de terceiros por mês}
%\rowcolors{2}{lightgray}{}%lightgray
\begin{tabular}{clr}
\toprule
%\hline\hline
%   \rowcolor{lgray}
	Mês   & Descrição & Custo (R\$) \\
	\midrule
	04    & Instalação de sistema de câmeras   & 2.000,00 \\
	07    & Computação em nuvem                & 3.000,00 \\
	08    & Computação em nuvem                & 3.000,00 \\
	09    & Computação em nuvem                & 3.000,00 \\
	10    & Computação em nuvem                & 3.000,00 \\
	11    & Computação em nuvem                & 3.000,00 \\
	12    & Computação em nuvem                & 3.000,00 \\
	13    & Computação em nuvem                & 3.000,00 \\
	14    & Computação em nuvem                & 3.000,00 \\
	15    & Computação em nuvem                & 3.000,00 \\
	16    & Computação em nuvem                & 3.000,00 \\
	17    & Computação em nuvem                & 3.000,00 \\
	17    & Computação em nuvem                & 3.000,00 \\
	24    & Taxas Open Access                  &10.500,00 \\
\midrule
 %  \rowcolor{lgray}
	      & Total                              &48.500,00 \\
\bottomrule
\end{tabular}
	\label{tab:ter}
\end{table}
\newpage

\subsection{Custeio}
Os itens de custeio apresentados na Tabela~\ref{tab:custeio} referem-se principalmente a material de escritório a ser utilizado necessariamente em atividades relacionadas à execução do projeto de pesquisa.
\begin{table}[!h]
\centering
%\scriptsize
	\caption{Custeio por mês}
%\rowcolors{2}{lightgray}{}%lightgray
\begin{tabular}{clrcr}
\toprule
%\hline\hline
%   \rowcolor{lgray}
	Mês   & Descrição              & Custo (R\$) & Quantidade & Total \\
	\midrule
	01    & Toner para impressora  & 400,00      & 01         & 400,00 \\
	01    & Resma de papel A4      &  25,00      & 10         & 250,00 \\
	13    & Toner para impressora  & 400,00      & 01         & 400,00 \\
	13    & Resma de papel A4      &  25,00      & 10         & 250,00 \\
\midrule
 %  \rowcolor{lgray}
	      &                        &             & Total      &1.300,00 \\
\bottomrule
\end{tabular}
	\label{tab:custeio}
\end{table}
 
\subsection{Capital}
Os itens de capital relacionados na Tabela~\ref{tab:capital} consistem em equipamentos para aquisição (câmera de vigilância, gravador digital, e hd) e processamento (desktop, workstation, e notebook) de vídeo digital. Os desktops e workstation possuem configuração para suportar as fases de treinamento e produção de sistemas baseados em aprendizado de máquina que lidem com grande volume de dados, o que é típico em aplicaçãoes de reconhecimento de atividades por vídeo e redes neurais convolucionais. Os notebooks são de alto desempenho e devem apoiar a realização de testes em campo do sistema de reconhecimento de atividades.
\begin{table}[!h]
\centering
\scriptsize
	\caption{Capital por mês}
%\rowcolors{2}{lightgray}{}%lightgray
\begin{tabular}{clp{0.35\textwidth}rcr}
\toprule
%\hline\hline
%   \rowcolor{lgray}
	Mês & Equipamento & Descrição               & Custo (R\$) & Quantidade & Total    \\
	\midrule
	01  & Câmera de vigilância  & Speed dome; Compactação de vídeo: H.264 (MPEG-4 Parte 10/AVC) Motion JPEG; Resoluções: HDTV 720p 1280x720 a 320x180; Taxa de quadros: H.264: Até 30/25 fps (60/50 Hz) em todas as resoluções Motion JPEG: Até 30/25 fps (60/50 Hz) em todas as resoluções & 6.000,00    & 02         &  12.000,00 \\
	01  & Gravador digital      & Gravador digital para sistema de câmeras de vigilância, 4 canais                     & 3.000,00    & 01         &   3.000,00 \\
	01  & HD                    & HD para gravador digital 8 TB                                                        & 2.500,00    & 01         &   2.500,00 \\
	01  & Desktop               & iAMD Ryzen 7 3700X (8 Núcleos e 16 Threads, 3.6GHz, Turbo até 4.4GHz, Cache de 32MB); Nvidia GeforceTM GTX 1650 Super 4GB 1280 cuda cores; RAM 16GB DDR4; SSD 1TB Workstation; Placa de rede Wireless Dual 802.11 AC;bMouse e teclado com fio; Monitor: 23.8" (1920x1080) (HDMI, DP); Frete: Transportadora com seguro - Grátis (CIF)                                                                                          & 10.757,34   & 04         &  43.029,36 \\
	13  & Workstation           & AMD Ryzen Threadripper 3970X (32 Núcleos e 64 Threads, 3.7GHz, Turbo até 4.5GHz, Cache de 144MB); Nvidia GeforceTM RTX 2080 TI 11GB 3584 cuda cores; 64GB DDR4 3200MHz (4x16GB); SSD M.2 PCIe X4 NVMe 1TB Workstation Class; HDD 2 TB 7200RPM 128MB SATA III Enterprise Class; Refrigeração: Refrigeração líquida dupla; Rede: Integrada 10/100/1000;Frete: Transportadora com seguro - Grátis (CIF)                           & 48.070,55   & 01         &  48.070,55 \\
	13  & Notebook              & Intel® Core™ i7 1065G7 1,3 GHz, 8 MB Cache, 16 GB (8 GB Onboard + 8 GB Offboard), 512 GB SSD PCIe NVME M2, Tela > 15'', Placa de Vídeo GeForce MX330 com 2GB de memória GDDR5; wifi 802.11ac, câmera frontal, Bluetooth, saídas: 1x HDMI 1.4 1x 3.5mm Combo Audio Jack 2x USB 2.0 Type-A 1x USB 3.2 Gen 1 Type-A 1x USB 3.2 Gen 1 Type-C,                                                                                      & 8.000,00    & 02         &  16.000,00 \\
\midrule
 %  \rowcolor{lgray}
             &                      &                                                                                      &             & Total      & 124.599,91 \\
\bottomrule
\end{tabular}
	\label{tab:capital}
\end{table}
\newpage

\subsection{Passagens e diárias}
Na Tabela~\ref{tab:viagem} estão especificados recursos para viabilizar a participação de pesquisadores em eventos técnicos de abrangência nacional e internacional. A finalidade de tais participações seria a atualização técnica da equipe e a apresentação de trabalhos produzidos no âmbito do projeto e aprovados por comitê de revisores. Face ao tempo para execução do projeto, estima-se a possibilidade a participação em dois eventos nacionais e um evento internacional. Para os eventos nacionais, é prevista a participação de dois membros do projeto, com cinco diárias de manutenção cada. Para o evento internacional, prevê-se a participação de um pesquisador da equipe e sete diárias. Para os valores de diárias nacionais e internacionais, utilizou-se como referência os valores correspondentes pagos pelo CNPq.
\begin{table}[!h]
\centering
%\scriptsize
	\caption{Passagens e diárias por mês}
%\rowcolors{2}{lightgray}{}%lightgray
\begin{tabular}{lcccrcr}
\toprule
%\hline\hline
%   \rowcolor{lgray}
	Tipo                         & Mês & Localidade & Finalidade               & Custo (R\$) & Quantidade & Total    \\
	\midrule
	Passagem aérea nacional      & 12  & a definir  & Apresentação de trabalho & 3.300,00    & 02         &  6.600,00 \\
	Diárias (nacional)           & 12  & a definir  & Manutenção em viagem     &   320,00    & 10         &  3.200,00 \\
	Passagem aérea nacional      & 24  & a definir  & Apresentação de trabalho & 3.300,00    & 02         &  6.600,00 \\
	Diárias (nacional)           & 24  & a definir  & Manutenção em viagem     &   320,00    & 10         &  3.200,00 \\
	Passagem aérea internacional & 24  & a definir  & Apresentação de trabalho & 7.000,00    & 01         &  7.000,00 \\
	Diárias (internacional)      & 24  & a definir  & Manutenção em viagem     & 2.220,00    & 07         & 15.540,00 \\
\midrule
 %  \rowcolor{lgray}
	                             &     &            &                          &             & Total      & 42.140,00 \\
\bottomrule
\end{tabular}
	\label{tab:viagem}
\end{table}
 
\subsection{Geral}
Nesta seção, apresentamos um resumo do orçamento. Na Tabela~\ref{tab:geral-mes-tipo} encontra-se o desembolso mensal para cada tipo de item de orçamento.
A Tabela~\ref{tab:geral-ano-rubrica}, por sua vez, exibe o investimento anual por rubrica. Nesse caso, a coluna \emph{Custeio} reúne \emph{Terceiros}, \emph{Consumo}, e \emph{Passagens/Diárias}.
\begin{table}[!h]
\scriptsize
	\caption{Desembolso mensal por tipo}
\rowcolors{2}{lightgray}{}%lightgray
\centering
\begin{tabular}{lcrrrrrr}
\toprule
%\hline\hline
   \rowcolor{lgray}
	Ano  & Mês      & Bolsas  & Terceiros  & Consumo & Pass/Diárias & Capital    & Valor \\
\midrule
       Ano 1 & 01       &  11.000,00 &      0,00 & 650,00  &     0,00     & 60.529,36 &  72.179,36\\
             & 02       &  11.000,00 &      0,00 &   0,00  &     0,00     &      0,00 &   11.000,00\\
             & 03       &  11.000,00 &      0,00 &   0,00  &     0,00     &      0,00 &   11.000,00\\
             & 04       &  11.000,00 &  2.000,00 &   0,00  &     0,00     &      0,00 &   13.000,00\\
             & 05       &  11.000,00 &      0,00 &   0,00  &     0,00     &      0,00 &   11.000,00\\
             & 06       &  11.000,00 &      0,00 &   0,00  &     0,00     &      0,00 &   11.000,00\\
             & 07       &  11.000,00 &  3.000,00 &   0,00  &     0,00     &      0,00 &   14.000,00\\
             & 08       &  11.000,00 &  3.000,00 &   0,00  &     0,00     &      0,00 &   14.000,00\\
             & 09       &  11.000,00 &  3.000,00 &   0,00  &     0,00     &      0,00 &   14.000,00\\
             & 10       &  11.000,00 &  3.000,00 &   0,00  &     0,00     &      0,00 &   14.000,00\\
             & 11       &  11.000,00 &  3.000,00 &   0,00  &     0,00     &      0,00 &   14.000,00\\
             & 12       &  11.000,00 &  3.000,00 &   0,00  & 9.800,00     &      0,00 &   23.800,00\\
\midrule
	     & Subtotal & 132.000,00 & 20.000,00 & 650,00  & 9.800,00     & 60.529,36 & 222.979,36\\
%   \rowcolor{lightblue}%lgray}
\midrule
\midrule
   \rowcolor{lgray}
	Ano  & Mês      & Bolsas  & Terceiros  & Consumo & Pass/Diárias & Capital    & Valor \\
\midrule
       Ano 2 & 13       &  11.000,00 &  3.000,00 & 650,00  &      0,00   & 64.070,55 &  78.720,55\\
             & 14       &  11.000,00 &  3.000,00 &   0,00  &      0,00   &      0,00 &  14.000,00\\
             & 15       &  11.000,00 &  3.000,00 &   0,00  &      0,00   &      0,00 &  14.000,00\\
             & 16       &  11.000,00 &  3.000,00 &   0,00  &      0,00   &      0,00 &  14.000,00\\
             & 17       &  11.000,00 &  3.000,00 &   0,00  &      0,00   &      0,00 &  14.000,00\\
             & 18       &  11.000,00 &  3.000,00 &   0,00  &      0,00   &      0,00 &  14.000,00\\
             & 19       &  11.000,00 &      0,00 &   0,00  &      0,00   &      0,00 &   11.000,00\\
             & 20       &  11.000,00 &  2.000,00 &   0,00  &      0,00   &      0,00 &   11.000,00\\
             & 21       &  11.000,00 &      0,00 &   0,00  &      0,00   &      0,00 &   11.000,00\\
             & 22       &  11.000,00 &      0,00 &   0,00  &      0,00   &      0,00 &   11.000,00\\
             & 23       &  11.000,00 &      0,00 &   0,00  &      0,00   &      0,00 &   11.000,00\\
             & 24       &  11.000,00 & 10.500,00 &   0,00  & 32.340,00   &      0,00 &  53.840,00\\
\midrule
	     & Subtotal & 132.000,00 & 28.500,00 & 650,00  & 32.340,00   & 64.070,55 & 257.560,55\\
%   \rowcolor{lgray}
\bottomrule
%\hline\hline   
\end{tabular}
	\label{tab:geral-mes-tipo}
\end{table}

\begin{table}[!h]
%\scriptsize
	\caption{Desembolso anual por rubrica}
%\rowcolors{2}{lightgray}{}%lightgray
\centering
\begin{tabular}{crrrr}
\toprule
%\hline\hline
   %\rowcolor{lgray}
	Ano   & RH         & Custeio   & Capital    & Total \\
\midrule
	Ano 1 & 132.000,00  & 30.450,00 & 60.529,36  & 222.979,36\\
	Ano 2 & 132.000,00  & 61.490,00 & 64.070,55  & 257.560,55\\
\midrule
	Total & 264.000,00 & 91.940,00 & 124.599,91 & 480.539,91\\
\bottomrule
%\hline\hline   
\end{tabular}
	\label{tab:geral-ano-rubrica}
\end{table}
\clearpage
%

\newpage

%\section{Referências}
%\label{sec:ref}
\printbibliography
%===

\end{document}
