% verificar os desafios do último artigo lido para incluir aqui (se necessário)
% uma aplicação:
% - segurança em ambientes industriais (é o que dá maiores possibilidades para financiamento):
%   - deteção de pessoas;
%   - contagem de pessoas;
%   - presença de EPI;
%   - identificação de ações (quais?)
%
% Desafios: 
% - Variação antropométricas
% - Variação de ângulos de visão para diferentes câmeras no ambiente monitorado
% - Diferentes condições de iluminaçao
% - Baixa qualidade dos vídeos ? qualidade vs taxas de reconhecimento
% - Oclusão
% - Variação de iluminação e presença de sombras
% - Dados insuficientes
% - Variações climáticas
%
% Abordagem:
% -
% 
Sistemas para reconhecimento automático de ações apresentam desempenho cada vez mais próximo da capacidade humana de interpretação visual, graças à popularização das técnicas de aprendizado de máquina, em especial de redes neurais profundas. 
%

O reconhecimento de atividades humanas visa identificar o tipo de ação executada por uma ou mais pessoas em uma cena registrada em vídeo não exibido anteriormente ao sistema.
%
A tarefa apresenta desafios técnicos relevantes, cujo grau de dificuldade depende de diversos elementos, tais como: número de pessoas envolvidas; complexidade da ação de interesse; características do ambiente em que ocorre a ação; grau de interação entre componentes (pessoas e objetos) da atividade; e qualidade do vídeo sob análise. 
%

Para fins de definição de escopo do projeto e atendimento a demandas de setores produtivos locais, optou-se pela aplicação em sistemas de apoio à proteção e segurança em ambientes industriais. Dentre as funcionalidades previstas, destacam-se: detecção e contagem de pessoas nos ambientes monitorados; verificação de presença e utilização correta de equipamento de proteção individual obrigatório; alerta de comportamentos anômalos; e identificação de ações potencialmente perigosas para integridade de trabalhadores e da operação.
%

A presente proposta de pesquisa concentra-se na investigação de novas técnicas em reconhecimento e predição de ações humanas, mediante o emprego de estratégias baseadas em visão computacional e aprendizagem profunda, bem como na implementação de provas de conceito que demonstrem a viabilidade dos sistemas concebidos. 
%
Em redes neurais profundas, duas abordagens principais se consolidaram para tratar o problema em questão: uso de redes \emph{multi-stream} e uso de redes convolucionais 3D. Ambas procuram combinar análise espacial (intra-quadro) e dependências temporais de eventos (inter-quadros) na caracterização das ações monitoradas. 
%

As contribuições científicas visadas pelo estudo proposto concentram-se em três desafios atuais da área: o uso de dados de entrada multi-modais para melhorar o desempenho do reconhecimento visual; a exploraração de mecanismos de interpretação dos sinais de entrada para identificar atributos que permitam reconhecer ações no curto prazo; e o desenvolvimento de abordagens para uso de redes pré-treinadas e o treinamento com conjunto de dados parcialmente rotulado.
%
%Apesar de seus resultados representarem o estado da arte, são conhecidos os problemas da aprendizagem profunda com impacto direto na capacidade de generalização das redes: a necessidade de grande poder computacional e de grande volume de dados rotulados na fase de treinamento (aprendizado supervisionado). Como alternativa, a fim de contornar tais restrições, pode-se empregar o uso de apredizado semi-supervisionado com base de treinamento parcialmente anotada, bem como proceder ao ajuste de parâmetros de redes previamente treinadas (transferência de conhecimento).
Outros ganhos esperados são: formação de recursos humanos especializados em projeto e implementação de sistemas baseados em aprendizado de máquina; prevenção de acidentes em espaços industriais que comprometam a saúde de colaboradores ou possuam efeito contaminante para o ambiente; e aumento de competitividade dos parceiros envolvidos pela inovação em produtos e processos advinda da pesquisa.

