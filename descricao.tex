% verificar os desafios do último artigo lido para incluir aqui (se necessário)
% uma aplicação:
% - segurança em ambientes industriais (é o que dá maiores possibilidades para financiamento):
%   - deteção de pessoas;
%   - contagem de pessoas;
%   - presença de EPI;
%   - identificação de ações (quais?)
%
% Desafios: 
% - Variação antropométricas
% - Variação de ângulos de visão para diferentes câmeras no ambiente monitorado
% - Diferentes condições de iluminaçao
% - Baixa qualidade dos vídeos ? qualidade vs taxas de reconhecimento
% - Oclusão
% - Variação de iluminação e presença de sombras
% - Dados insuficientes
% - Variações climáticas
%
% Abordagem:
% -
% 
Apesar das políticas públicas indutoras e do esforço das empresas, a ocorrência de acidentes industriais ainda é elevada e responde por grandes prejuízos. Dentre os quais, a perda de vidas, danos psicológicos em colaboradores e familiares, degradação ambiental, e despesas com indenizações e custos previdenciários. Nesse sentido, o investimento em procedimentos e sistemas para salvaguardar a segurança do colaborador industrial é estratégico, não só por evitar os prejuízos mencionados, mas também pelos reflexos positivos na competitividade da empresa.
 
Visando contribuir para o melhoramento de sistemas de apoio às boas práticas de segurança e proteção em ambientes industriais, este projeto propõe o desenvolvimento e a implementação de um sistema para reconhecimento automático de atividades humanas (HAR, \emph{human activity recognition}), baseado em estratégias de visão computacional e redes neurais profundas. Duas abordagens principais se consolidaram para tratar o problema em questão: uso de redes \emph{multi-stream} e uso de redes convolucionais 3D. Ambas procuram combinar análise espacial (intra-quadro) e dependências temporais de eventos (inter-quadros) na caracterização das ações monitoradas. O projeto pretende ainda usar informação multimodal (especificamente sinal térmico e de áudio) para aprimorar a qualidade da classificação de atividades em relação ao classificador baseado apenas em sinal de vídeo. Dentre as funcionalidades previstas, destacam-se: detecção e contagem de pessoas nos ambientes monitorados; verificação de presença e utilização correta de equipamento de proteção individual e coletiva obrigatórios; alerta de comportamentos anômalos; e identificação de ações potencialmente perigosas para integridade de trabalhadores e da operação. É previsto também o desenvolvimento de aplicativos web e mobile para consulta de resultados e recebimento de alertas emitidos pelo sistema HAR.

A metodologia de execução do projeto adota abordagens de desenvolvimento ágil, notadamente o \emph{scrum}, e prevê o lançamento periódico de protótipos funcionais do sistema de reconhecimento, visando ao final do projeto dispor de uma solução de  maturidade tecnológica nível 7 (TRL 7, technology readiness level 7).

 O problema a ser resolvido é de grande interesse e relevância científica. As contribuições técnicas visadas concentram-se em três desafios atuais da área: o uso de dados de entrada multimodais para melhorar o desempenho do reconhecimento visual; a exploraração de mecanismos de interpretação dos sinais de entrada para identificar atributos que permitam reconhecer ações no curto prazo; e o desenvolvimento de abordagens para melhor uso de redes pré-treinadas e o treinamento com conjunto de dados parcialmente rotulado. Outros ganhos esperados são: formação de recursos humanos especializados em visão computacional e aprendizado de máquina; prevenção de acidentes industriais; e aumento de competitividade dos parceiros envolvidos.

