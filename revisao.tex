%1- Short and precise **overview**; 
% Introdução
O reconhecimento de atividades humanas (\emph{human activity recognition}, HAR) é atualmente uma das áreas de maior interesse em visão computacional, com um amplo leque de aplicações abrangendo, por exemplo, monitoramento remoto de ambientes, acompanhamento de rotina diária, avaliação do comportamento de multidões, e classificação e busca automáticas em repositórios de imagem e vídeo~\parencite{hussain-2020,jegham-2020}.
%

% definições 1
Em razão de uma certa falta de consenso sobre os termos usados na área, é interessante que algumas definições seja postas, para bem caracterizar o escopo deste projeto e o objeto de pesquisa. 
Começaremos por dois termos encontrados amiúde e que são motivo de confusão: \emph{ação} e \emph{atividade}. Sem maiores aprofundamentos, e a despeito de algumas controvérsias, o presente texto adota a direção apontada por~\textcite{herath-2017} que considera que uma \emph{ação} é um evento primitivo, simples, e que é realizado por uma única pessoa (por exemplo, \emph{indivíduo levanta o braço}). Por sua vez, uma sucessão de ações diferentes forma \emph{atividades}, essas mais ou menos complexas (por exemplo, \emph{preparar café}).
%
% definições 2
Na literatura especializada, o reconhecimento de atividades é também referenciado como classificação de atividades e consiste na avaliação de um vídeo (sinal de entrada), associando-o a um dos itens de uma lista de atividades previamente definida. Outras tarefas relacionadas são: a detecção de atividades, onde o objetivo é precisar os tempos de início e término de uma atividade (localização temporal); e a predição de atividades, cuja finalidade é estimar qual atividade é a mais provável de  ocorrer, dado o que foi observado no vídeo de entrada~\parencite{yao-2019}. %pensar em AUC para diferentes valores de limiar temporal.
%
% sistema: modalidades
Além dos algoritmos de classificação em si, outros elementos influenciam o desempenho dos sistemas de HAR. Dentre os mais relevantes, tem-se a modalidade de aquisição de vídeo e suas sub-categorias. Por exemplo, de acordo com o tipo de sensor usado nas câmeras, o vídeo pode ser classificado como: vídeo convencional, onde o registro compreende a faixa de frequências do espectro visível; vídeo com informação de profundidade, cujo exemplo mais comum são os vídeos obtidos com o Microsoft Kinect; e vídeo obtido por sensores infravermelhos. Outra categoria refere-se à quantidade de câmeras usadas, podendo os sistemas ser classificados como: de câmera única; ou multi-câmera~\parencite{yao-2019}.
%
% cena: quantidade de elementos e suas interações
O conteúdo dos vídeos sob análise também interfere na qualidade dos resultados. Por exemplo, há variação de complexidade da tarefa de reconhecimento, dependendo da quantidade de elementos participantes (indivíduos e objetos) da atividade de interesse e seu grau de interação. %, que pode se dar entre . Pode-se ter situações em que apenas uma pessoa está presente; em outras, pode-se ter dois ou mais indivíduos interagindo; e ainda, o(s) indivíduo(s) pode(m) interagir com objetos.
% cena: background
O fundo (\emph{background}) diante do qual a atividade de interesse se desenrola é outro elemento significativo, podendo variar entre \emph{estático}, típico de ambientes controlados (estúdio de gravação ou laboratório de pesquisa, por exemplo), e altamente dinâmico, como em ruas de grandes cidades ou áreas abertas com a presença de diversas pessoas realizando diferentes atividades simultanemente. A dificuldade do reconhecimento tende a aumentar quanto maior a dinâmica do fundo, pois além da tarefa de estimar a atividade de interesse, o sistema precisa diferenciá-la da movimentação típica da cena~\parencite{jegham-2020}.
%

%2- **Present state** of the research and Open problems; 
Atualmente, a tendência e o estado da arte em HAR compreendem técnicas baseadas em redes neurais artificiais profundas. As razões são as mesmas que em outras áreas da visão computacional: os bons resultados decorrentes, em grande parte, da capacidade dessas redes em selecionar automaticamente os atributos mais adequados para a solução de um dado problema, considerando o conjunto de dados disponível. Contudo, um dos principais limitantes de desempenho, segundo \textcite{yao-2019}, é a representação da dependência temporal entre ações, necessária para a caracterização de uma atividade complexa. 
% abordagens: multi-stream, 3D, LSTM
Grosso modo, são duas as principais abordagens mais bem sucedidas em representação temporal. A primeira que trataremos é denominada \emph{multi-stream}, e procura caracterizar informações temporais e espaciais em dois ramos separados na rede neural profunda. O ramo temporal geralmente tem como entrada o fluxo óptico (ou estimativas deste) de cada quadro de vídeo, ao passo que o ramo espacial recebe o quadro propriamente dito. Cada uma das entradas passa por etapas de extração automática de atributos nas camadas convolucionais da rede, cujos resultados são processados e fundidos oportunamente para fins da classificação final. O uso de fluxo óptico, contudo, tem suas desvantagens uma vez que seu cálculo requer quadros adjacentes ao quadro de interesse, resultando em aumento de complexidade computacional e, consequentemente, dificuldade de implementação em tempo real. Em vídeo codificado, há propostas que prescindem do fluxo óptico substituindo-o pelo uso dos vetores de movimento~\parencite{yao-2019}. A segunda abordagem, denominada espaço-temporal, emprega redes profundas 3D, pretendendo reunir em um único fluxo informações temporais e espaciais simultaneamente. Ainda motivado pelas inconveniências do uso do fluxo óptico e pela necessidade de caracterização da informação temporal de longo e curto prazo no entendimento de atividades complexas, técnicas baseadas em redes profundas recursivas, tais como a LSTM (\emph{long short-term memory})~\parencite{hochreiter-1997}, têm sido propostas, apresentando compromisso competitivo entre complexidade computacional e qualidade de resultados~\parencite{donahue-2016, herath-2017, xia-2020}.
% situações realísticas e conjunto de dados 
Recentemente, tem havido um crescente interesse na aplicação de HAR em situações realísticas nas quais o controle das variáveis ambientais é baixo ou nenhum (a expressão correspondente em inglês seria \emph{in the wild}). Tal interesse é alimentado pela qualidade crescente dos sistemas de reconhecimento, resultante dos progressos nas áreas de redes profundas, e arquitetura de computadores. Como as limitações de acurácia em situações realísticas estão associadas às diferenças de características entre cenas apresentadas na fase de operação e cenas usadas no treinamento, diferentes conjuntos de dados disponíveis publicamente têm sido propostos abrangendo variações em número de vídeos disponíveis, grau de interação entre elementos, e complexidade das cenas registradas~\parencite{jegham-2020, kongr-2018}. A proposição de conjuntos de dados públicos atende também a uma outra necessidade: o estabelecimento de uma base comum para que sistemas de HAR possam ser comparados corretamente.
% esforço de criação e manutenção de dataset x estratégias alternativas de treinamento
 Tipicamente, redes profundas exigem um grande volume de dados de treinamento a fim de apresentar boa capacidade de generalização, mesmo em condições controladas. Tal fato impacta diretamente no tempo de treinamento necessário à sintonia dos parâmetros do sistema de reconhecimento. E em aprendizagem supervisionada (o caso mais comum), tal conjunto de treinamento requer exemplos previamente rotulados, elevando os custos exigidos para sua criação. Esses fatores motivam a busca de técnicas para redução de tempo e esforço na fase de pré-operação (formação de conjunto de dados e procedimentos de treinamento/validação/teste). Nesse sentido, procedimentos para  substituir o treinamento supervisionado completo (\emph{from scratch}) vêm se consolidando~\parencite{herath-2017}, dentre os quais: transferência de conhecimento de redes genéricas pré-treinadas, cujos parâmetros originais são reajustados  (\emph{fine-tuning}) mediante apresentação de exemplos típicos da aplicação pretendida (redução do tempo de treinamento/validação/teste); estratégias de treinamento usando conjunto de dados parcialmente rotulados (redução do tempo/custo de criação de conjunto de dados); e técnicas para ampliação artificial de conjunto de dados (\emph{data augmentation}, redução do tempo para formar o conjunto dados)~\parencite{wang-2015}.
%

% 3- Immediate conection with the research 
O ambiente típico a ser a ser monitorado no escopo deste projeto, áreas industriais, apresenta alta complexidade face às suas características: mudança dos níveis de luminosidade de acordo com os horários de monitoramento; ocorrência de interações de difícil antecipação entre indivíduos entre si e com objetos; variação de aparência de uma mesma atividade, em função de diferentes distâncias e ângulo entre cena de interesse e câmeras de monitoramento; e presença frequente de oclusões. Em suma, trata-se de um ambiente realístico. E a implementação sistemas efetivos de HAR nesse contexto constitui um desafio atual, de grande apelo estratégico em termos científicos e de inovação.
%

% 4- The **contribution**  to the area
Alinhada com o estado da arte, problemas em aberto e tendências em HAR, a presente proposta de pesquisa estabelece como tópicos prioritários de investigação: o uso de dados de entrada  multimodais, ou seja outros tipos de sinais de entrada além do vídeo, visando melhorar o desempenho do sistema de reconhecimento em ambientes de alta complexidade; a exploração de mecanismos de interpretação dos sinais de entrada para identificar atributos que permitam refinar o reconhecimento de ações de curto prazo, fator chave em monitoramento e prevenção visando segurança em ambientes; o aperfeiçoamento de abordagens para transferência efetiva de conhecimento, mediante o uso de redes pré-treinadas; e o desenvolvimento de esquema de treinamento com conjunto de dados parcialmente rotulados.
