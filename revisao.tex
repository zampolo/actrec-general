\nocite{jegham-2020, hussain-2020,yao-2019,kongr-2018,herath-2017}
% Elements:
% 1- Short and precise **overview**; 
%Este projeto trata do reconhecimento automático de atividades humanas mediante o uso de visão computacional baseada em redes profundas. 
% Introdução
O reconhecimento de atividades humanas (\emph{human activity recognition}, HAR) é atualmente uma das áreas de maior interesse em visão computacional, com um amplo leque de aplicações abrangendo de ambientes inteligentes à avaliação do comportamento de multidões, passando por classificação e busca automática em repositórios de imagem e vídeo, para citar algumas das mais investigadas REF.
%
% definições 1
Em razão de uma certa falta de consenso sobre os termos usados na área, é interessante que algumas definições seja postas, para bem caracterizar o escopo deste projeto e o objeto de pesquisa. 
Começaremos por dois termos encontrados amiúde e que são motivo de confusão: \emph{ação} e \emph{atividade}. Sem maiores aprofundamentos, e a despeito de algumas controvérsias, o presente texto adota a direção apontada por~\textcite{herath-2017} que considera que uma \emph{ação} é um evento primitivo, simples, fácil de caracterizar, e que é realizado por uma única pessoa (por exemplo, \emph{indivíduo levanta o braço}). Por outro lado, uma sucessão de ações diferentes forma \emph{atividades}, essas mais ou menos complexas (por exemplo, \emph{preparar café}).
%
% definições 2
Na literatura especializada, o reconhecimento de atividades é também referenciado como classificação de atividades e consiste na avaliação de um vídeo (sinal de entrada), associando-o a um dos itens de uma lista de atividades previamente definida. Outras tarefas relacionadas são: a detecção de atividades, onde o objetivo é precisar os tempos de início e término de uma atividade (localização temporal); e a predição de atividades, cuja finalidade é estimar qual atividade é a mais provável de  ocorrer, dado o que foi observado no vídeo de entrada~\parencite{yao-2019}.%pensar em AUC para diferentes valores de limiar temporal.
%
% sistema: modalidades
Além dos algoritmos de classificação em si, outros elementos influenciam o desempenho dos sistemas de HAR. Dentre os mais relevantes, tem-se a modalidade de aquisição de vídeo e suas sub-categorias. Por exemplo, de acordo com o tipo de sensor usado nas câmeras, o vídeo pode ser classificado como: vídeo convencional, onde o registro compreende a faixa de frequências do espectro visível; vídeo com informação de profundidade, cujo exemplo mais comum são os vídeos obtidos com o Microsoft Kinect; e vídeo obtido por sensores infravermelhos. Outra categoria refere-se à quantidade de câmeras usadas, podendo os sistemas ser classificados como: de câmera única; ou multi-câmera~\parencite{yao-2019}.
%
% cena: quantidade de elementos e suas interações
O conteúdo dos vídeos sob análise também interfere na qualidade dos resultados. Por exemplo, há variação de complexidade da tarefa de reconhecimento, dependendo da quantidade de elementos participantes (indivíduos e objetos) da atividade de interesse e seu grau de interação.%, que pode se dar entre . Pode-se ter situações em que apenas uma pessoa está presente; em outras, pode-se ter dois ou mais indivíduos interagindo; e ainda, o(s) indivíduo(s) pode(m) interagir com objetos.
% cena: background
O fundo (\emph{background}) diante do qual a atividade de interesse se desenrola é outro elemento significativo, podendo variar entre \emph{estático}, típico de ambientes controlados (estúdio de gravação ou laboratório de pesquisa, por exemplo), e altamente dinâmico, como em ruas de grandes cidades ou áreas abertas com a presença de diversas pessoas realizando diferentes atividades simultanemente. A dificuldade do reconhecimento tende a aumentar quanto maior a dinâmica do fundo, pois além da tarefa de estimar a atividade de interesse, o sistema precisa diferenciá-la da movimentação típica da cena.
%
%

%2- **Present state** of the research; 
Atualmente, a tendência e o estado da arte em HAR compreendem técnicas baseadas em redes neurais artificiais profundas. As razões são as mesmas que em outras áreas da visão computacional: os bons resultados decorrentes, em grande parte, da capacidade dessas redes em selecionar automaticamente os atributos mais adequados para a solução de um dado problema, considerando o conjunto de dados disponível. Contudo, um dos principais limitantes de desempenho, segundo \textcite{yao-2019}, é a representação da dependência temporal entre ações, necessária para a caracterização de uma atividade complexa. 
% abordagens: multi-stream, 3D, LSTM
Grosso modo, são duas as principais abordagens mais bem sucedidas em representação temporal. A primeira que trataremos é denominada \emph{multi-stream}, e procura caracterizar informações temporais e espaciais em dois ramos separados na rede neural profunda. O ramo temporal geralmente tem como entrada o fluxo óptico (ou estimativas deste) de cada quadro de vídeo, ao passo que o ramo espacial recebe o quadro propriamente dito. Cada uma das entradas passa por etapas de extração automática de atributos nas camadas convolucionais da rede, cujos resultados são processados e fundidos oportunamente para fins da classificação final. O uso de fluxo óptico, contudo, tem suas desvantagens uma vez que seu cálculo requer quadros adjacentes ao quadro de interesse, resultando em aumento de complexidade computacional e, consequentemente, dificuldade de implementação em tempo real. Em vídeo codificado, há propostas que prescindem do fluxo óptico substituindo-o pelo uso dos vetores de movimento~\parencite{yao-2019}. A segunda, denominada espaço-temporal, emprega redes profundas 3D, pretendendo reunir em um único fluxo informações temporais e espaciais simultaneamente. Ainda motivado pelas inconveniências do uso do fluxo óptico e pela necessidade de caracterização da informação temporal de longo e curto prazo no entendimento de atividades complexas, técnicas baseadas em redes profundas recursivas, tais como a LSTM (\emph{long short-term memory})~\parencite{hochreiter-1997}, têm sido propostas, apresentando compromisso competitivo entre complexidade computacional e qualidade de resultados~\parencite{donahue-2016, herath-2017, xia-2020}.
%
%
% >> I'm here !!
Particularizando para o contexto da aplicação visada neste projeto (reconhecimento de atividades humanas em áreas industriais), pode-se dizer que o ambiente típico a ser a ser monitorado apresenta alta complexidade. Avançando um pouco mais na caracterização desse tipo de ambiente, é importante considerar que outros fatores influenciarão no desempenho do reconhecimento, tais como: variação de luminosidade, de acordo com os horários de monitoramento; presença de mais de um indivíduo interagindo entre si e com objetos, de maneiras difíceis de prever na etapa de treinamento das redes; variação de aparência de uma determinada atividade, em função da distância e do ângulo entre objeto de interesse e câmeras; e ocorrência de oclusões (encobrimento de uma região de interesse por distratores).
%
Apesar bons resultados documentados em REFs, onde diversos dos elementos mencionados, típicos de uma situação realística, são considerados, observa-se acentuada degradação no desempenho dos sistemas de reconhecimento a medida em que o controle das variáveis ambientais é perdido (a expressão em inglês seria \emph{in the wild}). Ou seja, sistemas de reconhecimento automático de atividades humanas em contextos práticos constituem um desafio atual, de grande interesse, e estratégico em termos científicos e de inovação.
%
Em grande medida, as limitações de desempenho em situações realísticas são devidas às grandes variações entre cenas apresentadas na fase de operação e cenas usadas na fase de treinamento. Tipicamente, redes profundas exigem  conjunto de dados de treinamento enorme a fim de apresentar boa capacidade de generalização, mesmo em condições controladas. Considerando sistemas de aprendizagem supervisionada (o mais comum), tal conjunto de treinamento requer dados previamente rotulados, exigindo considerável esforço para sua geração e manutenção. Nesse sentido, as tendências alternativas ao treinamento supervisionado completo de sistema de reconhecimento (\emph{from scratch}) que vêm se consolidado na área~\parencite{herath-2017} são as seguintes: transferência de conhecimento, usando redes de genéricas pré-treinadas, mediante treinamento supervisionado expondo tais redes a exemplos rotulados típicos da aplicação pretendida, procedimento conhecido pelo termo \emph{fine-tuning}; estratégias para treinamento de conjunto de dados parcialmente rotulados, quando se dispõe de grande quantidade de exemplos no conjunto de dados, mas a atribuição de rótulos possui custo elevado; e técnicas para ampliação artificial de conjunto de dados (data augmentation)~\parencite{wang-2015}.

% Falar sobre datasets
O desempenho dos sistemas de reconhecimento de atividades, como qualquer outra aplicação que esteja baseada em redes neurais artificiais, sobre forte influência não apenas da arquitetura da rede, da estratégia de treinamento/validação/teste, mas também do conjunto de dados utizado. Dessa forma, a avaliação comparativa entre diferentes soluções, passa necessariamente pela padronização dos procedimentos de treinamento e teste, bem como pela padronização dos conjuntos de dados.  Ao longo do desenvolvimento da área, acompanhando as mudanças na tecnologia empregada para implementar os sistemas de reconhecimento, foram propostos diferentes conjuntos de dados, variando em número de vídeos, em interação entre elementos, e em complexidade da cena~\parencite{jegham-2020, kongr-2018}. 

%5- **Methodological implications**;
%6- Indicate an **open problem**; 
À medida em que a área avança, outras perspectivas de uso começaram a surgir. Dentre elas, está a que pode ser considerada o próximo passo em visão computacional nessa área: a predição de atividades humanas. Em linhas gerais, pode-se considerar que a distinção entre reconhecimento e predição está na localização temporal pretendida da estimativa de atividade fornecida pelo sistema~\parencite{kongr-2018}. Para um mesmo conjunto de entrada, o reconhecimento visa identificar a ação ocorrida ou caracterizada nesse conjunto de entrada. Por sua vez, a predição fornecerá uma estimativa da atividade que ocorrerá nos quadros de vídeo seguintes ao conjunto de entrada. O reconhecimento olha para o passado, enquanto que a predição considera o passado para projetar uma aposta sobre o que virá em seguida. Claramente, o problema da predição é bem mais desafiador, mas possui um apelo bastante evidente em direção autônoma e ambientes inteligentes voltados para segurança e prevenção de acidentes, onde melhor que remediar é previnir. No aspecto técnico e de implementação dos sistemas, predição e reconhecimento são semelhantes. Contudo, em relação aos resultados, a qualidade da predição é acentuadamente menor que em reconhecimento, aumentando ainda mais com o distanciamento temporal entre conjunto de entrada e predição. A representação temporal e da sucessão de ações/atividades assume importância ainda maior na tarefa de predição.
%7- The **contribution**  to the area
% uma outra contribuição pode ser a criação de um dataset que ficará disponível? Segredo industrical em risco !!
Alinhada com o estado da arte, problemas em aberto e tendências em reconhecimento automático de atividades humanas, a presente proposta de pesquisa estabelece como tópicos prioritários de investigação: o uso de dados de entrada  multimodais, ou seja outros tipos de entrada além do vídeo, para melhorar o desempenho do sistema de reconhecimento em ambientes de alta complexidade; a exploração de mecanismos de interpretação do sinais de entrada para identificar atributos que permitam reconhecer ações de curto prazo, fator chave para melhorar a qualidade de sistemas de predição visando segurança e proteção em ambientes industriais; o aperfeiçoamento de abordagens para transferência efetiva de conhecimento, mediante o uso de redes pré-treinadas; e o desenvolvimento de esquema de treinamento com conjunto de dados parcialmente rotulados.
