%#Elements:
%1- Short and precise **overview**; 
%  - Concentuação/Taxonomia: o que é o reconhecimento de ações (ações ou atividades?, interação, )
%  - Tipos de sistemas. Diferentes classificações: número de pessoas envolvidas, ambiente, interação com pessoas e/ou objetos, predição/reconhecimento, uso de hadcrafted features ou extração automática de features/atributos usando cnns, ou ainda abordagens híbridas
Este projeto trata do reconhecimento automático de atividades humanas mediante o uso de visão computacional baseada em redes profundas. Em razão da falta de consenso sobre os termos usados nas área envolvidas, é interessante que algumas definições sejam postas a fim de caracterizar o escopo deste projeto bem como o objeto de pesquisa. 
%
Assim, a primeira distinção a ser feita é entre \emph{ação} e \emph{atividade}. Sem maiores aprofundamentos, e a despeito das controvérsias, considera-se neste texto que uma \emph{ação} é um evento primitivo, simples, fácil de caracterizar, e que é realizado por uma única pessoa (por exemplo, \emph{homem levanta o braço}). Por outro lado, a sucessão de ações diferentes forma \emph{atividades}, essas mais ou menos complexas (por exemplo, \emph{preparar café}). 
%
Outro aspecto a ser considerado, diz respeito à quantidade de elementos participantes de uma cena. Pode-se ter situações em que apenas uma pessoa está presente; em outras, pode-se ter dois ou mais indivíduos interagindo; e ainda, o(s) indivíduo(s) pode(m) interagir com objetos. As possibilidades de variação aqui são inúmeras, influenciando no grau de complexidade da cena.
%
Cenas não são apenas compostas pelos atores principais que desempenham interações e/ou atividades de interesse, mas também pelo pelo contexto, simplificadamente denominado de \emph{fundo}, que pode variar entre \emph{estático} em ambientes controlados (em um estúdio ou laboratório, por exemplo) ou altamente dinâmicos, como em ruas de grandes cidades ou áreas com a presença de diversas pessoas realizando diferentes atividades. A dificuldade do reconhecimento aumenta com a dinamicidade do fundo, pois além da tarefa de estimar a atividade sendo desempenhada, o sistema precisa diferenciar entre movimentação dos objetos de interesse e das modificações típicas do fundo.
%
O reconhecimento automático de atividades é uma das áreas de maior interesse em visão computacional com um leque de possíveis aplicações abrangendo áreas como ambientes inteligentes, classificação e busca automática em repositórios de imagem e vídeo, e avaliação do comportamento de multidões, dentre outras.
%
%2- **Present state** of the research; 
% Abordagens:
%% Redes multi-streams
%% Redes 3D
%% Soluções híbridas.
Atualmente, a tendência e o estado da arte nessa área compreendem técnicas baseadas em redes neurais profundas. AUTOT-REF aponta como principal desafio a representação da dependência temporal entre ações na caracterização de uma atividade complexa. Tal desafio, em grande parte, tem orientado os esforços da comunidade científica nos avanços da área. 
%
Grosso modo, são duas as principais abordagens mais bem sucedidas na busca de soluções para a representação temporal. A primeira delas, denominada multi-fluxo (multi-stream), procura caracterizar informações temporais e espaciais em dois ramos separados na rede neural profunda. O ramo temporal geralmente tem como entrada o fluxo óptico (ou estimativas deste) de cada quadro de vídeo, ao passo que o ramo espacial recebe o quadro propriamente dito. Cada uma das entradas passa por etapas de extração automática de atributos nas camadas convolucionais da rede, cujos resultados são processados e fundidos oportunamente para fins da classificação final. O uso de fluxo óptico, contudo, tem suas desvantagens uma vez que seu cálculo requer quadros adjacentes ao quadro de interesse, resultando em aumento de complexidade computacional e, consequentemente, dificuldade de implementação em tempo real. A outra abordagem justamente prescinde do fluxo óptico, e compreende o uso de redes profundas 3D que poderiam reunir em um único fluxo informações temporais e espaciais simultaneamente. 
%
Ainda motivado pelas inconveniências do uso do fluxo óptico e pela necessidade de caracterização da informação temporal de longo e curto prazo no entendimento de atividades complexas, técnicas baseadas em redes profundas recursivas, tais como a LSTM (\emph{long short-term memory}) e outras REF, têm sido propostas, apresentando compromisso competitivo entre complexidade computacional e qualidade de resultados REF.

%3- **Immediate connection** with your own research; 
Particularizando para o contexto da aplicação visada neste projeto (reconhecimento de atividades humanas em áreas industriais), pode-se dizer que o ambiente típico a ser a ser monitorado apresenta alta complexidade. Avançando um pouco mais na caracterização desse tipo de ambiente, é importante considerar que outros fatores influenciarão no desempenho do reconhecimento, tais como: variação de luminosidade, de acordo com os horários de monitoramento; presença de mais de um indivíduo interagindo entre si e com objetos, de maneiras difíceis de prever na etapa de treinamento das redes; variação de aparência de uma determinada atividade, em função da distância e do ângulo entre objeto de interesse e câmeras; e ocorrência de oclusões (encobrimento de uma região de interesse por distratores).
%
Apesar bons resultados documentados em REFs, onde diversos dos elementos mencionados, típicos de uma situação realística, são considerados, observa-se acentuada degradação no desempenho dos sistemas de reconhecimento a medida em que o controle das variáveis ambientais é perdido (a expressão em inglês seria \emph{in the wild}). Ou seja, sistemas de reconhecimento automático de atividades humanas em contextos práticos constituem um desafio atual, de grande interesse, e estratégico em termos científicos e de inovação.

%4- Clear and logical **discussion of the theoretical framework**; 
Em grande medida, as limitações de desempenho em situações realísticas são devidas às grandes variações entre cenas apresentadas na fase de operação e cenas usadas na fase de treinamento. Tipicamente, redes profundas exigem  conjunto de dados de treinamento enorme a fim de apresentar boa capacidade de generalização, mesmo em condições controladas. Considerando sistemas de aprendizagem supervisionada (o mais comum), tal conjunto de treinamento requer dados previamente rotulados, exigindo considerável esforço para sua geração e manutenção. Nesse sentido, as tendências alternativas ao treinamento supervisionado completo de sistema de reconhecimento (\emph{from scratch}) que vêm se consolidado na área REF são as seguintes: transferência de conhecimento, usando redes de genéricas pré-treinadas, mediante treinamento supervisionado expondo tais redes a exemplos rotulados típicos da aplicação pretendida, procedimento conhecido pelo termo \emph{fine-tuning}; estratégias para treinamento de conjunto de dados parcialmente rotulados, quando se dispõe de grande quantidade de exemplos no conjunto de dados, mas a atribuição de rótulos possui custo elevado.
%5- **Methodological implications**;
%6- Indicate an **open problem**; 
À medida em que a área avança, outras perspectivas de uso começaram a surgir. Dentre elas, está a que pode ser considerada o próximo passo em visão computacional nessa área: a predição de atividades humanas. Em linhas gerais, pode-se considerar que a distinção entre reconhecimento e predição está na localização temporal pretendida da estimativa de atividade fornecida pelo sistema REF. Para um mesmo conjunto de entrada, o reconhecimento visa identificar a ação ocorrida ou caracterizada nesse conjunto de entrada. Por sua vez, a predição fornecerá uma estimativa da atividade que ocorrerá nos quadros de vídeo seguintes ao conjunto de entrada. O reconhecimento olha para o passado, enquanto que a predição considera o passado para projetar uma aposta sobre o que virá em seguida. Claramente, o problema da predição é bem mais desafiador, mas possui um apelo bastante evidente em direção autônoma e ambientes inteligentes voltados para segurança e prevenção de acidentes, onde melhor que remediar é previnir REF. No aspecto técnico e de implementação dos sistemas, predição e reconhecimento são semelhantes REF. Contudo, em relação aos resultados, a qualidade da predição é acentuadamente menor que em reconhecimento, aumentando ainda mais com o distanciamento temporal entre conjunto de entrada e predição. A representação temporal e da sucessão de ações/atividades assume importância ainda maior na tarefa de predição.
%7- The **contribution**  to the area
Alinhada com o estado da arte, problemas em aberto e tendências em reconhecimento automático de atividades humanas, a presente proposta de pesquisa estabelece como tópicos prioritários de investigação: o uso de dados de entrada  multimodais, ou seja outros tipos de entrada além do vídeo, para melhorar o desempenho do sistema de reconhecimento em ambientes de alta complexidade; a exploração de mecanismos de interpretação do sinais de entrada para identificar atributos que permitam reconhecer ações de curto prazo, fator chave para melhorar a qualidade de sistemas de predição visando segurança e proteção em ambientes industriais; o aperfeiçoamento de abordagens para transferência efetiva de conhecimento, mediante o uso de redes pré-treinadas; e o desenvolvimento de esquema de treinamento com conjunto de dados parcialmente rotulados.
