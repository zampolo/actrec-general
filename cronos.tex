% cronos.tex
\subsection{Atividades e metas}
\begin{itemize}
	\item Atividade A1: revisão bibliográfica e atualização do estado-da-arte
	\begin{itemize}
		\item Descrição: a fim de entregar real valor agregado, a equipe executora acompanhará com regularidade ao longo de todo o período de execução do projeto o estado-da-arte na área de videomonitoramento inteligente através de artigos publicados em períódicos e eventos técnico-científicos de comprovada qualidade.
		\item Metas associadas: texto de revisão da literatura científica e coleção de artigos e sites selecionados; primeira versão de ambos (texto e coleção) no terceiro mês de execução do projeto (M1a), e versões atualizadas nos meses 6 (M1b), 12 (M1c), 18 (M1d) e 24 (M1e).
		\item Meses: todos.
	\end{itemize}
	\item Atividade A2: aquisição de equipamento.
	\begin{itemize}
		\item Descrição: a aquisição de equipamento de videomonitoramento para montagem de setup experimental é essencial para o sucesso do projeto, sendo particularmente importante nas etapas de seleção e teste de técnicas candidatas; a atividade compreende a definição de empresas fornecedoras, e realização de procedimentos de compra, acompanhamento e recebimento de equipamento de videomonitoramento. 
		\item Meses: 1 a 3.
		\item Metas associadas: equipamento disponível para equipe executora com documentação pertinente (M2).
	\end{itemize}
	\item Atividade A3: instalação, teste e treinamento de equipamento adquirido.
	\begin{itemize}
		\item Descrição: montagem, instalação e teste de equipamento adquirido; deve ser realizado pela equipe executora ou empresa especializada, dependendo do tipo de equipamento; se necessário a equipe de executora deverá realizar treinamento para uso do equipamento; em caso de dano de transporte ou de fábrica, acionar a garantia do produto.
		\item Meses: 3 e 4.
		\item Metas associadas: equipamento instalado e operacional, pronto para ser usado (M3).
	\end{itemize}
	\item Atividade A4: investigação de técnicas candidatas.
	\begin{itemize}
		\item Descrição: implementar e testar em ambiente laboratorial abordagens com potencial para atender aos objetivos do projeto.
		\item Meses: 1 a 12.
		\item Metas associadas: setup experimental (M4a), protótipo funcional (M4b), implementação funcional em Python da técnica selecionada em repositório do projeto (M4c), documento justificando a técnica selecionada (M4d).
	\end{itemize}
	\item Atividade A5: realização de testes em ambiente relevante.
	\begin{itemize}
		\item Descrição: ambiente relevante significa um contexto de operação intermediário entre o ambiente laboratorial, em que as variáveis do experimento estão sob controle, e o ambiente operacional, onde o sistema deverá funcionar na prática. A realização de teste do sistema de videomonitoramento em ambiente relevante permitirá observar o desempenho e comportamento do sistema em situações não previstas inicialmente, bem como subsidiará as correções necessárias para seu correto funcionamento.
		\item Meses: 13 a 18.
		\item Metas associadas: protótipo funcional (M5a), nova versão funcional em Python do sistema armazenado em repositório do projeto (M5b), documentação com testes de desempenho, adaptações e correções feitas (M5c).
	\end{itemize}
	\item Atividade A6: integração do sistema em ambiente operacional (4o semestre).
	\begin{itemize}
		\item Descrição: compreende a adaptação do sistema testado em ambiente relevante para o local de operação real e integração com o sistema de monitoramento da empresa demandadora/parceira; inclui ainda ações de transferência de conhecimento/tecnologia ao corpo técnico da empresa.
		\item Meses: 19 a 24.
		\item Metas associadas: sistema proposto integrado ao aparato de monitoramento da empresa demandadora/parceira e operando em ambiente real (M6a); manual de operação do sistema (M6b); eventualmente, realização de minicurso/seminário apresentando o sistema implementado (M6c).
	\end{itemize}
	\item Atividade A7: redação de relatório técnico-financeiro (a cada 6 meses).
	\begin{itemize}
		\item Descrição: consolidação de registros técnicos e financeiros em relatórios redigidos para prestar contas dos recursos utilizados, subsidiar a redação de artigos técnicos e informar a quem de direito sobre o andamento geral do projeto.
		\item Meses: 6, 12, 18, e 24.
		\item Metas associadas: relatórios parciais (M7a, M7b, e M7c) e relatório final (M7d).
	\end{itemize}
	\item  Atividade A8: redação de artigos e participação em eventos técnicos-científicos
	\begin{itemize}
		\item Descrição: a participação nesses eventos deve ocorrer, de preferência, condicionada à aprovação e consequente de trabalhos de autoria da equipe executora em tema associado ao objeto de estudo do projeto; cumpre diversas finalidades: avaliação externa da qualidade do trabalho sendo desenvolvido, divulgação do projeto e parceiros; atualização sobre o estado-da-arte no tema da pesquisa; estabelecimento de contatos para possíveis futuras parcerias.
		\item Meses: não há mês definido \emph{a priori}, contudo, estima-se que o esforço de redação deve ser realizado ao final do segundo e terceiro semestres, ou seja, meses 11 e 12, e meses 17 e 18.
		\item Metas associadas: publicação de pelo menos dois artigos (M8a, e M8b) em eventos técnicos-científicos (um nacional e outro internacional) durante a vigência do projeto.
	\end{itemize}
	\item Atividade A9: redação de artigo para publicação em periódico.
	\begin{itemize}
		\item Descrição: a redação de artigo com estrutura e qualidade para publicação em periódicos especializados com revisão por pares indexado no Qualis da CAPES
		\item Meses: 19 a 24.
		\item Metas associadas: publicação de, pelo menos, um artigo em periódico com Qualis A (M9).
	\end{itemize}
\end{itemize}

\subsection{Cronograma}
Na Tabela \ref{tab:crono}, as atividades e metas estão dispostas ao longo dos meses de desenvolvimento do projeto.
\begin{table}[h!]
	\caption{Tabela de cronograma (as colunas numeradas referem-se aos meses de desenvolvimento do projeto)}
\rowcolors{2}{lightgray}{}%lightgray
\begin{tabular}{ p{0.35\textwidth} cccccccccccc}
\toprule
%\hline\hline
   \rowcolor{lgray}
   Atividades -- ano 01                                                            & 01 & 02 & 03 & 04 & 05 & 06 & 07 & 08 & 09 & 10 & 11 & 12 \\
\midrule
   Atividade A1: revisão bibliográfica e atualização do estado-da-arte             &  x & x  & M1a& x  & x  & M1b& x  & x  & x  & x  & x  & M1c\\
   Atividade A2: aquisição de equipamento                                          &  x & x  & M2 &    &    &    &    &    &    &    &    &    \\
   Atividade A3: instalação, teste e treinamento de equipamento adquirido          &    &    & x  & M3 &    &    &    &    &    &    &    &    \\
   Atividade A4: investigação de técnicas candidatas                               &  x & x  & x  & x  & x  & x  & x  & x  & x  & x  & x  & M4 \\
   Atividade A7: redação de relatório técnico-financeiro (a cada 6 meses)          &    &    &    &    &    & M7a&    &    &    &    &    & M7b\\
   Atividade A8: redação de artigos e participação em eventos técnicos-científicos &    &    &    &    &    &    &    &    &    &    & x  & M8a\\
	\midrule
%   \rowcolor{lightblue}%lgray}
   \rowcolor{lgray}
   Atividades -- ano 02                                                            & 13 & 14 & 15 & 16 & 17 & 18 & 19 & 20 & 21 & 22 & 23 & 24 \\
\midrule
   Atividade A1: revisão bibliográfica e atualização do estado-da-arte             & x  & x  & x  & x  & x  & M1d& x  & x  & x  & x  & x  & M1e\\
   Atividade A5: realização de testes em ambiente relevantem                       & x  & x  & x  & x  & x  & M5 &    &    &    &    &    &    \\
   Atividade A6: integração do sistema em ambiente operacional                     &    &    &    &    &    &    & x  & x  & x  & x  & x  & M6 \\
   Atividade A7: redação de relatório técnico-financeiro (a cada 6 meses)          &    &    &    &    &    & M7c&    &    &    &    &    & M7d\\
   Atividade A8: redação de artigos e participação em eventos técnicos-científicos &    &    &    &    & x  & M8b&    &    &    &    &    &    \\
   Atividade A9: redação de artigo para publicação em periódico                    &    &    &    &    &    &    & x  & x  & x  & x  & x  & M9 \\
\bottomrule
%\hline\hline   
\end{tabular}
	\label{tab:crono}
\end{table}
\clearpage
