% cronos.tex

\begin{itemize}
	\item Atividade A1: revisão bibliográfica e atualização do estado-da-arte
	\begin{itemize}
		\item Descrição: a fim de entregar real valor agregado, a equipe executora acompanhará com regularidade ao longo de todo o período de execução do projeto o estado-da-arte na área de videomonitoramento inteligente através de artigos publicados em períódicos e eventos técnico-científicos de comprovada qualidade.
		\item Metas associadas: texto de revisão da literatura científica e coleção de artigos e sites selecionados; primeira versão de ambos no terceiro mês de execução do projeto (M1a), e versões atualizadas nos meses 6 (M1b), 12 (M1c), 18 (M1d) e 24 (M1e).
		\item Meses: todos
	\end{itemize}
	\item Atividade A2: aquisição de equipamento
	\begin{itemize}
		\item Descrição: a aquisição de equipamento de videomonitoramento para montagem de setup experimental é essencial para o sucesso do projeto, sendo particularmente importante nas etapas de seleção e teste de técnicas candidatas; a atividade compreende a definição de empresas fornecedoras, e realização de procedimentos de compra, acompanhamento e recebimento de equipamento de videomonitoramento. 
		\item Meses: 1 -- 3
		\item Metas associadas: equipamento disponível para equipe executora com documentação pertinente (M2)
	\end{itemize}
	\item Atividade A3: instalação, teste e treinamento de equipamento adquirido
	\begin{itemize}
		\item Descrição: montagem, instalação e teste de equipamento adquirido; deve ser realizado pela equipe executora ou empresa especializada, dependendo do tipo de equipamento; se necessário a equipe de executora deverá realizar treinamento para uso do equipamento; em caso de dano de transporte ou de fábrica, acionar a garantia do produto.
		\item Meses: 3 -- 4
		\item Metas associadas: equipamento instalado e operacional, pronto para ser usado (M3).
	\end{itemize}
	\item Atividade A4: investigação de técnicas candidatas
	\begin{itemize}
		\item Descrição: implementar e testar em ambiente laboratorial abordagens com potencial para atender aos objetivos do projeto.
		\item Meses: 1 -- 12
		\item Metas associadas: setup experimental (M4a), protótipo funcional (M4b), implementação funcional em Python da técnica selecionada em repositório do projeto (M4c), documento justificando a técnica selecionada (M4d)
	\end{itemize}
	\item Atividade A5: realização de testes em ambiente relevante
	\begin{itemize}
		\item Descrição: ambiente relevante consistem em um contexto de operação intermediário entre o ambiente laboratorial em que as variáveis do experimento estão sob controle e o ambiente operacional onde o sistema deverá funcionar na prática. A realização de teste do sistema de videomonitoramento em ambiente relevante permitirá observar o desempenho e comportamento do sistema em situações não previstas inicialmente, bem como subsidiará as correções necessárias para seu correto funcionamento.
		\item Meses: 13 -- 18
		\item Metas associadas: protótipo funcional (M5a), nova versão funcional em Python do sistema em repositório do projeto (M5b), documentação com testes de desempenho, adaptações e correções feitas (M5c)
	\end{itemize}
	\item Atividade A6: integração do sistema em ambiente operacional (4o semestre)
	\begin{itemize}
		\item Descrição: compreende a adaptação do sistema testado em ambiente relevante para o local de operação real e integração com o sistema de monitoramento da empresa demandadora/parceira; inclui ainda ações de transferência de conhecimento/tecnologia ao corpo técnico da empresa.
		\item Meses: 19 -- 24
		\item Metas associadas: sistema proposto integrado ao aparato de monitoramento da empresa demandadora/parceira e operando em ambiente real (M6a); manual de operação do sistema (M6b); eventualmente, realização de minicurso/seminário apresentado o sistema implementado (M6c).
	\end{itemize}
	\item Atividade A7: redação de relatório técnico-financeiro (a cada 6 meses)
	\begin{itemize}
		\item Descrição:
		\item Meses: 6, 12, 18, e 24
		\item Metas associadas: relatórios parciais (M7a, M7b, e M7c) e relatório final (M7d)
	\end{itemize}
	\item  Atividade A8: participação em eventos técnicos-científicos
	\begin{itemize}
		\item Descrição:
		\item Meses:
		\item Metas associadas:
	\end{itemize}
	\item Atividade A8: redação de artigos em periódicos e eventos.
	\begin{itemize}
		\item Descrição:
		\item Meses:
		\item Metas associadas:
	\end{itemize}
\end{itemize}

