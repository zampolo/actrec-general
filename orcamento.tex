%orcamento.tex
\subsection{Recursos humanos}
A equipe de execução foi concebida como sendo composta por dois alunos de graduação (IC), dois mestrandos (MT), um doutorando (DR), dois pesquisadores seniors (PQ) e um coordenador. Tal configuração foi elaborada no sentido de obter um bom compromisso entre custo e maturidade técnica suficiente para executar o projeto no prazo estipulado. Os valores de bolsa para IC, MT e DR têm como referência os valores pagos pelo CNPq. A Tabela~\ref{tab:rhmes} mostra as despesas mensais totais e por tipo de bolsa nos primeiro e segundo anos de projeto.
A Tabela~\ref{tab:rhano} apresenta os valores anuais totais de desembolso com recursos humanos do projeto.
Por sua vez, a Tabela~\ref{tab:rhtipo} especifica o investimento total em recursos humanos por tipo.

\begin{table}[h!]
\scriptsize
	\caption{Recursos humanos -- desembolso mensal por ano}
\rowcolors{2}{lightgray}{}%lightgray
\begin{tabular}{ lrrrrrrrrrrrr}
\toprule
%\hline\hline
   \rowcolor{lgray}
   Tipo      & 01      & 02      & 03      & 04      & 05      & 06      & 07      & 08      & 09      & 10      & 11      & 12      \\
\midrule
   IC-01     &   400,00&   400,00&   400,00&   400,00&   400,00&   400,00&   400,00&   400,00&   400,00&   400,00&   400,00&   400,00\\
   IC-02     &   400,00&   400,00&   400,00&   400,00&   400,00&   400,00&   400,00&   400,00&   400,00&   400,00&   400,00&   400,00\\
   MT-01     & 1.500,00& 1.500,00& 1.500,00& 1.500,00& 1.500,00& 1.500,00& 1.500,00& 1.500,00& 1.500,00& 1.500,00& 1.500,00& 1.500,00\\
   MT-02     & 1.500,00& 1.500,00& 1.500,00& 1.500,00& 1.500,00& 1.500,00& 1.500,00& 1.500,00& 1.500,00& 1.500,00& 1.500,00& 1.500,00\\
   DR        & 2.200,00& 2.200,00& 2.200,00& 2.200,00& 2.200,00& 2.200,00& 2.200,00& 2.200,00& 2.200,00& 2.200,00& 2.200,00& 2.200,00\\
   PQ-01     & 1.500,00& 1.500,00& 1.500,00& 1.500,00& 1.500,00& 1.500,00& 1.500,00& 1.500,00& 1.500,00& 1.500,00& 1.500,00& 1.500,00\\
   PQ-02     & 1.500,00& 1.500,00& 1.500,00& 1.500,00& 1.500,00& 1.500,00& 1.500,00& 1.500,00& 1.500,00& 1.500,00& 1.500,00& 1.500,00\\
   CO        & 2.000,00& 2.000,00& 2.000,00& 2.000,00& 2.000,00& 2.000,00& 2.000,00& 2.000,00& 2.000,00& 2.000,00& 2.000,00& 2.000,00\\
	\midrule
   Total     & 5.200,00& 5.200,00& 5.200,00& 5.200,00& 5.200,00& 5.200,00& 5.200,00& 5.200,00& 5.200,00& 5.200,00& 5.200,00& 5.200,00\\
   
%   \rowcolor{lightblue}%lgray}
\midrule
\midrule
   \rowcolor{lgray}
   Tipo      & 13      & 14      & 15      & 16      & 17      & 18      & 19      & 20      & 21      & 22      & 23      & 24      \\
\midrule
   IC-01     &   400,00&   400,00&   400,00&   400,00&   400,00&   400,00&   400,00&   400,00&   400,00&   400,00&   400,00&   400,00\\
   IC-02     &   400,00&   400,00&   400,00&   400,00&   400,00&   400,00&   400,00&   400,00&   400,00&   400,00&   400,00&   400,00\\
   MT-01     & 1.500,00& 1.500,00& 1.500,00& 1.500,00& 1.500,00& 1.500,00& 1.500,00& 1.500,00& 1.500,00& 1.500,00& 1.500,00& 1.500,00\\
   MT-02     & 1.500,00& 1.500,00& 1.500,00& 1.500,00& 1.500,00& 1.500,00& 1.500,00& 1.500,00& 1.500,00& 1.500,00& 1.500,00& 1.500,00\\
   DR        & 2.200,00& 2.200,00& 2.200,00& 2.200,00& 2.200,00& 2.200,00& 2.200,00& 2.200,00& 2.200,00& 2.200,00& 2.200,00& 2.200,00\\
   PQ-01     & 1.500,00& 1.500,00& 1.500,00& 1.500,00& 1.500,00& 1.500,00& 1.500,00& 1.500,00& 1.500,00& 1.500,00& 1.500,00& 1.500,00\\
   PQ-02     & 1.500,00& 1.500,00& 1.500,00& 1.500,00& 1.500,00& 1.500,00& 1.500,00& 1.500,00& 1.500,00& 1.500,00& 1.500,00& 1.500,00\\
   CO        & 2.000,00& 2.000,00& 2.000,00& 2.000,00& 2.000,00& 2.000,00& 2.000,00& 2.000,00& 2.000,00& 2.000,00& 2.000,00& 2.000,00\\
	\midrule
   Total     & 5.200,00& 5.200,00& 5.200,00& 5.200,00& 5.200,00& 5.200,00& 5.200,00& 5.200,00& 5.200,00& 5.200,00& 5.200,00& 5.200,00\\
\bottomrule
%\hline\hline   
\end{tabular}
	\label{tab:rhmes}
\end{table}
%====================

\begin{table}[!h]
\centering
%\scriptsize
	\caption{Recursos humanos por ano do projeto}
%\rowcolors{2}{lightgray}{}%lightgray
\begin{tabular}{ lr}
\toprule
%\hline\hline
%   \rowcolor{lgray}
   Ano       & Total (R\$)  \\
\midrule
   Ano 01    &   132.000,00 \\
   Ano 02    &   132.000,00 \\
\midrule
 %  \rowcolor{lgray}
   Total     &   264.000,00 \\
\bottomrule
\end{tabular}
	\label{tab:rhano}
\end{table}
%=======================

\begin{table}[!h]
\centering
%\scriptsize
	\caption{Recursos humanos por tipo}
%\rowcolors{2}{lightgray}{}%lightgray
\begin{tabular}{ lr}
\toprule
%\hline\hline
%   \rowcolor{lgray}
   Tipo  & Total (R\$) \\
\midrule
   IC    &   19.200,00 \\
   MT    &   72.000,00 \\
   DR    &   52.800,00 \\
   PQ    &   72.000,00 \\
   CO    &   48.000,00 \\
\midrule
 %  \rowcolor{lgray}
   Total     &   264.000,00 \\
\bottomrule
\end{tabular}
	\label{tab:rhtipo}
\end{table}
 
\newpage
\subsection{Serviços de terceiros}
A título de justificativa dos itens relacionados na Tabela~\ref{tab:ter}: 
\begin{itemize}
	\item A \emph{instalação de sistema de câmeras} é normalmente oferecida ou indicada pela empresa que vende esse tipo de sistema e é recomendável fazer uso de tal serviço profissional, tanto para não se perder a garantia dos equipamentos, quanto para se poder utilizar o mais rapidamente possível o sistema de aquisição de vídeo. Esse último aspecto é de grande importância para o atendimento dos prazos do projeto;
	\item O item \emph{computação em nuvem} tem por objetivo resgardar os prazos de entrega nas diferentes fases da execução contra atrasos na aquisição de computadores de alto desempenho, depreciação cambial (o que impediria a aquisição de computadores com a qualidade/preço inicialmente cotados), e eventuais panes nos sistemas de processamento;
	\item As \emph{taxas open access} visam prover recurso para o pagamento referente à publicação de artigos técnicos relacionados à produção do projeto  em periódicos de acesso livre de prestígio internacional.
\end{itemize}
\begin{table}[!h]
\centering
%\scriptsize
	\caption{Serviços de terceiros por mês}
%\rowcolors{2}{lightgray}{}%lightgray
\begin{tabular}{clr}
\toprule
%\hline\hline
%   \rowcolor{lgray}
	Mês   & Descrição & Custo (R\$) \\
	\midrule
	04    & Instalação de sistema de câmeras   & 2.000,00 \\
	07    & Computação em nuvem                & 3.000,00 \\
	08    & Computação em nuvem                & 3.000,00 \\
	09    & Computação em nuvem                & 3.000,00 \\
	10    & Computação em nuvem                & 3.000,00 \\
	11    & Computação em nuvem                & 3.000,00 \\
	12    & Computação em nuvem                & 3.000,00 \\
	13    & Computação em nuvem                & 3.000,00 \\
	14    & Computação em nuvem                & 3.000,00 \\
	15    & Computação em nuvem                & 3.000,00 \\
	16    & Computação em nuvem                & 3.000,00 \\
	17    & Computação em nuvem                & 3.000,00 \\
	17    & Computação em nuvem                & 3.000,00 \\
	24    & Taxas Open Access                  &10.500,00 \\
\midrule
 %  \rowcolor{lgray}
	      & Total                              &48.500,00 \\
\bottomrule
\end{tabular}
	\label{tab:ter}
\end{table}
\newpage

\subsection{Custeio}
Os itens de custeio apresentados na Tabela~\ref{tab:custeio} referem-se principalmente a material de escritório a ser utilizado necessariamente em atividades relacionadas à execução do projeto de pesquisa.
\begin{table}[!h]
\centering
%\scriptsize
	\caption{Custeio por mês}
%\rowcolors{2}{lightgray}{}%lightgray
\begin{tabular}{clrcr}
\toprule
%\hline\hline
%   \rowcolor{lgray}
	Mês   & Descrição              & Custo (R\$) & Quantidade & Total \\
	\midrule
	01    & Toner para impressora  & 400,00      & 01         & 400,00 \\
	01    & Resma de papel A4      &  25,00      & 10         & 250,00 \\
	13    & Toner para impressora  & 400,00      & 01         & 400,00 \\
	13    & Resma de papel A4      &  25,00      & 10         & 250,00 \\
\midrule
 %  \rowcolor{lgray}
	      &                        &             & Total      &1.300,00 \\
\bottomrule
\end{tabular}
	\label{tab:custeio}
\end{table}
 
\subsection{Capital}
Os itens de capital relacionados na Tabela~\ref{tab:capital} consistem em equipamentos para aquisição (câmera de vigilância, gravador digital, e hd) e processamento (desktop, workstation, e notebook) de vídeo digital. Os desktops e workstation possuem configuração para suportar as fases de treinamento e produção de sistemas baseados em aprendizado de máquina que lidem com grande volume de dados, o que é típico em aplicaçãoes de reconhecimento de atividades por vídeo e redes neurais convolucionais. Os notebooks são de alto desempenho e devem apoiar a realização de testes em campo do sistema de reconhecimento de atividades.
\begin{table}[!h]
\centering
\scriptsize
	\caption{Capital por mês}
%\rowcolors{2}{lightgray}{}%lightgray
\begin{tabular}{clp{0.35\textwidth}rcr}
\toprule
%\hline\hline
%   \rowcolor{lgray}
	Mês & Equipamento & Descrição               & Custo (R\$) & Quantidade & Total    \\
	\midrule
	01  & Câmera de vigilância  & Speed dome; Compactação de vídeo: H.264 (MPEG-4 Parte 10/AVC) Motion JPEG; Resoluções: HDTV 720p 1280x720 a 320x180; Taxa de quadros: H.264: Até 30/25 fps (60/50 Hz) em todas as resoluções Motion JPEG: Até 30/25 fps (60/50 Hz) em todas as resoluções & 6.000,00    & 02         &  12.000,00 \\
	01  & Gravador digital      & Gravador digital para sistema de câmeras de vigilância, 4 canais                     & 3.000,00    & 01         &   3.000,00 \\
	01  & HD                    & HD para gravador digital 8 TB                                                        & 2.500,00    & 01         &   2.500,00 \\
	01  & Desktop               & iAMD Ryzen 7 3700X (8 Núcleos e 16 Threads, 3.6GHz, Turbo até 4.4GHz, Cache de 32MB); Nvidia GeforceTM GTX 1650 Super 4GB 1280 cuda cores; RAM 16GB DDR4; SSD 1TB Workstation; Placa de rede Wireless Dual 802.11 AC;bMouse e teclado com fio; Monitor: 23.8" (1920x1080) (HDMI, DP); Frete: Transportadora com seguro - Grátis (CIF)                                                                                          & 10.757,34   & 04         &  43.029,36 \\
	13  & Workstation           & AMD Ryzen Threadripper 3970X (32 Núcleos e 64 Threads, 3.7GHz, Turbo até 4.5GHz, Cache de 144MB); Nvidia GeforceTM RTX 2080 TI 11GB 3584 cuda cores; 64GB DDR4 3200MHz (4x16GB); SSD M.2 PCIe X4 NVMe 1TB Workstation Class; HDD 2 TB 7200RPM 128MB SATA III Enterprise Class; Refrigeração: Refrigeração líquida dupla; Rede: Integrada 10/100/1000;Frete: Transportadora com seguro - Grátis (CIF)                           & 48.070,55   & 01         &  48.070,55 \\
	13  & Notebook              & Intel® Core™ i7 1065G7 1,3 GHz, 8 MB Cache, 16 GB (8 GB Onboard + 8 GB Offboard), 512 GB SSD PCIe NVME M2, Tela > 15'', Placa de Vídeo GeForce MX330 com 2GB de memória GDDR5; wifi 802.11ac, câmera frontal, Bluetooth, saídas: 1x HDMI 1.4 1x 3.5mm Combo Audio Jack 2x USB 2.0 Type-A 1x USB 3.2 Gen 1 Type-A 1x USB 3.2 Gen 1 Type-C,                                                                                      & 8.000,00    & 02         &  16.000,00 \\
\midrule
 %  \rowcolor{lgray}
             &                      &                                                                                      &             & Total      & 124.599,91 \\
\bottomrule
\end{tabular}
	\label{tab:capital}
\end{table}
\newpage

\subsection{Passagens e diárias}
Na Tabela~\ref{tab:viagem} estão especificados recursos para viabilizar a participação de pesquisadores em eventos técnicos de abrangência nacional e internacional. A finalidade de tais participações seria a atualização técnica da equipe e a apresentação de trabalhos produzidos no âmbito do projeto e aprovados por comitê de revisores. Face ao tempo para execução do projeto, estima-se a possibilidade a participação em dois eventos nacionais e um evento internacional. Para os eventos nacionais, é prevista a participação de dois membros do projeto, com cinco diárias de manutenção cada. Para o evento internacional, prevê-se a participação de um pesquisador da equipe e sete diárias. Para os valores de diárias nacionais e internacionais, utilizou-se como referência os valores correspondentes pagos pelo CNPq.
\begin{table}[!h]
\centering
%\scriptsize
	\caption{Passagens e diárias por mês}
%\rowcolors{2}{lightgray}{}%lightgray
\begin{tabular}{lcccrcr}
\toprule
%\hline\hline
%   \rowcolor{lgray}
	Tipo                         & Mês & Localidade & Finalidade               & Custo (R\$) & Quantidade & Total    \\
	\midrule
	Passagem aérea nacional      & 12  & a definir  & Apresentação de trabalho & 3.300,00    & 02         &  6.600,00 \\
	Diárias (nacional)           & 12  & a definir  & Manutenção em viagem     &   320,00    & 10         &  3.200,00 \\
	Passagem aérea nacional      & 24  & a definir  & Apresentação de trabalho & 3.300,00    & 02         &  6.600,00 \\
	Diárias (nacional)           & 24  & a definir  & Manutenção em viagem     &   320,00    & 10         &  3.200,00 \\
	Passagem aérea internacional & 24  & a definir  & Apresentação de trabalho & 7.000,00    & 01         &  7.000,00 \\
	Diárias (internacional)      & 24  & a definir  & Manutenção em viagem     & 2.220,00    & 07         & 15.540,00 \\
\midrule
 %  \rowcolor{lgray}
	                             &     &            &                          &             & Total      & 42.140,00 \\
\bottomrule
\end{tabular}
	\label{tab:viagem}
\end{table}
 
\subsection{Geral}
Nesta seção, apresentamos um resumo do orçamento. Na Tabela~\ref{tab:geral-mes-tipo} encontra-se o desembolso mensal para cada tipo de item de orçamento.
A Tabela~\ref{tab:geral-ano-rubrica}, por sua vez, exibe o investimento anual por rubrica. Nesse caso, a coluna \emph{Custeio} reúne \emph{Terceiros}, \emph{Consumo}, e \emph{Passagens/Diárias}.
\begin{table}[!h]
\scriptsize
	\caption{Desembolso mensal por tipo}
\rowcolors{2}{lightgray}{}%lightgray
\centering
\begin{tabular}{lcrrrrrr}
\toprule
%\hline\hline
   \rowcolor{lgray}
	Ano  & Mês      & Bolsas  & Terceiros  & Consumo & Pass/Diárias & Capital    & Valor \\
\midrule
       Ano 1 & 01       &  11.000,00 &      0,00 & 650,00  &     0,00     & 60.529,36 &  72.179,36\\
             & 02       &  11.000,00 &      0,00 &   0,00  &     0,00     &      0,00 &   11.000,00\\
             & 03       &  11.000,00 &      0,00 &   0,00  &     0,00     &      0,00 &   11.000,00\\
             & 04       &  11.000,00 &  2.000,00 &   0,00  &     0,00     &      0,00 &   13.000,00\\
             & 05       &  11.000,00 &      0,00 &   0,00  &     0,00     &      0,00 &   11.000,00\\
             & 06       &  11.000,00 &      0,00 &   0,00  &     0,00     &      0,00 &   11.000,00\\
             & 07       &  11.000,00 &  3.000,00 &   0,00  &     0,00     &      0,00 &   14.000,00\\
             & 08       &  11.000,00 &  3.000,00 &   0,00  &     0,00     &      0,00 &   14.000,00\\
             & 09       &  11.000,00 &  3.000,00 &   0,00  &     0,00     &      0,00 &   14.000,00\\
             & 10       &  11.000,00 &  3.000,00 &   0,00  &     0,00     &      0,00 &   14.000,00\\
             & 11       &  11.000,00 &  3.000,00 &   0,00  &     0,00     &      0,00 &   14.000,00\\
             & 12       &  11.000,00 &  3.000,00 &   0,00  & 9.800,00     &      0,00 &   23.800,00\\
\midrule
	     & Subtotal & 132.000,00 & 20.000,00 & 650,00  & 9.800,00     & 60.529,36 & 222.979,36\\
%   \rowcolor{lightblue}%lgray}
\midrule
\midrule
   \rowcolor{lgray}
	Ano  & Mês      & Bolsas  & Terceiros  & Consumo & Pass/Diárias & Capital    & Valor \\
\midrule
       Ano 2 & 13       &  11.000,00 &  3.000,00 & 650,00  &      0,00   & 64.070,55 &  78.720,55\\
             & 14       &  11.000,00 &  3.000,00 &   0,00  &      0,00   &      0,00 &  14.000,00\\
             & 15       &  11.000,00 &  3.000,00 &   0,00  &      0,00   &      0,00 &  14.000,00\\
             & 16       &  11.000,00 &  3.000,00 &   0,00  &      0,00   &      0,00 &  14.000,00\\
             & 17       &  11.000,00 &  3.000,00 &   0,00  &      0,00   &      0,00 &  14.000,00\\
             & 18       &  11.000,00 &  3.000,00 &   0,00  &      0,00   &      0,00 &  14.000,00\\
             & 19       &  11.000,00 &      0,00 &   0,00  &      0,00   &      0,00 &   11.000,00\\
             & 20       &  11.000,00 &  2.000,00 &   0,00  &      0,00   &      0,00 &   11.000,00\\
             & 21       &  11.000,00 &      0,00 &   0,00  &      0,00   &      0,00 &   11.000,00\\
             & 22       &  11.000,00 &      0,00 &   0,00  &      0,00   &      0,00 &   11.000,00\\
             & 23       &  11.000,00 &      0,00 &   0,00  &      0,00   &      0,00 &   11.000,00\\
             & 24       &  11.000,00 & 10.500,00 &   0,00  & 32.340,00   &      0,00 &  53.840,00\\
\midrule
	     & Subtotal & 132.000,00 & 28.500,00 & 650,00  & 32.340,00   & 64.070,55 & 257.560,55\\
%   \rowcolor{lgray}
\bottomrule
%\hline\hline   
\end{tabular}
	\label{tab:geral-mes-tipo}
\end{table}

\begin{table}[!h]
%\scriptsize
	\caption{Desembolso anual por rubrica}
%\rowcolors{2}{lightgray}{}%lightgray
\centering
\begin{tabular}{crrrr}
\toprule
%\hline\hline
   %\rowcolor{lgray}
	Ano   & RH         & Custeio   & Capital    & Total \\
\midrule
	Ano 1 & 132.000,00  & 30.450,00 & 60.529,36  & 222.979,36\\
	Ano 2 & 132.000,00  & 61.490,00 & 64.070,55  & 257.560,55\\
\midrule
	Total & 264.000,00 & 91.940,00 & 124.599,91 & 480.539,91\\
\bottomrule
%\hline\hline   
\end{tabular}
	\label{tab:geral-ano-rubrica}
\end{table}
\clearpage
%
